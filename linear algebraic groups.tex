\documentclass[10pt,a4paper,twoside,openany,hidelinks]{book}

%JAPANESE
\usepackage{fontspec}
\defaultfontfeatures{Ligatures={NoCommon, NoDiscretionary, NoHistoric, NoRequired, NoContextual}}
\setmainfont{CMU Serif}
\usepackage{xeCJK}
\xeCJKsetup{CJKmath=true}
\setCJKmainfont{MS Mincho} % or some other Japanese font

\usepackage{maths}
\usepackage{stylish}
\DeclareMathOperator{\Ad}{Ad}

%\usepackage{tikz-graph}
%\GraphInit[vstyle = Shade]

\title{Lecture Notes to Linear Algebraic Groups \\ \large{Winter 2021-2022, Technion IIT}}
\author{Typed by Elad Tzorani}
\date{\today}

\usepackage{lipsum}
\usepackage{stmaryrd}

\tikzset{EdgeStyle/.style   = {thick,
                               double          = orange,
                               double distance = 1pt}}

\tikzset{
    labl0/.style={anchor=south, rotate=-30, inner sep=.5mm}
}

\tikzset{
    labl1/.style={anchor=south, rotate=30, inner sep=.5mm}
}

\tikzset{
    labl2/.style={anchor=south, rotate=60, inner sep=.5mm}
}

\tikzset{
    labl3/.style={anchor=south, rotate=90, inner sep=.5mm}
}

\tikzset{
    labl4/.style={anchor=south, rotate=-90, inner sep=.5mm}
}

\usepackage{scalerel,stackengine}

\begin{comment}
\stackMath
\newcommand\widecheck[1]{%
\savestack{\tmpbox}{\stretchto{%
  \scaleto{%
    \scalerel*[\widthof{\ensuremath{#1}}]{\kern-.6pt\bigwedge\kern-.6pt}%
    {\rule[-\textheight/2]{1ex}{\textheight}}%WIDTH-LIMITED BIG WEDGE
  }{\textheight}% 
}{0.5ex}}%
\stackon[1pt]{#1}{\scalebox{-1}{\tmpbox}}%
}
\parskip 1ex
\end{comment}

\newcommand{\from}{\leftarrow}

\usepackage[nottoc]{tocbibind}

\newcommand{\Ch}[1]{\catname{Ch}\catname{(mod-}#1\catname{)}}
\newcommand{\Chb}[1]{\catname{Ch}_{{b}}\catname{(mod-}#1\catname{)}}
\newcommand{\Chm}[1]{{\catname{Ch}_{{-}}\catname{(mod-}R\catname{)}}}
\newcommand{\Chp}[1]{{\catname{Ch}_{{+}}\catname{(mod-}R\catname{)}}}

\begin{document}
\frontmatter
\frontpage{front}{0.8\textwidth}{A cat.}
\tableofcontents
\countlectures
\newpage

\chapter*{Preface}
\addcontentsline{toc}{chapter}{Preface} \markboth{Preface}{}

\section*{Technicalities}
\addcontentsline{toc}{section}{Technicalities} %\markboth{Technicalities}{}

These aren't formal notes related to the course and henceforward there is \emph{absolutely no guarantee} that the recorded material is in correspondence with the course expectations, or that these notes lack any mistakes.\\
In fact, there probably are mistakes in the notes! I would highly appreciate if any comments or corrections were sent to me via email at \href{mailto:tzorani.elad@gmail.com}{tzorani.elad@gmail.com}.\\
Elad Tzorani.

\section*{Grade}

The course grade will consist of the following.

\begin{itemize}
\item 60\% for homework
\item 40\% for giving lectures on more advanced topics at the end of the semester
\end{itemize}

\mainmatter

\chapter{Linear Algebraic Groups}

\section{Preliminaries}

\subsection{Motivation \& Historical Background}

\subsubsection{Linear Algebraic Groups From Differential Equations}

Algebraic groups developed from the study of Lie groups. The latter were studied by Sophus Lie around 1870 in the context of differential equations. Lie groups can describe symmetries of solutions of differential equations; e.g. solutions of $\nabla y = 0$ are \emph{harmonic functions} and one is interested in linear isomorphisms $g \colon \mbb{R}^n \to \mbb{R}^n$ such that $\Delta\prs{y} = 0$ implies $\Delta\prs{y \circ g} = 0$. Lie noticed that such $g$ form a group $\mcal{O}_n\prs{\mbb{R}} \coloneqq \set{g \in \mrm{GL}_n\prs{\mbb{R}}}{g^T g = I_n}$.
Such groups for operators different than $\Delta$ are smooth manifolds with smooth group actions, called \emph{Lie groups.}
One of Lie's motivation was to have Galois theory for differentiable equations. It had already been known that in order to find roots of polynomials one uses the symmetries of field extensions.

Around 1880, Picard looked at differentiable equations of the form
\begin{align*}
\frac{\prs{\diff y}^n}{\diff x}^n + p_1\prs{x} \frac{\diff y^{n-1}}{\prs{\diff x}^{n-1}} + \ldots + p_n\prs{x} y = 0
\end{align*}
for $p_i$ rational functions.
The solution space for such an equation is the $n$-dimensional space
\[\spn\set{y_1\prs{x}, \ldots, y_n\prs{x}} \text{.}\]
Picard looked at a subgroup $G \leq \mrm{GL}_n\prs{\mbb{R}}$
which preserves the algebraic dependencies of the $y_i$ (i.e. preserves polynomials $p \in \mbb{R}_n\brs{x}$ for which $p\prs{y_1\prs{x}, \ldots, y_n\prs{x}} = 0$). These were the first treatments of algebraic groups.

Around 1870-1900, Mauren took homogeneous rational functions $f \colon \mbb{C}^n \to \mbb{C}$ (such as $f\prs{x_1, \ldots, x_n} = \sum_{i=1}^n x_i^2$ for which $G_f = \mcal{O}_n\prs{\mbb{C}}$) and studied the structure of
\[G_f \coloneqq \set{g \in \mrm{GL}_n\prs{\mbb{C}}}{f \circ g = f} \text{.}\]
One can take $f$ to be any quadratic form, e.g.
\[\mrm{Sp}_{2n}\prs{\mbb{C}} = \set{g \in \mrm{GL}_{2n}\prs{\mbb{C}}}{g^T \pmat{0 & I_n \\ -I_n & 0} g = \pmat{0 & I_n \\ -I_n & 0}}\]
is such  group.
Such groups are called \emph{classical groups}.

Mauren looked at the tangent space of such group. His motivation was his interest in Hilbert's 14\textsuperscript{th} problem:
Given $g \in G \leq \mrm{GL}_n\prs{\mbb{C}}$ we can consider $g$ as a map $\mbb{C}^n \to \mbb{C}^n$. Considering the action $G \acts \mbb{C}\brs{x_1, \ldots, x_n}$ on the coefficient, the problem is understanding the invariant space of this action.
E.g. the invariant space of $S_n$ are \emph{symmetric polynomials}.

\subsubsection{Later Developments}

The field of Lie groups gave great success. Semisimple Lie groups have complete combinatorical classification due to Cartan and Killing. This is considered one of the greatest achievements in mathematics.

Chevalley found out the every semisimple Lie group is defined by polynomials in integer coefficients, circa 1940. This led to the definition of algebraic groups and a new goal: to algebrize Lie theory and develop tools to study smooth symmetric ``without analysis'' and over more general fields. This should form a bridge between continuous groups and finite groups.
Chevalley used in his studies of the subject the formal expression
\[\mrm{exp}\prs{X} = \sum_{n=0}^\infty \frac{1}{n!} X^n\]
and required $\mrm{char}\prs{\mbb{F}} = 0$.
Later Kolchin returned to Picard's ideas and developed a differential Galois theory over a general field.

\subsubsection{Modern Developments}

From 1950 onwards, many mathematicians developed the study of algebraic groups, which was possible thanks to advances in algebraic geometry.
Some of the advances of the field are the following.

\begin{enumerate}
\item The classification of finite simple groups. Most of these groups are of ``Lie type'', which are of the form $\mrm{Sp}_{2n}\prs{\mbb{F}_q}$.

\item Results on \href{https://ncatlab.org/nlab/show/p-adic+completion}{$p$-adic} groups. For example, Bruhat-Tits buildings are homogeneous spaces with $p$-adic group actions and which are ``non-archimedean'' analogues to classical symmetric spaces.

\item Results in number theory.
\end{enumerate}

\subsubsection{The Langlands Program}

The Laglands program, circa 1960, tries to study properties in number theory through the study of groups. There are analogues to Riemann's zeta function which one hopes all arise from group actions in the following way.
Taking an algebraic group $G$, one looks at \emph{automorphic spaces} $V$ with $G\prs{\mbb{R}}$ and $G\prs{\mbb{Q}_p}$ actions which commute with each other, for some groups $G\prs{\mbb{R}}, G\prs{\mbb{Q}_p}$ over the respective fields.

\subsection{Definitions \& Course Goals}

\subsubsection{What are Algebraic Groups?}

Write $\mbb{F}$ for a field, and write $M_n\prs{\mbb{F}} \cong \mbb{F}^{n^2}$ for the space of $n \times n$ matrices over $\mbb{F}$.

\begin{definition}[Affine Algebraic Group]
A subset $G \leq M_n\prs{\mbb{F}}$ closed under multiplication and inverse is called an \emph{affine algebraic group} over $\mbb{F}$ if there are $f_1, \ldots, f_k \in \mbb{F}\brs{\set{x_{i,j}}_{i,j \in [n]}}$ such that
\[G = \set{A \in M_n\prs{\mbb{F}}}{f_1\prs{A} = \ldots = f_k\prs{A} = 0} \text{.}\]
\end{definition}

\begin{example}
$\mrm{SL}_n\prs{\mbb{F}}$ is an affine algebraic group. The determinant,
\[\det\prs{X} = \sum_{\sigma \in S_n} \prs{-1}^{\sgn\prs{\sigma}} x_{1,\sigma\prs{1}} \cdot \ldots \cdot x_{n,\sigma\prs{n}} \text{,}\]
is a polynomial and $\mrm{SL}_n\prs{\mbb{F}} = \set{A \in M_n\prs{\mbb{F}}{\det\prs{A} - 1 = 0}}$.
\end{example}

\begin{example}
Let $Q \in M_n\prs{\mbb{F}}$ and denote
\[\mcal{O}_{Q}\prs{\mbb{F}} \coloneqq \set{A \in \mrm{GL}_n\prs{\mbb{F}}}{A^t Q A = Q} \text{.}\]
Taking $Q = I_n$ one gets $\mcal{O}_Q\prs{\mbb{R}} = \mcal{O}_n\prs{\mbb{R}}$. More generally, matrix multiplication is polynomial and one can write $A^t Q A - Q = 0$ as a polynomial equation in the coefficients of $A$. We explain the condition $A \in \mrm{GL}_n\prs{\mbb{F}}$ later.
\end{example}

\begin{example}
Let $N \subseteq M_n\prs{\mbb{F}}$ be the subset of upper-triangular matrices with $1$ on the diagonal. This is an algebraic group with polynomial conditions $x_{i,j} = 0$ for $i > j$ and $x_{i,i} = 1$ for all $i \in [n]$.
One has $N \cong \mbb{F}^{\frac{n\prs{n-1}}{2}}$ as vector space, but this doesn't remember the group structure.
\end{example}

\begin{example}
The vector space $\mbb{F}^n$ with addition is an algebraic group.
We have
\begin{align*}
V \coloneqq \set{\pmat{1 & 0 & \cdots & 0 & * \\ 0 & & & & * \\ \vdots & & \ddots & & \vdots \\ & & & & * \\ 0 & & \cdots & & 1} \in M_{n+1}\prs{\mbb{F}}} \cong \mbb{F}^n \text{.}
\end{align*}
One denotes $G_a\prs{\mbb{F}} \coloneqq \prs{\mbb{F}, +}$ and calls this \emph{the additive group over $\mbb{F}$}.
\end{example}

\begin{remark}
One has
\begin{align*}
\mrm{GL}_n\prs{\mbb{F}} = \set{A \in M_n\prs{\mbb{F}}}{\det\prs{A} \neq 0}
\\&\cong \set{\pmat{A & 0 \\ 0 & a} \in M_{n+1}\prs{\mbb{F}}}{\det\prs{A} \cdot a = 1}
\end{align*}
and a bijection $A \leftrightarrow \pmat{A & 0 \\ 0 & \det\prs{A}^{-1}}$, but this looks weird. We then want the definition to be more general and capture groups that are isomorphic to what we defined as affine algebraic groups. We do that later in the course.
\end{remark}

\begin{example}
Every finite group $G$ is an algebraic group. One has an inclusion $G \rmono S_n$, and $S_n$ is an algebraic group where $\sigma$ is considered as $\prs{x_{i,j}}_{i,j \in [n]}$ with $x_{i,\sigma\prs{i}} = 1$ and $x_{i,j} = 0$ for any other $i,j \in [n]$.
\end{example}

\begin{exercise}
Every finite subset of $M_n\prs{\mbb{F}}$ is an algebraic set, in the sense that it's defined by the vanishing of polynomials.
\end{exercise}

To study properties of algebraic groups, one needs to use tools from algebraic geometry. Here there are two possible difficulties:

\begin{enumerate}
\item One needs to ask what generality is to be worked with. With our current definition it is difficult to use strong algebro-geometric tools, but with ``too general'' definitions it is more difficult to look at simple examples.

\item One should decide how much they want to rely on geometric results as facts and how much is to be proved.
\end{enumerate}

Our answer to the latter question is proving things at the beginning of the course and later on taking more things as facts. For the first difficulty\ldots you'll see as we go.

\subsubsection{A Course Overview}

During the course we plan to go over the following.

\begin{itemize}
\item Basic algebraic geometry.

\item General structure properties of algebraic groups. For example, a generalization of Jordan's decomposition to $\mrm{GL}_n$.

\item Generalization of the notion of an algebraic group.

\item Study of algebraic groups by looking at algebraic groups over the Galois closure and via Galois theory.

\item The classification of reductive groups over algebraically closed fields. An algebraic version of the Cartan-Killing classification.
\end{itemize}

\subsection{Preliminary Algebraic Geometry}

\subsubsection{Embedded $\mbb{F}$-Affine Varieties}

\begin{notation}
Denote $A_n \coloneqq \mbb{F}\brs{x_1, \ldots, x_n}$.
\end{notation}

\begin{definition}
For $C \subseteq A_n$ define
\[V\prs{C} \coloneqq \set{p \in \mbb{F}^n}{\forall f \in C \colon f\prs{p} = 0} \subseteq \mbb{F}^n \text{.}\]
A set of this form is called an \emph{embedded $\mbb{F}$-Affine Variety}.
\end{definition}

\begin{definition}
For $S \subseteq \mbb{F}^n$ define
\begin{align*}
I\prs{S} \coloneqq \set{f \in A_n}{\forall p \in S \colon f\prs{p} = 0} \text{.}
\end{align*}
\end{definition}

\begin{exercise}
For $S \subseteq \mbb{F}^n$ one has $I\prs{S} \ideal A_n$.
\end{exercise}

\begin{example}
One has $I\prs{\ns} = A_n$ and whenever $\mbb{F}$ is infinite one has $I\prs{\mbb{F}^n} = \set{0}$.
\end{example}

\begin{example}
For $I = \set{x_1^2 - x_2, x_1^3 - x_3}$ one has $V\prs{I} = \set{\prs{x, x^2, x^3}}{x \in \mbb{F}}$ which one calls the \emph{twisted cubic} over $\mbb{F}$
\end{example}

\begin{proposition}
One notices that for $S \subseteq \mbb{F}^n$ and $C \subseteq A_n$ we have
\begin{align*}
S &\subseteq V\prs{I\prs{S}} \\
C \subseteq I\prs{V\prs{C}} \text{.}
\end{align*}
\end{proposition}

\begin{definition}[The Zariski Topology]
The \emph{Zariski topology} on $\mbb{F}^n$ is the topology given by taking sets of the form $V\prs{C}$ for $C \subseteq A_n$ as the closed subsets.
\end{definition}

\begin{exercise}
Check that the above definition gives a well-defined topology.
\end{exercise}

\begin{exercise}
For $S \subseteq \mbb{F}^n$ one has $\overline{S} = V\prs{I\prs{S}}$.
\end{exercise}

\begin{example}
Consider the case $n=1$. Then closed subsets of $\mbb{F}$ are sets of the form $V\prs{C}$ for $C \subseteq A_n$. If $C$ contains a nonzero polynomial, $V\prs{C}$ is finite, and otherwise $V\prs{C} = \mbb{F}$. We get that the nontrivial closed sets are exactly the finite subsets of $\mbb{F}^n$.
\end{example}

\begin{remark}
$\mbb{F}^n$ with the Zariski topology is always \emph{quasi-compact}, meaning it's compact but not Hausdorff.
\end{remark}

\begin{theorem}[Hilbert's Basis Theorem]\label{theorem:basis_theorem}
Every ideal $I \ideal A_n$ is finitely-generated.
\end{theorem}

\begin{proof}
We prove the statement by induction on $n \in \mbb{N}$. The case $n = 0$ is trivial since $\mbb{F}$ is a field. Assume the statement is true for $n-1$, we show it for $n$.
Write $A_n \cong A_{n-1}\brs{x_n}$ and assume $I \ideal A_n$ is nonzero. Choose $f_1 \in I \setminus \set{0}$ of minimal degree and write $d_1 \coloneqq \deg_{A_{n-1}}{f_1}$. If $\prs{f_1} \neq I$, choose $f_2 \in I\setminus\prs{f_1}$ of minimal degree $d_2 \coloneqq \deg_{A_{n_1}}\prs{f_2}$. Continue this way to get $f_i$ with $d_i \coloneqq \deg_{A_{n-1}}\prs{f_i}$ and $d_1 \leq d_2 \leq d_3 \leq \ldots $.

Assume that this doesn't end a finite point (for otherwise we're done). Denote by $a_i \in A_{n-1}$ the leading coefficient of $f_i$. By assumption, $I' \coloneqq \prs{a_1, \ldots, a_i, \ldots} \ideal A_{n-1}$ is finitely-generated. We can then write $I' = \prs{a_1, \ldots, a_h}$ for some $h \in \mbb{N}$. Then
\[a_{h+1} = x_1 a_1 + \ldots + x_h a_h\]
for some $x_1, \ldots, x_h \in A_{n-1}$.
Let
\[g \coloneqq f_{h+1} - \sum_{i \in [h]} x_i \cdot f_i \cdot x^{d_{h+1} - d_i} \in I \text{.}\]
The coefficient of $x^{d_{h+1}}$ in $g$ vanishes so $\deg\prs{g} < d_{h+1}$ and $g \in \prs{f_1,\ ldots, f_h}$. Then also $f_{h+1} \in \prs{f_1, \ldots, f_h}$, in contradiction.
\end{proof}

\begin{definition}[Noetherian Topological Space]
A topological space $X$ is called \emph{Noetherian} if every decreasing sequence of closed subsets stabilises.
\end{definition}

\begin{corollary}
$\mbb{F}^n$ with the Zariski topology is Noetherian.
\end{corollary}

\begin{proof}
Let $\prs{X_i}_{i \in \mbb{N}_+} \subseteq \mbb{F}^n$ be a decreasing sequence of closed subsets, and for every $i$ denote $I_i \coloneqq I\prs{X_i}$. Then $X_i = V\prs{I_i}$. We get $I_1 \subseteq I_2 \subseteq I_3 \subseteq \ldots$. Denote $I = \bigcup_{i \in \mbb{N}_+} I_i$.  By Theorem \ref{theorem:basis_theorem} we have $I = \prs{f_1, \ldots, f_k}$ for some $k \in \mbb{N}_+$. Then $f_1, \ldots, f_k \in I_m$ for some $m \in \mbb{N}_+$. We get $I = I_m = I_{m+1} = \ldots$ and $X_m = X_{m+1} = X_{m+2} = \ldots$, as required.
\end{proof}

\begin{exercise}
Every closed subspace $X \subseteq \mbb{F}^n$ is quasi-compact.
\end{exercise}

\begin{example}
Consider $xy \in \mbb{F}\brs{x,y}$. $V\prs{x,y}$ is connected, but we would like to say it has two components. E.g. if $\mbb{F} = \mbb{R}$, the set $V\prs{x,y}$ is the union of two perpendicular axes. This leads to the following definition.
\end{example}

\begin{definition}[Irreducible Topological Space]
A topological space $X$ is \emph{irreducible} if there aren't strict closed subsets $X_1, X_2 \subsetneq X$ such that $X = X_1 \cup X_2$.
\end{definition}

\begin{exercise}
An irreducible Hausdorff topological space is a point.
\end{exercise}

\begin{exercise}
In a Noetherian space $X$ there are finitely many maximal irreducible subsets $X_1, \ldots, X_k$, and $X = \bigcup_{i \in [k]} X_i$.
\end{exercise}

\begin{proposition}
An algebraic variety $V \subseteq \mbb{F}^n$ is irreducible if and only if $I\prs{V}$ is prime.
\end{proposition}

\begin{proof}
Assume $V$ is irreducible. Let $f_1, f_2 \in A_n$, such that $f_1 f_2 \in I\prs{V}$, we want to show $f_1 \in I\prs{V}$ or $f_2 \in I\prs{V}$. We have $V \subseteq V\prs{f_1 f_2} = V\prs{f_1} \cup V\prs{f_2}$. Now $V = \prs{V \cap V\prs{f_1}} \cup \prs{V \cap V\prs{f_2}}$ and by irreducibility $V = V \cap V\prs{f_i}$ for $i \in [2]$, in which case $V \subseteq V\prs{f_i}$ and therefore $f_i \in I\prs{V}$.

The other direction is left as an exercise.
\end{proof}

%LECTURE 03.11.2021

\begin{example}
Consider $G \coloneqq \mrm{GL}_1\prs{\mbb{F}} \cong \mbb{F}^\times \subseteq \mbb{F}$.
We have $V\prs{x-1} = \set{1}$ and similarly $V\prs{\prs{x-1}^2} = \set{1}$. If $\mbb{F} = \mbb{R}$, one has $V\prs{x^5 - 1} = \set{1}$ and one gets $V\prs{x^5 - 1} = V\prs{x-1}$.
However, over $\mbb{F} = \mbb{C}$ the group $V\prs{x^5 - 1}$ is the roots of unity of order $5$.
\end{example}

\begin{example}
Let $V \subseteq \mbb{F}^n$ be closed, and for $f \in A_n$ define
\begin{align*}
V_f \coloneqq \set{x \in V}{f\prs{x} \neq 0} = V \setminus V\prs{f} \text{.}
\end{align*}
This is open in $V$ and such a set is called a \emph{principal open set}.
Every open set $U$ is a finite union of such sets, so the principal open sets form a basis for the Zariski topology:

If $U$ is open in $V$ let $W \coloneqq V \setminus U$ so that there are $\prs{f_i}_{i \in [k]} \subseteq A_n$ for which
\begin{align*}
V \setminus U = V \cap W = V\prs{f_1, \ldots, f_k}
\end{align*}
so
\begin{align*}
U = \bigcup_{i \in [k]} V_{f_i} \text{.}
\end{align*}

We sometimes want to think of $V_f$ as closed sets. This can be done by considering
\begin{align*}
\tilde{V}_f \coloneqq \set{\prs{v,y}}{\substack{v \in \mbb{F}^n \\ y \in \mbb{F} \\ f\prs{v} \cdot y = 1}} \subseteq \mbb{F}^{n+1} \text{.}
\end{align*}
There's a clear bijection $V_f \riso \tilde{V}_f$.
\end{example}

\subsubsection{Regular Maps}

\begin{definition}[Regular Map]
For embedded algebraic varieties $V \subseteq \mbb{F}^n$ and $W \subseteq \mbb{F}^m$, a \emph{morphism} $\phi \colon V \to W$, called also a \emph{regular map} is a map of the form
\[\phi\prs{x} = \prs{f_1\prs{x}, \ldots, f_m\prs{x}}\]
for $\prs{f_i}_{i \in \brs{m}} \subseteq A_n$.
\end{definition}

\begin{example}
The map
\begin{align*}
\phi \colon \mbb{F}^2 &\to \mbb{F}^2 \\
\prs{x,y} &\mapsto \prs{xy,y} 
\end{align*}
is a regular map.
\end{example}

\begin{exercise}
A regular map is continuous in the Zariski topology.
\end{exercise}

\begin{definition}
A \emph{regular function} on an embedded algebraic variety $V$ over $\mbb{F}$ is a regular map $V \to \mbb{F}$.
\end{definition}

\begin{remark}
Regular functions on $V$ are of the form $\rest{f}{V}$ for $f \in A_n$. We can think of these as elements of $\mbb{F}\brs{V} \coloneqq A_n / I\prs{V}$.
\end{remark}

\begin{definition}
A regular map $\phi \colon V \to W$ gives an $\mbb{F}$-algebra homomorphism
\begin{align*}
\phi^* \colon \mbb{F}\brs{W} &\to \mbb{F}\brs{V} \\
f &\mapsto f \circ \phi \text{.}
\end{align*}
\end{definition}

\begin{remark}
Sending every $V$ to $\mbb{F}\brs{V}$ and every $\phi$ to $\phi^*$ is a contravariant functor from the category of embedded algebraic varieties to that of finite-dimensional $\mbb{F}$-algebras.
\end{remark}

\begin{exercise}
\begin{enumerate}
\item Consider regular maps
\begin{align*}
\phi_1,\phi_2 \colon V \to W
\end{align*}
such that $\phi_1^* = \phi_2^*$. Show that $\phi_1 = \phi_2$.

\item Show that if $\phi \colon V \to W$ is a regular map such that $\phi^*$ is an isomorphism, $\phi$ is also an isomorphism.
\end{enumerate}
\end{exercise}

\begin{exercise}
Let $\phi \colon V \to W$ be a regular map, and assume $V$ is irreducible. Show that $\overline{\phi\prs{V}}$ is irreducible. 
\end{exercise}

\subsubsection{Revising Affine $\mbb{F}$-Varieties}

For a set $X$ one can consider the algebra of functions $M \coloneqq \hom_{\mathbf{Set}}\prs{X,\mbb{F}}$. Every $x \in X$ defines a homomorphism $\ev_x \colon M \to \mbb{F}$ given by $\ev_x\prs{f} = f\prs{x}$.

\begin{definition}[Affine $\mbb{F}$-Variety]
An \emph{affine $\mbb{F}$-variety} is a pair $\prs{X,A}$ where $X$ is any set and $A \subseteq \hom_{\mathbf{Set}}\prs{X,\mbb{F}}$ such that the following conditions hold.

\begin{enumerate}
\item $A$ is finite-generated.
\item The map
\begin{align*}
X &\to \hom_{\mbb{F}\mathbf{-Alg}}\prs{A, \mbb{F}} \\
x &\mapsto \rest{\ev_x}{A}
\end{align*}
is a bijection.
\end{enumerate}
\end{definition}

%LECTURE 4.11.2021

\begin{remark}
This new definition of an affine $\mbb{F}$-variety is coordinate-free and gives good results even for non-algebraically-closed fields.
\end{remark}

\begin{remark}
We could take any finitely-generated $\mbb{F}$-algebra $A$ and define $X = \hom_{\mbb{F}\mathbf{-Alg}}\prs{A,\mbb{F}}$. This would given an homomorphism $A \to \hom_{\mathbf{Set}}\prs{X,\mbb{F}}$ taking $f \in A$ to the map $x \mapsto x\prs{f}$.
If we require that this map is injective, we get an equivalent definition to that of an affine $\mbb{F}$-variety.
\end{remark}

\begin{definition}[Regular Maps]
Let $\prs{X,A}$ and $\prs{Y,B}$ be affine $\mbb{F}$-varieties. A \emph{regular map} $\phi \colon \prs{X,A} \to \prs{Y,B}$ is a map of sets $\phi \colon X \to Y$ such that for every $f \in B$ it holds that $f \circ \phi \in A$. 
\end{definition}

\begin{definition}[Regular Isomorphism]
A regular map $\phi \colon \prs{X,A} \to \prs{Y,B}$ is an \emph{isomorphism} if there's $\psi \colon \prs{Y,B} \to \prs{X,A}$ such that $\phi \circ \psi = \id$ and $\psi \circ \phi = \id$.
\end{definition}

\begin{remark}
Every homomorphism $\alpha \colon B \to A$ gives a regular map $\phi \colon X \to Y$ by sending $x$ to the $y \in Y$ for which $\ev_x \circ \alpha = \ev_Y$.
\end{remark}

\begin{proposition}
Embedded $\mbb{F}$-affine varieties are $\mbb{F}$-affine varieties.
\end{proposition}

\begin{proof}
Let $V \subseteq \mbb{F}^n$ be an embedded $\mbb{F}$-affine variety. We claim $\prs{V, \mbb{F}\brs{V}}$ is an affine $\mbb{F}$-variety.

$\mbb{F}\brs{V} = A_n/I\prs{V}$ is finitely-generated, so we have to show that all maps $\hom_{\mbb{F}\mathbf{-Alg}}\prs{\mbb{F}\brs{V}, \mbb{F}}$ are of the form $\ev_x$ (since $\mbb{F}\brs{V}$ separate points).
Let $\eps \colon \mbb{F}\brs{V} \to \mbb{F}$, we find $x \in V$ such that $\eps = \ev_x$. Consider the quotient map $\pi \colon A_n \to \mbb{F}\brs{V}$. Then $\eps \circ \pi \colon A_n \to \mbb{F}$. Let $x = \prs{\eps \circ \pi \prs{x_i}}_{i \in [n]} \in \mbb{F}^n$. Now $\ker\prs{\eps \circ \pi}$ is a maximal ideal contained in $I\prs{x}$ so
\begin{align*}
I\prs{V} = \ker\prs{\pi} \subseteq \ker\prs{\eps \circ \pi} = I\prs{x}
\end{align*}
so $x \in V$.
\end{proof}

\begin{remark}
Let $\prs{X,A}$ be an affine $\mbb{F}$-variety. Write $A \cong A_n / I$ and $\pi \colon A_n \riso A_n/I$ the quotient map, and define a map
\begin{align*}
\phi \colon X &\to \mbb{F}^n \\
x &\mapsto \prs{\ev_x \circ \pi\prs{x_i}}_{i \in \brs{n}} \text{.}
\end{align*}
Now $I = \ker\prs{\pi} \subseteq \ker\prs{\ev_p}$ for all $p \in \phi\prs{X}$, so $I \subseteq I\prs{\phi\prs{X}}$.
In the other direction, if $f \in I\prs{\phi\prs{X}}$ then $\ev_x\prs{\pi\prs{f}}$ for all $x \in X$. Therefore $\pi\prs{f} = 0$ so $I\prs{\phi\prs{X}} \subseteq \ker\prs{\pi} = I$.
Then $A \cong A_n / I\prs{\phi\prs{X}} = \mbb{F}\brs{\phi\prs{X}}$.
Then
$\prs{X,A} \cong \prs{\phi\prs{X}, \mbb{F}\brs{\phi\prs{X}}}$.

We want $\phi\prs{X}$ to be closed, which we explain later.
\end{remark}

\begin{exercise}
Check that $\phi$ in the above remark is injective.
\end{exercise}

\begin{definition}[Zariski Topology on an Affine $\mbb{F}$-Variety]
Let $\prs{X,A}$ be an affine $\mbb{F}$-variety. Define the \emph{Zariski topology} on $\prs{X,A}$ by choosing the closed sets to be sets of the form
\[V\prs{C} \coloneqq \set{x \in X}{\forall f \in C \colon f\prs{x} = 0}\]
for $C \subseteq A$.
\end{definition}

\begin{definition}
Let $\prs{X,A}$ be an affine $\mbb{F}$-variety. For $S \subseteq X$ we define
\begin{align*}
I\prs{S} \coloneqq \set{f \in A}{\forall x \in X \colon f\prs{x} = 0} \ideal A \text{.}
\end{align*}
\end{definition}

\begin{remark}
As before, if $Y \subseteq X$ is closed, we have $Y = V\prs{I\prs{Y}}$. Then $Y$ is itself an affine $\mbb{F}$-variety as $\prs{Y, A/I\prs{Y}}$ where $A/I\prs{Y}$ is considered as embedded in $\hom_{\mathbf{Set}}\prs{Y,\mbb{F}}$ by considering the restriction to $Y$.
Elements of
$\hom_{\mbb{F}\mathbf{-Alg}}\prs{A/I\prs{Y}, \mbb{F}}$
are elements of $\hom\prs{A,\mbb{F}} \cong X$ that vanish on $I\prs{Y}$. These are exactly $V\prs{I\prs{Y}} \cong Y$.
\end{remark}

\begin{definition}[Closed Embedding]
A regular map $\phi \colon \prs{X,A} \to \prs{Y,B}$ is called a \emph{closed embedding} if $\im\phi$ is closed and $\rest{\phi}{X}$ is a regular isomorphism.
\end{definition}

\begin{remark}
Requiring that $\phi$ is injective would not suffice. Consider
\begin{align*}
\phi \colon \mbb{F}_p &\to \mbb{F}_p \\
x &\mapsto x^p \text{.}
\end{align*}
We have
\begin{align*}
\phi^* \colon \mbb{F}_p\brs{x} &\to \mbb{F}_p\brs{x} \\
x &\mapsto x^p \text{,}
\end{align*}
which isn't surjective. Hence $\phi$ isn't an isomorphism.
\end{remark}

\begin{proposition}
Let $\phi \colon \prs{X,A} \to \prs{Y,B}$. $\phi$ is a closed embedding if and only if $\phi^* \colon \mbb{F}\brs{Y} \to \mbb{F}\brs{X}$ is surjective.
\end{proposition}

\begin{proof}
\begin{itemize}
\item Assume $\phi$ is a closed embedding. Consider the inclusion $i \colon \phi\prs{X} \to Y$ and write $\phi = i \circ \phi_0$. Then $\phi^* = \phi_0^* \circ i^*$. $\phi_0^*$ is an isomorphism, hence $\phi_0$ is an isomorphism. We're left to show that $i^*$ is surjective.
$\phi\prs{X} \subseteq Y$ is closed and $i^*$ is the quotient map from $\mbb{F}\brs{Y}$ to $\mbb{F}\brs{\phi\prs{X}}$, hence we get the result.

\item
Assume $\phi^*$ is surjective. We show that $\phi\prs{X} = V\prs{\ker\prs{\phi^*}}$ and $\ker\prs{\phi^*} = I\prs{\phi\prs{X}}$.

\begin{description}
\item[$\phi\prs{X} = V\prs{\ker\prs{\phi^*}}$:]
Let $f \in \ker\prs{\phi^*} \subseteq \mbb{F}\brs{Y}$. For every $x \in X$ we have
\[\ev_{\phi\prs{x}}\prs{f} = \ev_x \prs{\phi^*\prs{f}} = \ev_x\prs{0} = 0\text{.}\]
Hence $f \in I\prs{\phi\prs{X}}$. Hence $\ker\prs{\phi^*} \subseteq I\prs{\phi\prs{X}}$ so
$\phi\prs{X} \subseteq V\prs{\ker\prs{\phi^*}}$.

In the other direction, let $y \in V\prs{\ker\prs{\phi^*}}$. Then $\rest{\ev_y}{\ker\prs{\phi^*}} = 0$. Then $\ev_y \colon \mbb{F}\brs{Y} \to \mbb{F}$ factors through $\lambda \colon \mbb{F}\brs{X} \to \mbb{F}$ where $\lambda = \ev_y \circ \phi^*$. Hence there's $x \in X$ such that $\lambda = \ev_x$. Hence $y = \phi\prs{x} \in \phi\prs{X}$ so $V\prs{\ker\prs{\phi^*}} \subseteq \phi\prs{X}$.

\item[$\ker\prs{\phi^*} = I\prs{\phi\prs{X}}$:]
We saw one inclusion in the previous part. Let $f \in I\prs{\phi\prs{X}}$, we have to show $f \in \ker\prs{\phi^*}$. Indeed, $\phi^*\prs{f} = f \circ \phi = 0$.
\end{description}
\end{itemize}
\end{proof}

\section{Algebraic Groups}

\subsection{Definitions}

In order to define affine algebraic groups, we want the product and inverse maps to be regular. For that, we need to define the product variety. This structure comes from tensor products of $\mbb{F}$-algebras.

Let $\prs{X,A}, \prs{Y,B}$ be affine $\mbb{F}$-varieties. We have an embedding
\begin{align*}
\iota \colon A \otimes B &\to \hom_{\mathbf{Set}}\prs{X \times Y, F} \\
f \otimes g &\mapsto \prs{\prs{x,y} \mapsto f\prs{x} \cdot g\prs{y}} \text{.}
\end{align*}
We show this is injective.
Let $f_1, \ldots, f_k \in A$ linearly independent and $g_1, \ldots, g_\ell \in B$ linearly independent. Using properties of tensor products, we show that for $F = \sum_{\substack{i \in [k] \\ j \in [\ell]}} a_{i,j} f_i \otimes g_j$ such that $\iota\prs{F} = 0$ it holds that $F = 0$.
Indeed, for $y \in Y$ define
\[b_i\prs{y} = \sum_{j \in \brs{\ell}} a_{i,j} g_j\prs{y} \text{.}\]
Then
\[0 = \iota\prs{F} \prs{x,y} = \sum_{i \in \brs{k}} b_i\prs{y} f_i\prs{x}\]
for all $x \in X$. Hence
\[\sum_{i \in [k]} b_i\prs{y} f_i = 0\]
so for every $i \in [k]$ we have $b_i\prs{y} = 0$. Hence
\begin{align*}
0 = b_i = \sum_{j \in \brs{\ell}} a_{i,j} g_j \in B
\end{align*}
so $a_{i,j} = 0$ for every $i \in \brs{k}$ and $j \in \brs{\ell}$.

\begin{definition}[$\mbb{F}$-Algebraic Group]
An \emph{$\mbb{F}$-algebraic group} $G$ is an $\mbb{F}$-affine variety such that the product and inverse maps are regular.
\end{definition}

%LECTURE 10.11.2021

\begin{exercise}
Let $\prs{X,A}$ be an affine $\mbb{F}$-variety. Let $f \in A$, and define the \emph{principle open set}
\[X_f \coloneqq \set{v \in X}{f\prs{v} \neq 0} \subseteq X \text{.}\]
Show that $\prs{X_f, A\brs{\frac{1}{f}}}$ is an affine $\mbb{F}$-variety.
\end{exercise}

\begin{remark}
\begin{enumerate}
\item Consider $\set{0} \subseteq A_n$ and $X = V\prs{\set{0}} = \mbb{F}^n$. Then $I\prs{X} = \prs{0}$ if and only if $\mbb{F}$ is infinite. This is equivalent to $\mbb{F}\brs{X} \cong A_n$.
In particular, when $\mbb{F}$ is finite, $\prs{\mbb{F}, \mbb{F}\brs{x}}$ isn't an affine $\mbb{F}$-variety in our sense.

\item
Consider $\mbb{F} = \overline{\mbb{F}_p}$ and take $X = \mbb{F}$. We have
\begin{align*}
\mbb{F}\brs{X} = \mbb{F}\brs{X}
\end{align*}
and the map
\begin{align*}
\phi \colon \mbb{F} &\to \mbb{F} \\
v &\mapsto v^p
\end{align*}
is bijective (check this).
The map
\begin{align*}
\phi^* \colon \mbb{F}\brs{x} &\to \mbb{F}\brs{x} \\
x &\mapsto x^p
\end{align*}
isn't surjective, so $\phi$ isn't an isomorphism!
\end{enumerate}
\end{remark}

Our current goals are the following.

\begin{enumerate}
\item Every embedded $\mbb{F}$-group is an $\mbb{F}$ algebraic group.

\item Every $\mbb{F}$ algebraic group has an embedding into $\mbb{F}^n$ for some $n \in \mbb{N}_+$.
\end{enumerate}

\begin{proposition}
Every embedded $\mbb{F}$-group is an algebraic group.
\end{proposition}

\begin{proof}
Firstly we notice that a restriction of a regular map to a closed subset is regular. Hence, if $G$ is an algebraic group and $H \leq G$ is closed, $H$ is algebraic.
It therefore suffices to show that $\mrm{GL}_n\prs{\mbb{F}}$ is an $\mbb{F}$-algebraic group.

Now, \[\mrm{GL}_n\prs{\mbb{F}} = \set{A \in M_n\prs{\mbb{F}}}{\det\prs{A} \neq 0} = \prs{\mbb{F}^{n^2}}_{\det}\]
is a principle open set in $\mbb{F}^{n^2}$. We have
\begin{align*}
\mbb{F}\brs{\mrm{GL}_n\prs{\mbb{F}}} = \mbb{F}\brs{\bigcup_{i,j \in [n]} \set{X_{i,j}, \frac{1}{\det\prs{X_{i,j}}}}} \text{.}
\end{align*}
Let $G \coloneqq \mrm{GL}_n\prs{\mbb{F}}$ and $m \colon G \times G \to G$ be the multiplication map.
Then
\begin{align*}
m^*\prs{X_{i,j}} = \sum_{k\in[n]} X_{i,k} \otimes X_{k,j} \in \mbb{F}\brs{G\times G} \cong \mbb{F}\brs{G} \otimes \mbb{F}\brs{G} \text{.}
\end{align*}
From multiplicativity of $\det$ we get
\begin{align*}
m^*\prs{\frac{1}{\det\prs{X_{i,j}}}} = \frac{1}{\det\prs{X_{i,j}}} \otimes \frac{1}{\det\prs{X_{i,j}}} \text{.}
\end{align*}

Let $i \colon G \to G$ be the inverse map. Cramer's rule described the inverse of a matrix as a polynomial by the coefficients, hence the inverse map is also regular.
\end{proof}

\begin{definition}The [Multiplicative Group]
Define \emph{the multiplicative group} over $\mbb{F}$ as $\mbb{G}_m\prs{\mbb{F}} \coloneqq \mrm{GL}_1\prs{\mbb{F}}$.
\end{definition}

\begin{exercise} \label{exercise:irreducible_product}
The product of irreducible algebraic varieties is irreducible.
\end{exercise}

\begin{proposition}
Let $G$ be an $\mbb{F}$ algebraic group. Let $G^\circ \subseteq G$ be the irreducible component of $G$ containing the identity element $e$ of $G$. Then

\begin{enumerate}
\item $G^\circ \ideal G$ is a normal subgroup of $G$ of finite index.

\item Every closed subgroup $H \leq G$ of finite index contains $G^\circ$.

\item $G = G^\circ$ if and only if $G$ is connected. \label{connected_group_component}
\end{enumerate}
\end{proposition}

\begin{proof}
\begin{enumerate}
\item By exercise \ref{exercise:irreducible_product}, $G^\circ \times G^\circ \subseteq G \times G$ is irreducible. Hence $e = e \cdot e \in \overline{G^\circ \cdot G^\circ}$. Since $G^\circ \subseteq \overline{G^\circ \cdot G^\circ}$ we get $G^\circ = \overline{G^\circ \cdot G^\circ}$. Hence $G^\circ$ is closed to multiplication.
Now, $g \mapsto g^{-1}$ is a homeomorphism so $\prs{G^\circ}^{-1}$ is irreducible.
Now $\overline{G_0 \cdot \prs{G_0}^{-1}}$ is irreducible containing $e$ so one gets $G_0 = \overline{G_0 \cdot \prs{G_0}^{-1}} = \prs{G_0}^{-1}$.

For every $x \in G$, the map $\lambda_x \colon G \to G$ given by $x \mapsto xy$ is a homeomorphism. Every coset $xG^\circ$ is an irreducible component of $G$ and $\bigsqcup_{x \in G} x G^\circ$. In an affine variety, there are finitely many irreducible components, hence $G^\circ$ has finitely many cosets in $G$. This proves part \ref{connected_group_component} as well.

We're left to show that $G^\circ$ is normal.
Indeed, conjugation $y \mapsto xyx^{-1}$ is a homeomorphism. For every $x \in G$, $x G^\circ x^{-1}$ is an irreducible component. Hence $e \in G^\circ x^{-1}$ implies $G_0 = x G_0 x^{-1}$.

\item
If $H \leq G$ is a closed subgroup of finite index, we can write $G = \bigcup_{x \in G} xH$ which is a finite disjoint union of the different cosets.
Then
$G^\circ = \bigcup_{x \in G} \prs{xH \cap G^\circ}$.
From irreducibility there's $x \in G$ such that $G^\circ = xH \cap G^\circ$. Then $e \in G^\circ \subseteq xH = H$.
\end{enumerate}
\end{proof}

%LECTURE 11.11.21

\begin{example}
$\mrm{GL}_1\prs{\mbb{R}}$ isn't connected as a Lie group (with the Euclidean topology from $\mbb{R}$, but \emph{is} connected as an algebraic group.
\end{example}

\begin{example}
Finite non-trivial groups aren't connected, since $G^\circ = \set{1}$.
\end{example}

\begin{example}
Consider the group
\[\mrm{O}_n\prs{\mbb{F}} \coloneqq \set{M_n\prs{\mbb{F}}}{g^t Q g = Q}\]
where $Q = \pmat{0 & & 1  \\ & \rddots & \\ 1 & & 0}$.
For $g \in O_n\prs{\mbb{F}}$ we get
\begin{align*}
\det\prs{g}^2 \det\prs{Q} &= \det\prs{g^t Q g}
\\&=
\det\prs{Q}
\end{align*}
so $\det\prs{g} \in \set{\pm 1}$.
$\det \colon \mrm{O}_n\prs{\mbb{F}} \to \set{\pm 1}$ is a homomorphism.
Now, $\mrm{SO}_n\prs{\mbb{F}} \coloneqq \mrm{O}_n\prs{\mbb{F}} \cap \mrm{SL}_n\prs{\mbb{F}}$ is a subgroup of index $2$ (or $1$). For this, it suffices to show that there's $g \in \mrm{O}_n\prs{\mbb{F}}$ such that $\det\prs{g} = -1$, which is the cast when $1+1 \neq 0$ (otherwise, the index is $1$).
Usually, $\mrm{SO}_n\prs{\mbb{F}} = \mrm{O}_n\prs{\mbb{F}}^\circ$. When $n=2$ and $\mbb{F}$ is algebraically-closed we get
\begin{align*}
\mrm{GL}_1\prs{\mbb{F}} \cong \mrm{SO}_2\prs{\mbb{F}} = \set{\pmat{a & 0 \\ 0 & a^{-1}}}{a \in \mbb{F} \setminus \set{0}}
\end{align*}
which is connected. Then there's \[g = \pmat{0 & 1 \\ 1 & 0} \in \mrm{O}_2\prs{\mbb{F}} \setminus \mrm{SO}_2\prs{\mbb{F}}\text{.}\]
\end{example}

\subsection{Embedding Algebraic Groups}

\subsubsection{Group Actions}

We want to show that every $\mbb{F}$-algebraic group is isomorphic to a closed subgroup of $\mrm{GL}_n\prs{\mbb{F}}$. To every $G$, we later find a rational representation $\phi \colon G \to \mrm{GL}_n\prs{\mbb{F}}$ (i.e. with $\phi$ regular) such that $\phi$ is a closed embedding.
For this, we look at actions of algebraic groups.

\begin{definition}[$\mbb{F}$-Affine $G$-Space]
Let $\prs{X,A}$ be an affine $\mbb{F}$-variety. We say $X$ is an \emph{$\mbb{F}$-affine $G$-space} if it has a regular map $G \times X \to X$ for an $\mbb{F}$-affine algebraic group $G$.
\end{definition}

\begin{example}
Take $X = G$. Then $G$ acts on $X$ by left multiplication $x \mapsto gx$, by right multiplication $x \mapsto x g^{-1}$ or by conjugation $x \mapsto gxg^{-1}$.
\end{example}

For $X$ an $\mbb{F}$-affine $G$-space, we get an action of $G$ on $\mbb{F}\brs{X}$.
Let $f \in \mbb{F}\brs{X}$ and $g \in G$. We have an action
\begin{align*}
\prs{gf} \prs{x} = f\prs{g^{-1} \cdot x} \text{.}
\end{align*}
The map
\begin{align*}
\mbb{F}\brs{X} &\to \mbb{F}\brs{x} \\
f &\mapsto gf
\end{align*}
is $\mbb{F}$-linear.
We get a map
\begin{align*}
G \to \mrm{GL}\prs{\mbb{F}\brs{X}} \text{.}
\end{align*}
Writing the $G$-action as
\begin{align*}
a \colon G \times X &\to X
\end{align*}
we get
\begin{align*}
a^* \colon \mbb{F}\brs{X} &\to \mbb{F}\brs{G} \otimes \mbb{F}\brs{X} \text{.}
\end{align*}
Now,
\begin{align*}
g \circ f = \prs{\ev_{g^{-1}} \otimes \id}\prs{a^*\prs{f}} \text{.}
\end{align*}

\begin{proposition}
\begin{enumerate}
\item If $V \subseteq \mbb{F}\brs{X}$ is a finite-dimensional $G$-invariant subspace, then
\begin{align*}
a^*\prs{V} \subseteq \mbb{F}\brs{G} \otimes V \text{,}
\end{align*}
and the resulting homomorphism $\phi \colon G \to \mrm{GL}\prs{V}$ is regular (given any choice of basis for $V$).

\item For every finite-dimensional $W \subseteq \mbb{F}\brs{X}$ there's $V \subseteq \mbb{F}\brs{X}$ which contains $W$ and is $G$-invariant.
\end{enumerate}
\end{proposition}

\begin{proof}
\begin{enumerate}
\item Let $V \subseteq \mbb{F}\brs{X}$ be finite-dimensional and $G$-invariant.
Choose $\set{r_1, r_2, \ldots} \subseteq \mbb{F}\brs{X}$ linearly independent such that $\mbb{F}\brs{X} = V \oplus \spn\set{r_1, r_2, \ldots}$.
For $f \in V$ we can write
\[a^*\prs{f} = s + \sum_{i \in [k]} u_i \otimes r_{j_i}\]
for $s \in \mbb{F}\brs{G} \otimes V$ and $u_i \in \mbb{F}\brs{G}$.
Since $g^{-1}\cdot f \in V$ we get
\begin{align*}
\prs{\ev_v \otimes \id} \prs{a^*\prs{f}} \in V
\end{align*}
so
\begin{align*}
\sum_{i \in [k]} u_i\prs{g} r_{j_i} = 0 \in \mbb{F}\brs{X}
\end{align*}
for all $g \in G$.
By linear independency we get $u_i\prs{g} = 0$ for all $i$, so $u_i \equiv 0$.
Hence $a^*\prs{f} \in \mbb{F}\brs{G} \otimes V$.

Choose a basis $\prs{f_1, \ldots, f_n}$ for $V$. We get $a^*\prs{f_i} = \sum_{j \in [n]} m_{j,i} f_j$ for $m_{i,j} \in \mbb{F}\brs{G} \otimes V$.
If we look at the action of $G$ on $V$ and write the map $\phi \colon G \to \mrm{GL}_n\prs{\mbb{F}}$ given by choosing the basis $\prs{f_1, \ldots, f_n}$, we get
\[\phi\prs{g} = \prs{m_{j,i} \prs{g^{-1}}}\]
and
\[g \cdot f_i = \sum_{j \in [k]} m_{j,i} \prs{g^{-1}} f_j \text{.}\]
Then, the matrix coefficients of $\phi\prs{g}$ are given by regular functions.

\item Take a basis $\prs{h_1, \ldots, h_k}$ of $W$ and examine the finite sum
$a^*\prs{h_i} = \sum_j u_j \otimes f_{i_j}$.
We have
\[g \cdot h_i = \sum_j u_j\prs{g^{-1}} f_{i,j} \in \mbb{F}\brs{X} \text{.}\]
Take $V' = \spn\set{f_{i,j}}_{i,j} \subseteq \mbb{F}\brs{X}$ which is finite-dimensional.
Then
\[V = \spn\set{g \cdot f}{\substack{g \in G \\ f \in \mbb{F}\brs{X}}} \subseteq V'\]
is a $G$-invariant subspace.
\end{enumerate}
\end{proof}

\begin{proposition}
Let $G$ be an $\mbb{F}$-algebraic group. There's a regular homomorphism $\phi \colon G \to \mrm{GL}_n\prs{\mbb{F}}$ which is a closed embedding.
\end{proposition}

\begin{proof}
Consider the right action of $G$ on itself. Take generators $f_1, \ldots, f_k \in \mbb{F}\brs{G}$ of the $\mbb{F}$-algebra $\mbb{F}\brs{G}$. From the previous proposition, there's a finite-dimensional $G$-invariant $V \subseteq \mbb{F}\brs{G}$ under the induced action on $\mbb{F}\brs{G}$, which contains each $f_i$.

Choose a basis $\prs{e_1, \ldots, e_n}$ of $V$. We get a homomorphism $\phi \colon G \to \mrm{GL}_n\prs{\mbb{F}} \cong \mrm{GL}\prs{V}$ as in the previous proposition. We want to show that
\begin{align*}
\phi^* \colon \mbb{F}\brs{\mrm{GL}_n\prs{\mbb{F}}} \to \mbb{F}\brs{G}
\end{align*}
is surjective.
We may assume $\phi\prs{g} = \prs{m_{j,i}\prs{g^{-1}}}$ for $m_{i,j} \in \mbb{F}\brs{G}$ for which
\begin{align*}
\forall x,g \in G \colon e_i\prs{xg} = \sum_j m_{j,i}\prs{g^{-1}} e_j\prs{x} \text{.}
\end{align*}
Taking $x = e$ we get
\begin{align*}
e_i\prs{g} &= \sum_j m_{j,i} e_j\prs{e} m_{j,i}\prs{g^{-1}} \text{.}
\end{align*}
Write $\tilde{m}_{i,j}\prs{g} = m_{j,i}\prs{g^{-1}}$.
Then
\[e_i = \sum_j e_j\prs{e} \tilde{m}_{j,i} \in \mbb{F}\brs{G}\]
but
$\tilde{m}_{i,j} = \phi^*\prs{T_{i,j}}$.
Hence $e_i \in \im\prs{\phi}$.
Hence $f_j \in \im\prs{\phi^*}$ and these generate $\mbb{F}\brs{G}$ so $\im\prs{\phi^*} = \mbb{F}\brs{G}$.
\end{proof}

Consider a homomorphism $\phi \colon G \to H$ of $\mbb{F}$-algebraic groups. The kernel $\ker\prs{\phi} = \phi^{-1}\prs{e} \leq G$ is a closed subgroup of $G$. A question we would like to ask whether or not $\im\prs{\phi}$ is closed. This happens to be true when $\mbb{F}$ is algebraically-closed, but requires use of an intricate result in algebraic geometry.

\begin{remark}
In Lie groups, the situation is different than that mentioned above. Consider the map
\begin{align*}
\phi \colon \mbb{R} &\to \mrm{GL}_4\prs{\mbb{R}} \\
\theta &\mapsto \pmat{\cos \theta & \sin \theta  & 0 & 0 \\ -\sin \theta & \cos \theta & 0 & 0 \\ 0 & 0 & \cos\prs{\alpha \theta} & \sin\prs{\alpha \theta} \\ 0 & 0 & -\sin\prs{\alpha \theta} & \cos\prs{\alpha \theta}} \text{.}
\end{align*}
If $\alpha / 2 \pi$ is irrational, this is an embedding of $\mbb{R}$ which isn't closed. Moreover, $\im\phi$ has an empty interior.
\end{remark}

We later show that if $\phi \colon X \to Y$ is regular between $\mbb{F}$-affine varieties and $\mbb{F}$ is algebraically-closed, there's an open $U \subseteq \overline{\phi\prs{X}}$ such that $U \subseteq \phi\prs{X}$.

\begin{example}
Consider $X \coloneqq \set{\prs{x,y} \in \mbb{F}^2}{xy = 1}$ and the map
\begin{align*}
\phi \colon X &\to \mbb{F} \\
\prs{x,y} &\mapsto x \text{.}
\end{align*}
The image $\im\prs{\phi} = \mbb{F} \setminus \set{0}$ is dense, non-closed and open.
\end{example}

\begin{example}
Consider
\begin{align*}
\phi \colon \mrm{GL}_1\prs{\mbb{R}} \to \mrm{GL}_1\prs{\mbb{R}} \\
x &\mapsto x^2 \text{.}
\end{align*}
We have $\im\prs{\phi} = \set{y \in \mbb{R}}{y > 0}$ which is (Zariski) dense in $\mrm{GL}_1\prs{\mbb{R}}$ but doesn't contain an open set.
\end{example}

\begin{definition}[Dominant Regular Map]
Let $\phi \colon X \to Y$ be a regular map between $\mbb{F}$-affine varieties. We say $\phi$ is \emph{dominant} if $\overline{\phi\prs{X}} = Y$.
\end{definition}

\subsubsection{Algebraic Interlude}

\begin{proposition}\label{proposition:dominant-injective}
$\phi \colon X \to Y$ is dominant if and only if $\phi^* \colon \mbb{F}\brs{Y} \to \mbb{F}\brs{X}$ is injective.
\end{proposition}

\begin{proof}
Assume $\phi$ isn't dominant. We have $Z \coloneqq \overline{\phi\prs{X}} \subsetneq Y$.
Then there's $f \in \mbb{F}\brs{Y} \setminus 0$ such that $\rest{f}{Z} = 0$. Then $\phi^*\prs{f} = 0$ so $\ker\prs{\phi^*} \neq \prs{0}$ so $\phi^*$ isn't injective.

Assume $\phi^*$ isn't injective. Then $\ker\prs{\phi^*} \neq \prs{0}$. Then there's $f \in \mbb{F}\brs{Y} \setminus 0$ such that $\phi^*\prs{f} = 0$. Then $\phi\prs{X} \subseteq V\prs{f} \subsetneq Y$ so $\phi\prs{X}$ isn't dense, a contradiction.
\end{proof}

\begin{theorem}\label{theorem:nonempty_interior}
Let $\phi \colon X \to Y$ be a dominant regular map between $\mbb{F}$-affine varieties, for $\mbb{F}$ algebraically-closed.
Then $\im\prs{\phi}$ has non-empty interior.
\end{theorem}

An equivalent statement is the following.

\begin{theorem}
Let $\phi \colon X \to Y$ be a dominant regular map between $\mbb{F}$-affine varieties, for $\mbb{F}$ algebraically-closed. There's $f \in \mbb{F}\brs{Y} \setminus \set{0}$ such that $\ev_y\prs{f} \neq 0$ implies there's $x \in X$ for which $\ev_x \circ \phi^* = \ev_y$.
\end{theorem}

\begin{example}
$\phi \colon \mrm{GL}_1\prs{\mbb{F}} \to \mbb{F}$ sending $x \mapsto x$ gives a map $\phi^* \colon \mbb{F}\brs{x} \to \mbb{F}\brs{x,x^{-1}}$.
For $y \in \mbb{F} \setminus \set{0}$ we can lift $\ev_y$ to a homomorphism
\begin{align*}
\mbb{F}\brs{x,x^{-1}} &\to
f &\mapsto f\prs{y} \text{.}
\end{align*}
If we take $y=0$ we get
\begin{align*}
\ev_0 \colon \mbb{F}\brs{x} &\to \mbb{F} \\
f &\mapsto f\prs{0} \text{,}
\end{align*}
which cannot be lifted to $\mbb{F}\brs{x,x^{-1}}$. The polynomial $x \in \mbb{F}\brs{x}$ satisfies our required property.
We get that we can lift every $\ev_y$ with $\ev_y\prs{x} \neq 0$.
\end{example}

\begin{definition}[Reduced Ring]
A ring is \emph{reduced} if it has no non-zero nilpotent elements.
\end{definition}

\begin{lemma}
Let $\mbb{F}$ be an algebraically-closed field. Let $A \rmono B$ be a subring of a reduced ring $B$ (which is unital and commutative) such that $B$ is generated by $A$ and $t \in B$. Write $B \cong A\brs{t} / I$. Assume there are a homomorphism $\eps \colon A \to \mbb{F}$ and an element $f \coloneqq f_0 + f_1 t + \ldots + f_n t^n \in I$ such that $\eps\prs{f_n} \neq 0$.
Then there's $\eps' \colon B \to \mbb{F}$ such that $\rest{\eps'}{A} = \eps$.
\end{lemma}

\begin{proof}
Assume that $m$ is minimal for which $\eps\prs{f_m} \neq 0$. We prove the statement by induction on $m$.
Consider
\begin{align*}
\tilde{\eps} \colon A\brs{t} \to \mbb{F}\brs{t} \\
\sum_{i = 0}^k a_i t^i &\mapsto \sum_{i = 0}^k \eps\prs{a_i} t^i \text{.}
\end{align*}
If
$\trs{\tilde{\eps}\prs{I}} \neq \mbb{F}\brs{t}$, then $\prs{\tilde{\eps}\prs{I}} = \prs{p}$ for $p \in \mbb{F}\brs{t}$. Since $\mbb{F}$ is algebraically-closed, there's a root $\alpha \in \mbb{F}$ of $p$. Consider
$\eps' = \ev_\alpha \circ \tilde{\eps}$, we have $I \subseteq \ker\prs{\eps}$.

We prove that indeed $\trs{\tilde{\eps}\prs{I}} \neq \mbb{F}\brs{t}$ via contradiction. Otherwise, there is a polynomial
$g = \sum_{i=0}^n g_i t^i \in I$
for which $\deg \tilde{\eps}\prs{g} = 0$. (check this!)
Then $\eps\prs{g_0} \neq 0$ and $\eps\prs{g_i} = 0$ for all $i > 0$.
We have $A \leq B$ so $A \cap I = \set{0}$. Hence $n \geq 1$. We may assume $n<m$ by the following argument:

We ``divide'' $g$ by $f$. It can be shown that there are $q,r \in A\brs{t}$ for which $f_m^d g  = qf + r$ for some $d \geq 1$. Applying $\tilde{\eps}$ on the  equation we get
\begin{align*}
0 \neq \eps\prs{f_m}^d \cdot \eps\prs{g_0} = \tilde{\eps}\prs{q}\tilde{\eps}\prs{f} + \tilde{\eps}\prs{r} \text{.}
\end{align*}
Since the left-hand-side is of degree $0$, the right-hand side is a constant polynomial. But, $\deg \tilde{\eps}\prs{m} = 0$, so $\tilde{\eps}\prs{q} = 0$. Also $r = f_m^d \cdot g - q \cdot f \in I$. We therefore may consider $r$ instead of $g$ as it satisfies the same conditions. This shows that case $m = 1$.

We now show the induction step. Assume $m>1$. For $h = \sum_{i=0}^s h_i t^i \in A\brs{t}$ we define
\[\tilde{h} = \sum_{i=0}^s h_{s-i} t^i \text{.}\]
As a function $\tilde{h}\prs{t} = \prs{t^s} \circ h \circ \prs{t^{-1}}$ which is an involution reminding of those in complex function theory.
Consider the ideal
\[\tilde{I} \coloneqq \trs{\set{\tilde{h}}{h \in I}} \ideal A\brs{t} \text{.}\]
We have
$\tilde{g} \in \tilde{I}$ and $\eps\prs{g_0} \neq 0$. Define
\[\bar{A} = A / \prs{A \cap \tilde{I}}\]
and
\[\tilde{B} = A\brs{t} / \tilde{I} \text{,}\]
so that there's an inclusion
$\bar{A} \rmono \bar{B}$.
For $h \in I$ with $\tilde{h} \in \tilde{I} \cap A$ we have $h = a \cdot t^s$ for some $a \in $. Then $\prs{at}^s = a^{s-1}\prs{at^s} \in I$. Since $B$ has no nilpotent elements we get that $at \in I$.
Since $m > 1$ we get $\eps\prs{a} = 0$.
Now, $\tilde{h} = a$ so $\eps\prs{\tilde{h}} = 0$, so $\eps\prs{A \cap \tilde{I}} = 0$. We get that $\eps$ factors via
$\bar{\eps} \colon \bar{A} \to \mbb{F}$.
Since $n<m$, and by the induction step (check that $\bar{B}$ is reduced), there's $\bar{\eps}' \colon \bar{B} \to \mbb{F}$ such that $\rest{\bar{\eps}'}{\bar{A}} = \bar{\eps}$.
We have
\[\bar{\eps}'\prs{\tilde{g}} = \eps\prs{g_0} \cdot \bar{\eps}\prs{t}^n \text{,}\]
but since $\tilde{g} \in \tilde{I}$ we have $\bar{\eps}'\prs{\tilde{g}} = 0$. So, $\bar{\eps}'\prs{t}^n$.
Hence
\begin{align*}
0&=
\bar{\eps}' \prs{\tilde{f}} \\&= \sum_{i = 0}^m \eps\prs{f_{m-i}} \bar{\eps}'\prs{t}^i
\\&= \eps\prs{f_m} \neq 0 \text{,}
\end{align*}
a contradiction.
\end{proof}

\begin{example}
In the previous example, take $A \coloneqq \mbb{F}\brs{x}$ and $B \coloneqq \mbb{F}\brs{x,x^{-1}}$. Then $B \cong A\brs{t} / \prs{xt - 1}$. Then $f\prs{t} = -1+xt$ is the element in the lemma.
\end{example}

\begin{corollary}
Let $\phi \colon G \to H$ be a homomorphism of algebraic groups over an algebraically-closed field $\mbb{F}$. Then $\im\prs{\phi}$ is closed.
\end{corollary}

A generalization of the lemma is the following.

\begin{proposition}\label{proposition:going_up}
Let $\mbb{F}$ be an algebraically-closed field. Let $B$ be a finitely-generated $\mbb{F}$-algebra, which is an integral domain. Let $A \leq B$.
Then for every $b \in B \setminus \set{0}$ there's $a \in A \setminus \set{0}$ such that for every homomorphism $\eps \colon A \to \mbb{F}$ with $\eps\prs{a} \neq 0$ there is a homomorphism $\eps' \colon B \to \mbb{F}$ with $\rest{\eps'}{A} = \eps$ and $\eps\prs{b} \neq 0$.
\end{proposition}

\begin{proof}
Since $B$ is finitely-generated, we can write
\[A = A_0 \subsetneq A_1 \subsetneq \ldots \subsetneq A_k = B\]
with $A_i \cong A_{i-t}\brs{t} / I_i$.
By induction, it suffices to prove the case $k=1$.
Write $B \cong A\brs{t} / I$ and pick $b \in A\brs{t} / I$ non-zero. Let $h \in A\brs{t}$ with quotient image $\bar{h} = b$.

\begin{itemize}
\item If $I = \prs{0}$ we have $B = A\brs{t}$, in which case we define
\begin{align*}
\tilde{\eps} \colon A\brs{t} \to \mbb{F}\brs{t} \\
\sum_{i=0}^n a_i t^i &\mapsto \sum_{i = 0}^n \eps\prs{a_i} t^i \text{.}
\end{align*}
Now, $b \neq 0$ so $h \coloneqq \sum_{i = 0}^n h_i t^i \neq 0$. Then there's $i$ such that $h_i \neq 0$. Take $a = h_i$. If $\eps\prs{a} \neq 0$ we have $\tilde{\eps} \in \mbb{F}\brs{t} \setminus 0$. Then there's $\alpha \in \mbb{F}$ such that $\tilde{\eps}\prs{h}\prs{\alpha} \neq 0$. Define $\eps' = \ev_\alpha \circ \tilde{\eps}$.
Then $\eps'\prs{b} = \eps'\prs{h} \neq 0$. 

\item %LECTURE 18.11.2021
Take $f = \sum_{i = 0}^m f_i t^i \in I$ of minimal degree. $B$ is an integral domain, so $I$ is prime so $f$ is irreducible. Since $b \neq 0$ we have $h \notin I$. Hence $f \nmid h$. By working over the field of fractions and then multiplying by a common denominator, there are $u,v \in A\brs{t}$ such that $uf + vh = a'$ for some $a' \in A$. Here we have $vh \equiv a' \mod{I}$.
Take $a \coloneqq f_M \cdot a' \in A \setminus \set{0}$. If $\eps \colon A \to \mbb{F}$ is homomorphism such that $\eps\prs{a} \neq 0$, we have $\eps\prs{a'}, \eps\prs{f_m} \neq 0$. From the lemma it follows that there's $\eps' \colon B \to \mbb{F}$ such that $\rest{\eps'}{A} = \eps$. Now,
\[0 \neq \eps\prs{a'} = \tilde{\eps}\prs{v} \tilde{\eps}\prs{h} = \tilde{\eps}\prs{v} \cdot \eps'\prs{b}\]
where $\tilde{\eps} = \eps' \circ \pi$. Hence $\eps\prs{b} \neq 0$.
\end{itemize}
\end{proof}

\begin{proof}[\ref{theorem:nonempty_interior}]
\begin{itemize}
\item If $X$ is irreducible, so is $Y$ and we get $\phi^* \colon \mbb{F}\brs{X} \rmono \mbb{F}\brs{X}$. $X$ is irreducible, hence $\mbb{F}\brs{X}$ is an integral domain. From \ref{proposition:going_up} it follows that there's $f \in \mbb{F}\brs{Y} \setminus \set{0}$ such that for $\eps \colon \mbb{F}\brs{Y} \to \mbb{F}$ satisfying $\eps\prs{f} \neq 0$ there is a lift $\eps' \colon \mbb{F}\brs{x} \to \mbb{F}$.

For $y \in Y$ with $\ev_y\prs{f} = f\prs{y} \neq 0$ we can find $\eps' \colon \mbb{F}\brs{X} \to \mbb{F}$ lifting $\ev_y$. But, $\eps' = \ev_x$ for some $x \in X$. We get that $\ev_y = \ev_x \circ \phi^*$, which is equivalent to $\phi\prs{x} = y$. Then $V_f \subseteq \phi\prs{X}$.

\item Assume $X = \bigcup_{i \in [s]} X_i$ is a decomposition of $X$ to irreducible components. It follows from the irreducible case that there are open sets $U_1, \ldots, U_s \subseteq Y$ such that $\overline{\phi\prs{X_i}} \cap U_i \subseteq \phi\prs{X_i}$ for all $i \in [s]$. Write
$U \coloneqq \bigcap_{i \in [s]} U_i'$ for $U_i' = U_i \cup \prs{Y \setminus \overline{\phi\prs{X_i}}}$. For every $i \in [s]$ we get
\[\overline{\phi\prs{X_i}} \cap U \subseteq \overline{\phi\prs{X_i}} \cap U_i' \subseteq \phi\prs{X_i} \text{.}\]
Also, $U \subseteq \bigcup_{i \in [s]} \phi\prs{X_i} = \phi\prs{X}$.
Check that $U \neq \ns$ by using the irreducibility of one of the $X_i$.
\end{itemize}
\end{proof}

\subsubsection{Back to Algebraic Groups}

\begin{proposition}
Let $G$ be an $\mbb{F}$-algebraic group and let $H \leq G$ be an abstract subgroup. Then
\begin{enumerate}
\item $\bar{H} \leq G$ is a subgroup.
\item If $H$ contains a non-empty open subset of $\bar{H}$ then $H = \bar{H}$.
\end{enumerate}
\end{proposition}

\begin{proof}
\begin{enumerate}
\item For $x \in H$ we have
\[H = x^{-1} H \subseteq x^{-1} \bar{H} \text{.}\]
Hence $xH \subseteq \bar{H}$ for all $x$. Since $g \mapsto x^{-1} g$ is a homeomorphism we get $x\bar{H} \subseteq \bar{H}$. Hence $H \cdot \bar{H} \subseteq \bar{H}$.
For $y \in \bar{H}$ we have $Hy \subseteq \bar{H}$ s $H \subseteq \bar{H} y^{-1}$. Hence $\bar{H} \subseteq \bar{H} y^{-1}$ hence $\bar{H} y \subseteq \bar{H}$, so $\bar{H} \cdot \bar{H} \subseteq \bar{H}$.

Since $g \mapsto g^{-1}$ is a homeomorphism we also get $\bar{H}^{-1} = \overline{H^{-1}} = \bar{H}$.

\item Assume $U \subseteq H$ is open in $\bar{H}$. Then $H = \bigcup_{x \in H} xU$ is open in $\bar{H}$ as a union of open sets. For every $y \in \bar{H}$ we get $yH \cap H$ is an intersection of dense open subsets of $\bar{H}$, which is therefore non-empty.
Hence there are $x_1, x_@ \in H$ such that $y x_1 = x_2$. Then $y = x_2 x_1^{-1} \in H$.
\end{enumerate}
\end{proof}

\begin{proposition}
If $\phi \colon G \to H$ is a regular homomorphism of algebraic groups, $\phi\prs{G}$ is a closed subgroup of $H$.
\end{proposition}

\begin{proof}
$\phi\prs{G} \leq H$ is a subgroup, and from \ref{theorem:nonempty_interior}, $\phi\prs{G}$ contains an open subset of $\overline{\phi\prs{G}}$. From the previous proposition we get that $\phi\prs{G}$ is closed.
\end{proof}

\subsection{Jordan Decomposition}

Let $\mbb{F}$ be an algebraically-closed field. It follows from Jordan's theorem that for $A \in M_n\prs{\mbb{F}}$ one can write
\[A = P^{-1} D P + PNP\]
for $D$ diagonal and $N$ nilpotent.
Denote $A_s \coloneqq P^{-1} D P$ which is then semi-simple, and $A_n \coloneqq P^{-1} N P$ which is nilpotent.
We have $A_s A_n = A_n A_s$.
We call $A = A_n + A_s$ the \emph{Jordan-Chevalley decomposition} for $A$.

\begin{definition}
We call $g \in \mrm{GL}\prs{V}$ \emph{unipotent} if $g - I$ is nilpotent.
\end{definition}

\begin{remark}
$g$ is unipotent iff its only eigenvalue is $1$.
\end{remark}

For $g \in \mrm{GL}\prs{V}$, write $g = g_s + g_n$ for its Jordan-Chevalley decomposition. Let $g_u \coloneqq I + g_s^{-1} g_n$. We have
\[g = g_s g_u = g_u g_s \text{.}\]

We want to show that for an algebraic group $G \leq \mrm{GL}\prs{V}$ and $g \in G$ we have $g_s g_u \in G$ and that $g_s, g_u$ are independent of the embedding of $G$.

\begin{remark}
In fact, if $\mbb{F}$ is a perfect field, one can consider $A \subseteq M_n\prs{\mbb{F}} \subseteq M_n\prs{\overline{\mbb{F}}}$. One gets that $A_s, A_n \in M_n\prs{\mbb{F}}$ which requires a proof. The rest of our statements in this section will work for perfect fields.
\end{remark}

\begin{proposition}
Let $A \in M_n\prs{\mbb{F}}$. There is $p \in \mbb{F}\brs{t}$ for which $A_s = p\prs{A}$ and $A_n = \prs{1-p}\prs{A}$.
\end{proposition}

\begin{proof}
Write $p_A\prs{t} = \prod_{i \in [k]} \prs{t - \lambda_i}^{r_a\prs{\lambda_i}}$ for the characteristic polynomial of $A$ where $\prs{\lambda_i}_{i \in [k]}$ are the different eigenvalues of $A$.
It suffices to find a polynomial $p \in \mbb{F}\brs{t}$ such that for every $i \in [k]$ there's $q_i \in \mbb{F}\brs{t}$ for which
\[p\prs{t} = q_i\prs{t} \prs{t-\lambda_i}^{r_a\prs{\lambda_i}} + \lambda_i \text{.}\]
This is  equivalent to 
\[\rest{L_{p\prs{A}}}{\ker\prs{\prs{L_A - \lambda_i \id}^n}} = \lambda_i I = \rest{L_{A_s}}{\ker\prs{\prs{L_A - \lambda_i \id}^n}}\]
where
\begin{align*}
L_A \colon \mbb{F}^n &\to \mbb{F}^n \\
v &\mapsto Av \text{.}
\end{align*}
Since $\prs{t - \lambda_i}^{r_a\prs{\lambda_i}}$ are coprime, it follows from the Chinese Remainder Theorem that such $p$ exists.
\end{proof}

%LECTURE 24.11

\begin{proposition}
Let $a \in \End\prs{V}$. There are polynomials $p_s, p_u  \in \mbb{F}\brs{x}$ such that $a_s = p_s\prs{a}$ and $a_n = p_n\prs{a}$.
\end{proposition}

\begin{corollary}
Let $a \in \End\prs{V}$ (or $g \in \mrm{GL}\prs{V}$). The decomposition $a = a_s + a_n$ with $a_s$ semi-simple and $a_n$ nilpotent such that $a_s a_n = a_n a_s$ (or $g = g_s g_u$ such that $g_s$ is semi-simple and $g_u$ is unipotent such that $g_s g_u = g_u g_s$), is unique.
\end{corollary}

\begin{proof}
Write $a = a_s + a_n = b_s + b_n$ where $a_s, b_s$ are semi-simple, $a_n,b_n$ are nilpotent and each of the pairs $a_s, a_n$ and $b_s, b_n$ commutes. Write $a_s = p_s\prs{a}$ for a polynomial $p_s \in \mbb{F}\brs{x}$.
Then
\begin{align*}
b_s a_s = a_s b_s
\end{align*}
since $b_s$ commutes with $a$ and $a_s = p_s\prs{a}$.
Similarly, $b_S a_n = a_n b_s$ and $b_n a_n = a_n b_n$.
Then
$a_s - b_s = b_ - a_n$ is semi-simple and nilpotent (a sum of commuting semi-simple/nilpotent matrices is semi-simple/nilpotent), hence zero.
\end{proof}

\begin{lemma}
Let $a \in \End\prs{V}$, $b \in \End\prs{W}$, and $\phi \in \hom\prs{V,W}$ such that $b \circ \phi = \phi \circ a$. Then $b_s \circ \phi = \phi \circ a_s$ and $b_n \circ \phi = \phi \circ a_n$.
\end{lemma}

\begin{proof}
Consider the commutative diagram
\[
\begin{tikzcd}
V \arrow[rr, "\phi"] \arrow[dr, "\iota"] & & W \\
& V \oplus W \arrow[ur, "p"]
\end{tikzcd}
\]
with $i\prs{v} = \prs{v, \phi\prs{v}}$ and $p\prs{v,w} = w$.
By considering $a \oplus b \in \End\prs{V \oplus W}$ we see that we may assume $\phi$ is either injective or surjective.

In other word, for $T \in \End\prs{V}$ and $Z \leq V$ which is $T$-invariant, we want to show that $prs{\rest{T}{Z}}_s = \rest{\prs{T_s}}{Z}$ (for when $\phi$ is injective) and if $\tilde{T} \in \End\prs{V/Z}$ is the induced map on the quotient, we have to show $\tilde{T}_s = \widetilde{\prs{T_s}}$.

Now, $\rest{\prs{T_s}}{Z}$ is semi-simple as a restriction of such, and $\rest{T_n}{Z}$ is nilpotent, and these commute. By uniqueness, we get the case for the restriction. Similarly one gets the case for quotients.
\end{proof}

\subsubsection{Jordan-Chevalley Decomposition and Groups}

Let $G \leq \mrm{GL}\prs{V}$ be a closed subgroup. Choose $v \in V$ and $\phi \in V^*$, and write $f_{v,\phi} \in \mbb{F}\brs{G}$ for $f_{v,\phi}\prs{g} = \phi\prs{g \cdot v}$. This is an analogue for a representing matrix of a linear transformation given two bases.

For every $\phi \in V^*$ let
\begin{align*}
r^\phi \colon V &\to \mbb{F}\brs{G} \\
v &\mapsto f_{v,\phi} \text{.}
\end{align*}
This commutes with the group action.

Consider $G \acts G$ with the right-action $g * h = h g^{-1}$. This gives a linear action $G \acts \mbb{F}\brs{G}$ given by
\[\rho\prs{g_0}\prs{f}\prs{g} = f\prs{g g_0}\]
for $g_0 \in G$ and $f \in \mbb{F}\brs{G}$. We call this right-translation.
For every $g \in G$ we get
\[\rho\prs{g} \circ r^{\phi} = r^\phi \circ g \text{.}\]
We would like to say the following things.

\begin{enumerate}
\item That
\[\rho\prs{g}_s \circ r^\phi = r^\phi \circ g_s \text{,}\]
but $\rho\prs{g}$ is an operator on an infinite-dimensional space.
\item That $\rho\prs{g}_s = \rho\prs{\hat{g}}$ for some $\hat{g} \in G$.
\item That $g_s = \hat{g} \in G$.
\end{enumerate}

%LECTURE 25.11

\begin{notation}
For a group $G$ denote by $\lambda, \rho$ its action on itself from the left and right respectively. Denote the actions of an element $g$ by $\lambda_g, \rho_g$.
\end{notation}

\begin{remark}
For an algebraic group $G$ we get maps
\begin{align*}
\lambda\prs{g} \colon \mbb{F}\brs{G} &\to \mbb{F}\brs{G} \\
\lambda\prs{g}\prs{f}\prs{h} &= f\prs{\lambda_{g^{-1}}\prs{h}} = f\prs{g^{-1} h}
\end{align*}
and
\begin{align*}
\rho\prs{g} \colon \mbb{F}\brs{G} &\to \mbb{F}\brs{G} \\
\rho\prs{g}\prs{f}\prs{h} &= f\prs{\rho_{g^{-1}}\prs{h}} = f\prs{hg} \text{.}
\end{align*}
\end{remark}

We would like to write each $\rho\prs{g}$ as $\rho\prs{g}_s + \rho\prs{g}_n$ for semi-simple and nilpotent parts, but we don't have such a decomposition yet since $\mbb{F}\brs{G}$ is infinite-dimension.

\begin{definition}
Let $V$ be a vector space and let $a \in \End\prs{V}$ such that for every $v \in V$ there's $W_v \leq V$ finite-dimensional and $a$-invariant such that $v \in W_v$.
Define
\begin{align*}
a_s \colon V &\to V \\
v &\mapsto \prs{\rest{a}{W}}_s \prs{v}
\end{align*}
for $W_v$ as above.
\end{definition}

\begin{exercise}
The above definition is well-defined. This follows from the fact that restriction and taking the semi-simple part commute.
\end{exercise}

\begin{theorem} \label{theorem:jordan_chevalley_groups}
Let $\mbb{F}$ be algebraically-closed and let $G \leq \mrm{GL}\prs{V}$ be an embedded $\mbb{F}$-algebraic group. For every $g \in G$ it holds that $g_s, g_u \in G$.
\end{theorem}

\begin{exercise}
It holds that $\prs{a \otimes b}_s = a_s \otimes b_s$.
\end{exercise}

\begin{proof}[\ref{theorem:jordan_chevalley_groups}]
Consider the map
\begin{align*}
m \colon \mbb{F}\brs{G} \otimes \mbb{F}\brs{G} &\to \mbb{F}\brs{G} \\
f_1 \otimes f_2 &\mapsto f_1 \cdot f_2 \text{.}
\end{align*}
Since $\rho\prs{g}$ is a homomorphism for every $g \in G$, it holds that
\[m\prs{\rho\prs{g} \otimes \rho\prs{g}} = \rho\prs{g} \circ m \text{.}\]
Hence
\[m\prs{\rho\prs{g}_s \otimes \rho\prs{g}_s} = \rho\prs{g}_s \circ m\]
since this is true locally (on finite-dimensional subspaces).
Hence $\rho\prs{g}_s \colon \mbb{F}\brs{G} \to \mbb{F}\brs{G}$ is a homomorphism of $\mbb{F}$-algebras.

We get a regular map
$\psi \colon G \to G$ such that $\psi^* = \rho\prs{g}_s$.
Let $\hat{g} \coloneqq \psi\prs{e} \in G$.
For $h \in G$ we have
\begin{align*}
\rho\prs{g} \circ \rho\prs{h} = \lambda\prs{h} \circ \rho\prs{g}
\end{align*}
so
\begin{align*}
\rho\prs{g}_s \circ \rho\prs{h} = \lambda\prs{h} \circ \rho\prs{g}_s \text{.}
\end{align*}
Hence $\lambda\prs{h} = \lambda_{h^{-1}}^*$.
Then
\[\psi^* \circ \lambda_h^* = \lambda_h^* \circ \psi^*\]
so
\[\psi \circ \lambda_h = \lambda_h \circ \psi\]
and then
\[\psi\prs{h} = \psi\prs{\lambda_h\prs{e}} = \lambda_h\prs{\psi\prs{e}} = h \hat{g} \text{.}\]
Hence $\psi = \rho_{\hat{g}^{-1}}$ so $\rho\prs{g}_s = \rho\prs{\hat{g}}$.

For $\phi \in V^*$ remind that
\begin{align*}
r^\phi \colon V &\to \mbb{F}\brs{G} \\
\phi &\mapsto f_{v,\phi}
\end{align*}
and $f_{v,\phi}\prs{g} = \phi\prs{g\cdot v}$.
It holds that $\rho\prs{g} \circ r^\phi = r^\phi \circ g$ for all $g \in G$.
Then
\begin{align*}
r^\phi \circ g_s = \rho\prs{g}_s \circ r^\phi = \rho\prs{\hat{g}} \circ r^\phi = r^\phi \circ \hat{g}
\end{align*}
for every $\phi \in V^*$. Hence $g_s\prs{v} = \hat{g}\prs{v}$ for all $v \in V$, so $g_s = \hat{g} \in G$.
Then also $g_u = g g_s^{-1} \in G$.
\end{proof}

\begin{theorem}
Let $\mbb{F}$ be an algebraically-closed field.

\begin{enumerate}
\item Let $G$ be an $\mbb{F}$-algebraic group.
For every $g \in G$, the elements $g_s, g_u \in G$ are independent of the embedding of $G$.

\item For a homomorphism $\phi \colon G_1 \to G_2$ of $\mbb{F}$-algebraic groups it holds that
\begin{align*}
\phi\prs{g_s} &= \phi\prs{g}_s \\
\phi\prs{g_u} &= \phi\prs{g}_u
\end{align*}
for all $g \in G$.
\end{enumerate}
\end{theorem}

\begin{proof}
\begin{enumerate}
\item In the previous proof, we saw that for any embedding, $g_s$ is determined by $\rho\prs{g}_s$.

\item For $g,h \in G$ it holds that
\[\phi\prs{hg} = \phi\prs{h} \phi\prs{g}\]
so
\[\rho_{G_1}\prs{g} \circ \phi^* = \phi^* \circ \rho_{G_2} \prs{\phi\prs{g}} \text{.}\]
Indeed, for $f \in \mbb{F}\brs{G_2}$ we have
\begin{align*}
\rho_{G_1}\prs{g}\phi^*\prs{f}\prs{g'} &= f\prs{\phi\prs{g'g}}
\end{align*}
and
\begin{align*}
\phi^*\prs{\rho_{G_2}\prs{\phi\prs{g}}\prs{f}\prs{g'}} &= f\prs{\phi\prs{g'}\phi\prs{g}}
\end{align*}
so these are equal.
We get
\begin{align*}
\rho_{G_1}\prs{g_s} \circ \phi^* = \phi^* \circ \rho_{G_2}\prs{\phi\prs{g}_s} \text{.}
\end{align*}
Then
\begin{align*}
\ev_{g_s} \circ \phi^* &= \ev_{e_{G_2}} \circ \rho_{G_2}\prs{g_s} \circ \phi^*
\\&= \ev_{e_{G_2}} \circ \phi^* \circ \rho_{G_2}\prs{\phi\prs{g}_s}
\\&= \ev_{e_{G_1}} \circ \rho\prs{\phi\prs{g}_s}
\\&= \ev_{\phi\prs{g}_s}
\end{align*}
so $\rho\prs{g_s} = \rho\prs{g}_s$.
\end{enumerate}
\end{proof}

\begin{definition}
Let $G$ be an $\mbb{F}$-algebraic group. Call an element $g \in G$ \emph{semi-simple} if $g = g_s$ and \emph{unipotent} if $g = g_u$.
\end{definition}

\subsection{Unipotent and Reductive Groups}

\begin{definition}[Unipotent Group]
An $\mbb{F}$-algebraic group $G$ is called \emph{unipotent} if every $g \in G$ is unipotent.
\end{definition}

\begin{notation}
Denote $U_n\prs{\mbb{F}} \leq M_n\prs{\mbb{F}}$ for the group of upper-triangular matrices with $1$ on the diagonal.
\end{notation}

\subsubsection{Some Representation Theory}

Every closed subgroup of $U_n\prs{\mbb{F}}$ is unipotent. We want to show these are all the unipotent subgroups.

\begin{definition}[Representation]
Let $G$ be an $\mbb{F}$-algebraic group. A \emph{representation} of $G$ is an $\mbb{F}$-vector space $V$ together with a regular homomorphism $\phi \colon G \to \mrm{GL}\prs{V}$.
\end{definition}

\begin{definition}[Irreducible Representation]
A representation $\prs{\phi, V}$ of $G$ is \emph{irreducible} if there's no non-trivial $W \leq V$ which is $\phi\prs{G}$-invariant.
\end{definition}

\begin{definition}[Subrepresentation]
If $W \leq V$ is $\phi\prs{G}$-invariant, the restrictions $\rest{\phi}{W}\prs{g} \coloneqq \rest{\phi\prs{g}}{W} \in \mrm{GL}\prs{W}$ form a representation $\prs{\rest{\phi}{W}, W}$ which we call a \emph{subrepresentation} of $\prs{\phi, V}$. 
\end{definition}

\begin{definition}[Quotient Representation]
If $W \leq V$ is $\phi\prs{G}$-invariant, the quotient maps $\overline{\phi\prs{g}} \in \mrm{GL}\prs{V/W}$ form a representation $\prs{\bar{\phi}, V/W}$ called the \emph{quotient representation}.
\end{definition}

\begin{lemma}[Schur]
Let $\mbb{F}$ be algebraically-closed and let $V$ be a finite-dimensional vector space over $\mbb{F}$.
Let $S \subseteq \End\prs{V}$ be such that there's no non-trivial $W \leq V$ which is $S$-invariant (usually one takes $S = \phi\prs{G}$).
If $T \in \End\prs{V}$ is such that $T \circ a = a \circ T$ for all $a \in S$, then $T = \lambda \id_V$ for some $\lambda \in \mbb{F}$.
\end{lemma}

\begin{proof}
Since $\mbb{F}$ is algebraically-closed, $T$ has an eigenvalue $\lambda \in \mbb{F}$. For every $a \in S$ and $v \in \ker\prs{T - \lambda \id_V}$ it holds that
\begin{align*}
\T\prs{a\prs{v}} &= a\prs{T\prs{v}}
\\&= a\prs{\lambda v}
\\&= \lambda a\prs{v} \text{.}
\end{align*}
Since $a\prs{v} \in \ker\prs{T - \lambda \id_V}$ we get that $\ker\prs{T - \lambda I}$ is $S$-invariant. From the assumption, it follows that $\ker\prs{T - \lambda \id_V} = V$ so $T = \lambda \id_V$.
\end{proof}

\begin{theorem}[Burnside, Wedderburn; Density Theorem]\label{theorem:density}
Let $\mbb{F}$ be an algebraically-closed field, let $V$ be a finite-dimensional $\mbb{F}$-vector space and let $S \subseteq \End\prs{V}$ such that there's no non-trivial $S$-invariant $W \leq V$.
Then
\[\spn_{\mbb{F}}\prs{S} = \End\prs{V} \text{.}\]
\end{theorem}

\begin{notation}
For $A \in M_{r,n}\prs{\mbb{F}}$ let
\begin{align*}
\tau_A \colon V^{\oplus n} &\to V^{\oplus n} \\
\prs{v_1, \ldots, v_r} &\mapsto \prs{v_1, \ldots, v_r} \cdot A \text{.}
\end{align*}
(If we identify $V^{\oplus n} \cong \mbb{F}^n \otimes_{\mbb{F}} V$ we get $\tau_A = L_A \otimes \id_V$.)
For every $a \in \End\prs{V}$ one gets
\[\tau_A \circ \phi_r \prs{a} = \phi_n\prs{a} \circ \tau_A \text{.}\]

Denote also
\begin{align*}
\phi_n \colon \End\prs{V} &\to \End\prs{V^{\oplus n}} \\
a &\mapsto a^{\oplus n} \text{.}
\end{align*}
\end{notation}

\begin{lemma}\label{lemma:density}
Let $\mbb{F}$ be an algebraically-closed field, let $V$ be a finite-dimensional $\mbb{F}$-vector space and let $S \subseteq \End\prs{V}$ such that there's no non-trivial $S$-invariant $W \leq V$.
Let $W \lneq V^{\oplus n}$ which is $\phi_n\prs{S}$-invariant. Then there's $A \in M_{n,r}\prs{\mbb{F}}$ for $r < n$ and such that $W = \im\prs{\tau_A}$.
\end{lemma}

\begin{proof}
\begin{itemize}
\item Assume first that $W \neq \set{0}$ is irreducible (i.e. has no non-trivial $\phi_n\prs{S}$-invariant subspace).
Let $p_i \colon V^{\oplus n} \to V$ be the projection on the $i$\textsuperscript{th} component.
Since $W$ is non-zero, there's $i_0 \in [n]$ such that $\rest{p_{i_0}}{W} \neq 0$. Then $\psi \coloneqq \rest{p_{i_0}}{W}$ is an isomorphism since its domain and range are irreducible (one gets $\ker\prs{\psi} = \set{0}$ and $\im\prs{\psi} = V$).

Let $t_i \coloneqq p_i \circ \psi^{-1} \in \End\prs{V}$. This commutes with every $a \in S$. By Schur's lemma, $t_i\prs{v} = \lambda_i v$ for some $\lambda_i \in \mbb{F}$. Now
\begin{align*}
W &= \set{\prs{\lambda_1 v, \ldots, \lambda_n v}}{v \in V}
\\&= \im \prs{\tau_{\pmat{\lambda_1, \ldots, \lambda_n}}} \text{.}
\end{align*}

\item 
Now let $\set{0} \neq W \subseteq V^{\oplus n}$ be general. We use induction on $n$.
There's $\set{0} \neq W_0 \subseteq W$ irreducible and $\phi_n\prs{S}$-invariant.
By the previous case, we can write $W_0 = \im \tau_{\prs{q_1, \ldots, q_n}}$. There's $g \in \mrm{GL}_n\prs{\mbb{F}}$ such that
\[\prs{q_1, \ldots, q_n} \cdot g = \prs{1, 0, \ldots, 0} \text{.}\]
Then
\begin{align*}
\tau_g\prs{W} &= \im \tau_{\prs{1, 0, \ldots, 0}} = \set{\prs{v,0,\ldots,0}}{v \in V}
\\&= \tau_g\prs{W_0} \oplus \prs{\tau_g\prs{W} \ker p_1}
\end{align*}
where
$p_1$ is the projection the the first component.
Denote $W' \coloneqq \prs{\tau_g\prs{W} \ker p_1}$. By induction we can write $W' = \im\tau_{A'}$ for $A \in M_{r', n-1}\prs{\mbb{F}}$ with some $r' < n-1$. Then $\tau_g\prs{W} = \im \tau_{A''}$ for $A'' \coloneqq \pmat{1 & 0 \\ 0 & A'}$.
Then $W = \im \prs{\tau_{A'' g^{-1}}} \in M_{r'+1, n} \prs{\mbb{F}}$.
\end{itemize}
\end{proof}

%LECTURE  8.12 (half of the above proof and)

\begin{proof}[\ref{theorem:density}]
We assume that $S$ is multiplicatively closed, which might be necessary.

Let $\prs{e_1, \ldots, e_n}$ be a basis for $V$ and let $\underline{e} = \prs{e_1, \ldots, e_n} \in V^{\oplus n}$.
Let $W \coloneqq \spn\set{\phi_n\prs{S}\prs{\underline{e}}} \leq V^{\oplus n}$ which is $\phi_n\prs{S}$-invariant.
If $W \lneq V^{\oplus n}$, it follows from the lemma that there's $A \in M_{r,n}\prs{\mbb{F}}$ with $r < n$ and $W = \im \prs{\tau_A}$. In particular, in that case $\underline{e} \in \im\prs{\tau_A}$, so $\dim\prs{\spn\set{e_1, \ldots, e_n}} \leq r$,
a contradiction.
Hence $W = V^{\oplus n}$.

Take $T \in \End\prs{V}$. Then $\prs{T\prs{e_1}, \ldots, T\prs{e_n}} \in W$, so there are $\prs{s_i}_{i \in [k]} \subseteq S$ and $\prs{\alpha_i}_{i \in [k]} \subseteq \mbb{F}$ such that $U \coloneqq \sum_{i \in [k]} \alpha_i s_i$ gives $T\prs{e_i} = U\prs{e_i}$ for all $i \in [k]$, so $T = U \in \spn\prs{S}$.
\end{proof}

\begin{theorem}
Let $G$ be a unipotent group over an algebraically-closed field $\mbb{F}$. Let $\phi \colon G \to \mrm{GL}\prs{V}$ be an irreducible rational representation. Then $\varphi$ is the trivial representation.
\end{theorem}

\begin{proof}
From properties we've shown, elements of $\tilde{G} \coloneqq \phi\prs{G}$ are unipotent transformations.
For $g \in \tilde{G}$ we have
$\tr\prs{g} = n$ and for every $g,h \in \tilde{G}$ we have
\[\tr\prs{\prs{\id-g}h} = \tr\prs{h} - \tr\prs{gh} = n-n = 0\text{.}\]
By \ref{theorem:density} we get that
\[\tr\prs{\prs{\id-g}A} = 0\]
for all $A \in \End\prs{V}$.

Now,
\[\trs{Y,X} \coloneqq \tr\prs{Y^t X}\]
pulls back to a non-degenerate form on $\End\prs{V}$, %TODO make sure this is fine
so we get that $\id - g = 0$ for all $g \in \tilde{G}$. Then $\tilde{G} = \set{I}$ and by irreducibility of $V$ we get $\dim\prs{V} = 1$.
\end{proof}

%LECTURE 9.12

\begin{corollary}
Every unipotent group $G$ over an algebraically-closed field $\mbb{F}$ is isomorphic to a closed subgroup of $\mrm{U}_n \coloneqq \set{\pmat{1 & & * \\ & \ddots & \\ 0 & 1}}$.

In other words, for any rational representation $\phi \colon G \to \mrm{GL}\prs{V}$ there's a basis $B$ of $V$ such that $\set{\brs{T}_B}{T \in \phi\prs{G}} \subseteq U_n$.
\end{corollary}

\begin{proof}
Let $\phi \colon G \to \mrm{GL}\prs{V}$ be a rational representation.
There's a minimal irreducible subrepresentation $\set{0} \neq W \leq V$ and by the theorem $W$ is $1$-dimensional. Write $W = \spn\set{e_1}$ for some $e_1$. Then $\phi\prs{G} e_1 = e_1$, and we look at the $\phi\prs{G}$-action on $V/W$. By induction, there's a basis $f_2, \ldots, f_n$ of the quotient for which the action of $\phi\prs{G}$ is represented by matrices in $U_{n-1}$. Lift $f_2, \ldots, f_n$ to elements $e_2, \ldots, e_n \in V$. We get a basis $\prs{e_1, \ldots, e_n}$ of $V$ for which elements of $\phi\prs{G}$ are of the given form.
\end{proof}

\begin{proposition}
Let $G$ be a unipotent group acting algebraically on an affine $\mbb{F}$-variety $X$. %TODO shouldn't we require $\mbb{F}$ to be algebraically-closed?
Then every $G$-orbit is closed.
\end{proposition}

\begin{proof}
Let $x_0 \in X$ and
\begin{align*}
\phi \colon G &\to X \\
g &\mapsto g \cdot x_0 \text{.}
\end{align*}
Let $Y \coloneqq \overline{\phi\prs{G}} \leq X$, we want to show $Y = \phi\prs{G}$.

There's $U \leq Y$ open and non-empty such that $U \subseteq \phi\prs{G}$.
It holds that $\phi\prs{G} = \bigcup_{g \in G} g \cdot U$ so $\phi\prs{G} \leq Y$ is open.
Indeed, taking $u_0 \in U$ there's $g_0 \in G$ such that $u_0 = g_0 \cdot x_0$, so for all $g \in G$ we have
\[g \cdot x_0 = g g_0^{-1} u_0 \in g g_0^{-1} U \text{.}\]

Now, $Z \coloneqq Y \setminus \phi\prs{G}$ is closed, and assume towards a contradiction that it's non-empty.
We get $\prs{0} \neq I\prs{Z} \ideal \mbb{F}\brs{Y}$.
$Y$ is $G$-invariant as the closure of an orbit. (Check this. It's similar to showing that $H \leq G$ implies $\bar{H} \leq G$.)
We have a linear action $G \acts \mbb{F}\brs{Y}$ via $g \cdot f\prs{x} = f\prs{g^{-1}x}$ and $Z$ is $G$-invariant, so $I\prs{Z}$ is $G$-invariant. There's a finite-dimensional $G$-invariant $V \leq I\prs{Z}$. 
From the theorem on unipotent groups, there's $f \in I\prs{Z}$ non-zero such that $G \cdot f = f$. %TODO why?
Then $f \in \mbb{F}\brs{Y}$ is constant on the $G$-orbits.
$\phi\prs{G}$ is dense in $Y$ so $\rest{f}{\phi\prs{G}} \equiv 0$ (since $\rest{f}{Z} \equiv 0$), so $f = 0$, a contradiction.
\end{proof}

\begin{remark}
The above isn't true without the unipotent requirement.
For example, consider $G \coloneqq \mbb{F}^\times$ acting on $\mbb{F}$ via multiplication.
The orbits are $\set{0}$ and $\mbb{F}^\times$, where the latter isn't closed unless $\mbb{F}$ is finite.
\end{remark}

\begin{example}
Consider the action of $G \coloneqq \mrm{GL}_2\prs{\mbb{F}}$ on $M_2\prs{\mbb{F}}$ by conjugation. We have
\begin{align*}
\set{\pmat{0 & a \\ 0 & 0}}{a \in \mbb{F}^{\times}} \subseteq G \cdot \pmat{0 & 1 \\ 0 & 0}
\end{align*}
but $0 \notin G \cdot \pmat{0 & 1 \\ 0 & 0}$ and this is clearly in the closure.
\end{example}

\begin{example}
Examine
\[\prs{\mbb{F}, +} \cong U_2 \coloneqq \set{\pmat{1 & a \\ 0 & 1}}{a \in \mbb{F}}\]
with the usual action $U_2 \acts \mbb{F}^2$.
For $a \in \mbb{F} \setminus \mbb{F}$ let
\[Y_a \coloneqq \set{\pmat{x \\ a}}{x \in \mbb{F}} \text{.}\]
These are closed orbits, but the the rest of the orbits are points $\set{\pmat{b \\ 0}}$ with $b \in \mbb{F}$. These orbits are all closed.
\end{example}

\section{Tangent Spaces}

\subsection{Definitions}

\begin{definition}[Tangent Space]
Let $X$ be an embedded affine $\mbb{F}$-variety and let $p \in X$.
Define \emph{the tangent space of $X$ at $p$} to be
\begin{align*}
T_p X \coloneqq \set{v \in \mbb{F}^n}{\frac{\del f_i}{\del v} \prs{p} = 0}
\end{align*}
where
\[\frac{\del f}{\del v}\prs{p} = \sum_{i \in [n]} \frac{\diff f}{\diff x_i} c\dot v_i \text{.}\]
\end{definition}

\begin{notation}
Denote $\nabla f = \prs{\frac{\del f}{\del x_1}, \frac{\del f}{\del x_2}, \ldots, \frac{\del f}{\del x_n}}$.
\end{notation}

\begin{example}
Let $X = V\prs{y - x^2, z - x^3} \subseteq \mbb{F}^3$ where $f_1 \coloneqq y - x_2$ and $f_2 \coloneqq z - x^3$.
Then $X = \set{\prs{x,x^2,x^3}}{x \in \mbb{F}}$ and
\begin{align*}
\nabla f_1 &= \prs{-2x, 1, 0} \\
\nabla f_2 &= \prs{-3x^2, 0, 1} \text{.}
\end{align*}
So,
\[T_p X = \set{\prs{v_1, v_2, v_3}}{
\begin{split}
-2x v_1 + v_2 &= 0 \\ -3x^2 v_1 + v_3 &= 0
\end{split}
} \text{.}\]
\end{example}

\begin{example}
Let $X \coloneqq V\prs{xy} \subseteq \mbb{F}^2$ and $f = xy$ so that $\nabla f = \prs{y,x}$.
For $p = \prs{p_1, p_2} \in X$ we have
\begin{align*}
T_p X &=
\begin{cases}
\spn\set{\prs{0,1}} & p_1 = 0 \wedge p_2 \neq 0 \\
\spn\set{\prs{1,0}} & p_1 \neq 0 \wedge p_2 = 0 \\
\mbb{F}^2 & p = \prs{0,0}
\end{cases} \text{.}
\end{align*}
We later call a point such as $p$ where the tangent space is of \emph{different dimension}, a \emph{singular point}.
\end{example}

\begin{example}
Given $X \coloneqq V\prs{y^2 - x^3}$ we have $T_{\prs{0,0}}X = \mbb{F}^2$.
\end{example}

Examine the Jacobian
\[J\prs{f_1, \ldots, f_k} \coloneqq \pmat{\nabla f_1 \\ \vdots \\ \nabla f_k} \in M_{k,n}\prs{A_n}\]
where $A_n \coloneqq \mbb{F}\brs{x_1, \ldots, x_n}$.
At every point $p$ we can evaluate
\[J\prs{f_1, \ldots, f_k}\prs{p} \in M_n\prs{\mbb{F}}\]
and
\[T_p X = \mrm{null}\prs{J\prs{f_1, \ldots, f_k}} \leq \mbb{F}^n \text{.}\]

Assume now that $X$ is irreducible, so that $\mbb{F}\brs{X}$ is an integral domain. Let $K$ be the fraction field of $\mbb{F}\brs{X}$, and consider $\Omega \coloneqq \overline{J\prs{f_1, \ldots, f_k}} \in M_n\prs{K}$ where $\mbb{F}\brs{X} \cong \mbb{F}\brs{x_1, \ldots, x_n}/\prs{f_1, \ldots, f_k}$ and $f_i \in \mbb{F}\brs{x_1, \ldots, x_n}$. Let $d \coloneqq \dim\prs{\mrm{null}\prs{\Omega}}$. Note that $\Omega$ is supposedly dependent on the choice of generators, but it is in fact not.

\begin{proposition}
Let $X$ be an irreducible $\mbb{F}$-variety and let $\Omega$ and $d$ be as above. For every $p \in X$ it holds that $\dim T_p X \geq d$ and for every non-empty $U \subseteq X$ with $p \in U$ it holds that $\dim T_p X = d$.
\end{proposition}

\begin{proof}
We have $\mrm{rank}\prs{\Omega} = n - d$. Examine the matrix's minors of size $n-d$ which we denote $g_1, \ldots, g_t \in \mbb{F}\brs{X}$.
Now $U \ceq X \setminus V\prs{g_1, \ldots, g_t} \neq \ns$ since the $g_i$ aren't all zero.
For $p \in U$ there's $i \in [t]$ for which $g_i\prs{p} \neq 0$, hence
\[\rank\prs{\Omega p} \geq n-d\]
so
\[\rank\prs{\Omega p} \leq \rank\prs{\Omega} = n - d\]
and for $p \in X$ we have $\dim T_p X \geq d$.
\end{proof}

\subsection{Definitions Independent of the Embedding}

\begin{definition}[Module Over an Algebra]
Let $A$ be an $\mbb{F}$-algebra. An $A$-module $M$ is a vector space over $\mbb{F}$ together with an $\mbb{F}$-algebra homomorphism
\[\phi \colon A \to \End_{\mbb{F}}\prs{M} \text{.}\]
\end{definition}

\begin{notation}
We usually write $a \cdot m$ instead of $\phi\prs{a}\prs{m}$, and don't write $\phi$.
\end{notation}

\begin{example}
An ideal $I \ideal A$ is an $A$-module.
\end{example}

\begin{example}
$A/I$ is an $A$-module.
\end{example}

For $X$ an affine $\mbb{F}$-variety and $p \in X$, examine $\mfrak{m}_p \coloneqq \mbb{F}\brs{X} / \ker\prs{\ev_p} \cong \mbb{F}$. This is an $\mbb{F}\brs{X}$-module where $f \in \mbb{F}\brs{X}$ acts on $t \in \mbb{F}$ via $f \cdot t = f\prs{p} \cdot t$ under the above isomorphism.

\begin{definition}[Derivation Space]
Let $M$ be an $A$-module.
We define the \emph{derivation space}
\[\mrm{Der}_{\mbb{F}}\prs{A,M} \coloneqq \set{D \in \hom_{\mbb{F}}\prs{A, M}}{\forall a,b \in A \colon D\prs{ab} = a \cdot D\prs{b} + b \cdot D\prs{a}} \text{.}\]
\end{definition}

\begin{example}
Let $v \in \mbb{F}^n$. We have $\frac{\del}{\del v} \in \mrm{Der}_{\mbb{F}}\prs{A_n, A_n}$ since (regular) derivation satisfies the Leibniz rule.
\end{example}

\begin{definition}
Let $A$ be a finitely-generated $\mbb{F}$-algebra with generators $x_1, \ldots, x_n \in A$. Write $A \cong A_n / I$ and $I = \prs{f_1, \ldots, f_k}$.
Let $\Omega \coloneqq J\prs{f_1, \ldots, f_k} \in M_{k,n}\prs{A}$.
Taking an $A$-module $M$, we get that
$\Omega$ acts on the left of $M^n$ via matrix multiplication.
We define
\[\mcal{I}_{A,M} = \set{v \in M^n}{\Omega v = 0} \text{.}\]
\end{definition}

%LECTURE 15.12.21

\begin{notation}
For $M \in \mbb{F}\brs{X}\mathbf{-Mod}$, denote
\begin{align*}
\mcal{I}_{\mbb{F}\brs{X}, M} \coloneqq \set{v \in M^n}{\Omega v = 0} \leq M^n \text{.}
\end{align*}
Denote also $\mbb{F}_p \coloneqq \mbb{F}\brs{X} / \prs{p}$.
\end{notation}

\begin{remark}
It holds that
\[T_p X = \mcal{I}_{\mbb{F}\brs{X}, \mbb{F}_p}\]
and we've seen that
\[\dim_k \mcal{I}_{\mbb{F}\brs{X}, K} \leq \dim_{\mbb{F}} T_p X\]
where there's equality on each $p$ in an open subset of $X$.
\end{remark}

\begin{definition}[Derivation Space]
For $M \in A\mathbf{-Mod}$ let
\begin{align*}
\mrm{Der}_{\mbb{F}}\prs{A,M} \coloneqq \set{D \in \hom_{\mbb{F}}\prs{A,M}}{
\forall a,b \in A \colon D\prs{ab} = a \cdot D\prs{b} + b \cdot D\prs{A}
} \text{.}
\end{align*}
\end{definition}

\begin{definition}
For $v = \pmat{v_1 \\ \vdots \\ v_n} \in \mcal{I}_{A,M}$ consider the quotient map $\pi \colon \mbb{F}\brs{x_1, \ldots, x_n} \to \mbb{F}\brs{x_1, \ldots, x_n}/\prs{f_1, \ldots, f_k}$, and define
\begin{align*}
D_v \colon A &\to M \\
\pi\prs{f} &\mapsto \sum_{i \in \brs{n}}\pi\prs{\frac{\diff f}{\diff x_i}} \cdot v_i \text{.}
\end{align*}
\end{definition}

\begin{proposition}
\begin{enumerate}
\item $D_v \in \mrm{Der}_{\mbb{F}}\prs{A,M}$, and in particular this is well-defined.

\item The map
\begin{align*}
\eta \colon \mcal{I}_{A,M} &\to \mrm{Der}_{\mbb{F}}\prs{A,M} \\
v &\mapsto D_v
\end{align*}
is an $A$-module isomorphism.
\end{enumerate}
\end{proposition}

\begin{proof}
\begin{enumerate}
\item Let $v \in \mcal{I}_{A,M}$. We want to show that $D_v\prs{\ker\prs{\pi}} = \set{0}$ so that $D_v$ is well-defined.
For $j \in [n]$ it holds that
\[\sum_{i \in [n]} \pi\prs{\frac{\diff f_j}{\diff x_i}} \cdot v_i = 0 \text{.}\]
For a general $f \in \ker\prs{\pi}$ we can write $f = \sum_{i \in [k]} h_i f_i$, so
\begin{align*}
\pi\prs{\frac{\diff f}{\diff x_i}} = \sum_{j \in [k]} \pi\prs{h_j} \pi\prs{\frac{\diff f_j}{\diff x_i}} + \pi\prs{\frac{\diff h_j}{\diff x_i}} \cancelto{0}{\pi\prs{f_j}} = 0
\end{align*}
where the last equality follows from the previous computation. Hence $D_v$ is well-defined.
It easily seen that the Leibniz law holds, so $D_v$ is a derivation.

\item If $D_v = 0$ we have
\begin{align*}
0 &= D_v\prs{\pi\prs{x_j}} = v_j
\end{align*} 
for all $j \in [k]$, so $v = 0$. Hence $\eta$ is injective.

Assume $D \in \mrm{Der}_{\mbb{F}}\prs{A,M}$ and let $\tilde{D} \coloneqq D \circ \pi \in \mrm{Der}_{\mbb{F}}\prs{\mbb{F}\brs{x_1, \ldots, x_n}, M}$. Let
\[v \coloneqq \pmat{\tilde{D}\prs{x_1} \\ \vdots \\ \tilde{D}\prs{x_n}} \in M^n \text{.}\]
For $j \in [n]$ we have
\begin{align*}
\tilde{D}\prs{x_j} &= \sum_{i \in [n]} \frac{\diff x_j}{\diff x_i} \cdot v_i = \tilde{D}_v\prs{x_j} \text{.}
\end{align*}
As before, derivations are determined by generators and we get that
\[\tilde{D} = \tilde{D}_v \text{.}\]
If we show that $v \in \mcal{I}_{A,M}$ we get that $\eta$ is surjective.
This is true since for every $j \in [k]$ it holds that
\[\tilde{D}_v\prs{f_j} = \tilde{D}\prs{f_j} = 0 \text{.}\]
\end{enumerate}
\end{proof}

\begin{corollary}
For every $p \in X$ it holds that
\begin{align*}
T_o X \cong \mrm{der}_{\mbb{F}}\prs{\mbb{F}\brs{X}, \mbb{F}_p} \text{.}
\end{align*}
\end{corollary}

\begin{proposition}
For $p \in X$ denote $\mfrak{m}_p \coloneqq \ker\prs{\ev_p} \ideal \mbb{F}\brs{X}$, which is maximal.
There is a natural isomorphism
\[T_p X \cong \prs{\mfrak{m}/\mfrak{m}^2}^* \text{.}\]
\end{proposition}

%LECTURE 16.12.21

\begin{proof}
Let $\ell \in \hom_{\mbb{F}\mathbf{-Vect}}\prs{\mfrak{m}, \mfrak{F}}$ such that $\rest{\ell}{\mfrak{m}^2} \equiv 0$.
Define
\begin{align*}
\mu \colon \mbb{F}\brs{X} &\to \mbb{F} \\
f &\mapsto \ell\prs{f-f\prs{p}\cdot 1} \text{.}
\end{align*}
One can check that $\mu \in \mrm{Der}_{\mbb{F}}\prs{\mbb{F}\brs{X}, \mbb{F}\prs{p}}$.

For the other direction, given $\mu$ in this space, we can take $\ell = \rest{\mu}{\mfrak{m}}$.
\end{proof}

Let $A,B$ be $\mbb{F}$-algebras, $N$ a $B$-module and $\psi \colon A \to B$ an $\mbb{F}$-module homomorphism.
There is an induced $A$-module structure on $N$ by $a \cdot n \coloneqq \psi\prs{a} n$.
In particular, there is an induced map
\begin{align*}
\tilde{\psi} \colon \mrm{Der}_{\mbb{F}}\prs{B,N} &\to \mrm{Der}_{\mbb{F}}\prs{A,N} \\
D &\mapsto D \circ \psi \text{.}
\end{align*}

If $X,Y$ are affine $\mbb{F}$-varieties and $\phi \colon X \to Y$ is regular, we have a map $\phi^* \colon \mbb{F}\brs{Y} \to \mbb{F}\brs{X}$.
For $p \in X$ the module $\mbb{F}\prs{p} \coloneqq \mbb{F}\brs{X}/\ker\prs{\ev_p}$ has the structure of an $\mbb{F}\brs{Y}$-module and is isomorphic to $\mbb{F}\prs{\phi\prs{p}}$ with this structure.

\begin{definition}
Denote
\begin{align*}
\diff_p \phi \coloneqq \tilde{\phi}^* \colon T_p\prs{X} \to T_{\phi\prs{p}}\prs{Y}
\end{align*}
and call this the \emph{differential} of $\phi$ at $p$.
\end{definition}

\begin{exercise}
If one identifies $T_p\prs{X} \leq \mbb{F}^n$ and $T_{\phi\prs{p}}\prs{Y} \leq \mbb{F}^m$, show that $\phi$ is a restriction of a polynomial $\mbb{F}^n \to \mbb{F}^m$ and that $\diff_p\phi$ is identified with the matrix $M_{n,m}\prs{\mbb{F}}$ which is the differential of $\phi$ (as over $\mbb{R}$).
\end{exercise}

\subsection{Dimension}

In analogy to differentiable manifolds, we define algebraic varieties as a space which locally looks like $\mbb{F}^n$.

\begin{definition}
Let $X$ be an irreducible affine $\mbb{F}$-variety, and let $K \coloneqq \mrm{Frac}\prs{\mbb{F}\brs{X}}$.
Define $\dim X \coloneqq \mrm{trdeg}_{\mbb{F}}\prs{K}$.
\end{definition}

\begin{fact}
Let $\mbb{F}$ be a perfect field and $X$ be an irreducible affine $\mbb{F}$-variety. Then
\[
\dim X = \dim_K \mrm{Der}_{\mbb{F}}\prs{\mbb{F}\brs{X}, K} \text{.}
\]
\end{fact}

\begin{corollary}
Let $X$ be an irreducible affine $\mbb{F}$-variety over a perfect field $\mbb{F}$.
There is $U \subseteq X$ open and non-empty such that for every $p \in U$ it holds that $\dim T_p X = \dim X$.
\end{corollary}

\begin{definition}
For $X$ as above, a point $p \in X$ is called \emph{simple} or \emph{smooth} if $\dim T_p X = \dim X$, and \emph{singular} otherwise (in which case $\dim T_p X > \dim X$).
\end{definition}

\begin{definition}[Smooth Variety]
For $X$ as above, we say $X$ is \emph{smooth} if it has no singular points.
\end{definition}

\begin{proposition}
Let $G$ be an $\mbb{F}$-algebraic group for $\mbb{F}$ a perfect field.
Let $X$ be an $\mbb{F}$-affine $G$-space on which $G$ acts transitively.
Then, $X$ is smooth.
\end{proposition}

\begin{proof}
Let $x_0 \in X$ be a smooth point (which exists by the corollary).
For $x \in X$ there's $g \in G$ such that $x = g \cdot x_0$.
\begin{align*}
\lambda_g \colon X &\to X \\
x &\mapsto g \cdot x
\end{align*}
is an isomorphism, so
\[\diff_{x_0}\prs{\lambda_g} \colon T_{x_0}\prs{X} \to T_x\prs{X}\]
is an isomorphism.
Hence $x$ is smooth.
\end{proof}

\begin{proposition}
Let $X$ be an irreducible affine $\mbb{F}$-variety and let $Y \lneq X$ be closed and irreducible. Then $\dim Y < \dim X$.
\end{proposition}

\begin{proof}
We can write
\[\mbb{F}\brs{Y} \cong \mbb{F}\brs{X} / I\]
for some non-zero ideal $I \ideal \mbb{F}\brs{X}$, which is prime since $Y$ is irreducible.
Let $f_1, \ldots, f_k \in \mbb{F}\brs{Y}$ be algebraically-independent where $k \coloneqq \dim Y$.
Let $g_1, \ldots, g_k \in \mbb{F}\brs{X}$ be such that $\bar{g}_i = f_i$.
Let $g_0 \in I \setminus \set{0}$, we want to show that $g_0, \ldots, g_k$ are algebraically-independent.

Assume towards a contradiction that there's a non-zero $p \in \mbb{F}\brs{y_0, \ldots, y_k}$ such that $p\prs{g_0, \ldots, g_k} = 0$. We may assume $p$ is irreducible. Write
\[p\prs{y_0, \ldots, y_k} = y_0 p_1\prs{y_0, \ldots, y_k} + p_2\prs{y_1, \ldots, y_k}\]
for polynomials $p_1, p_2$.
We have
\begin{align*}
\bar{0} &= \overline{p\prs{g_0, \ldots, g_k}}
\\&= \overline{g_0 p_1\prs{g_0, \ldots, g_k} + p_2\prs{g_1, \ldots, g_k}}
\\&= \cancelto{0}{\overline{g_0}} \overline{p_1\prs{g_0, \ldots, g_k}} + \overline{p_2\prs{g_1, \ldots, g_k}}
\\&= \overline{p_2\prs{g_1, \ldots, g_k}}
\\&= p_2\prs{f_1, \ldots, f_k} \text{.}
\end{align*}
Hence $p_2 \equiv 0$ so $p\prs{y_0, \ldots, y_k} = y_0 p_1\prs{y_0, \ldots, y_k}$, so $p$ is reducible, a contradiction.
\end{proof}

\begin{lemma} \label{lemma:charts}
Let $X \subseteq \mbb{F}^n$ be an embedded affine irreducible variety. For $p \in X$ identify $T_p X \leq \mbb{F}^n$. For $p_0 \in X$ simple there's $f \in \mbb{F}\brs{X}$ such that $p_0 \in X_f$ and there are regular maps
\begin{align*}
\psi_i \colon X_f \to \mbb{F}^n
\end{align*}
for $i \in d \coloneqq \dim X$ such that
$\prs{\psi_1\prs{x} ,\ldots, \psi_d\prs{x}}$ is a basis for $T_x X$ for all $x \in X_f$.
\end{lemma}

\begin{proof}
Identify
\begin{align*}
\mbb{F}\brs{X} = \mbb{F}\brs{x_1, \ldots, x_n}/\prs{f_1, \ldots, f_k} \text{.}
\end{align*}
There's a non-vanishing minor of size $n-d$ in $J\prs{f_1, \ldots, f_k}\prs{p_0}$.
We have
\[T_{p_0}\prs{X} = \mrm{null}\prs{J\prs{f_1, \ldots, f_{n-d}}\prs{p_0}}\]
for $n-d \leq k$.
Then we can write
\[J\prs{f_1, \ldots, f_{n-d}} = \pmat{A & B} \in M_{n-d,n}\prs{A_n}\]
for $A \in M_{n-d}\prs{A_n}$ and $B \in M_{n-d,d}\prs{A_n}$.
Let $f = \det\prs{A}$. Then up to reordering the columns we have $\det A\prs{p_0} \neq 0$.
$X_f$ consists of simple points of $X$. For $x \in X_f$ we have
\[T_x\prs{X} = \mrm{null}\prs{J\prs{f_1, \ldots, f_{n-d}}\prs{x}}\]
so
\[\psi\prs{x} = \prs{\psi_1\prs{x}, \ldots, \psi_d\prs{x}} = \pmat{- A^{-1} B \\ I_{d \times d}}\prs{x} \in M_{n,d}\prs{\mbb{F}}\]
so
\[J\prs{f_1, \ldots, f_{n-d}}\prs{x} \cdot \psi\prs{x} = 0 \text{.}\]
\end{proof}

\begin{proposition}[Differential Criterion for Dominance]
Let $\mbb{F}$ be perfect and infinite.
Let $X,Y$ be irreducible affine $\mbb{F}$-varieties and let $\phi \colon X \to Y$.
Assume there's $p \in X$ such that $\diff_p \prs{\phi}$ is surjective onto $T_{\phi\prs{p}}\prs{Y}$ and that $\phi\prs{p} \in Y$ is simple (the second condition might be unnecessary).
Then $Y = \overline{\phi\prs{X}}$.
\end{proposition}

\begin{proof}
Using \ref{lemma:charts} with respect to $p \in X$, we get $X_f \subseteq X$ such that for all $x \in X_f$ one has
\[T_x X = \spn\prs{\psi_1\prs{x}, \ldots, \psi_d\prs{x}}\]
where
\[\psi\prs{x} = \prs{\psi_1\prs{x}, \ldots, \psi_d\prs{x}} \in M_{n,d}\prs{A_n} \text{.}\]
Now
\[p \in U_1 \coloneqq \set{x \in X_f}{\mrm{rank}\prs{\diff_x\prs{\phi}} \geq \dim\prs{Y}} \subseteq X\]
is open since the condition is equivalent to non-vanishing of minors of the matrix $\diff \phi\prs{x} \cdot \psi\prs{x}$.

Denote $Z \coloneqq \overline{\phi\prs{X}}$ and assume towards a contradiction that $Z \subsetneq Y$. Then $\dim Z \leq \dim Y$. Let
\[U_2 \coloneqq \set{x \in X}{\phi\prs{x} \text{ is simple in $Z$}} \subseteq X \text{.}\]
Then $U_2$ is the inverse image of an open set under $\phi$, and is therefore open. Hence $\phi\prs{X}$ contains an open subset of $Z$. Using irreducibility we get that $\phi\prs{X}$ contains simple points of $Z$, so $U_1 \cap U_2 \neq \ns$.
Let $x_0 \in U_1 \cap U_2$. We get
\[\dim Z = \dim T_{\phi\prs{x_0}}\prs{Z} \geq \rank \diff_{x_0}\prs{\phi} \geq \dim\prs{Y} \text{,}\]
a contradiction.
\end{proof}

%LECTURE 22.12.21

\section{Lie Algebras}

\subsection{Definitions}



\begin{definition}[Lie Algebra]
A \emph{Lie algebra} $\mfrak{g}$ over $\mbb{F}$ is a nonassociative algebra over $\mbb{F}$ whose (bilinear) product, denoted $\brs{x,y}$, and called the \emph{Lie bracket}, satisfying the following for all $x,y,z \in \mfrak{g}$.
\begin{description}
\item[Skew-symmetry:] $\brs{x,y} = -\brs{y,x}$
\item[Jacobi's Identity:] $\brs{x,\brs{y,z}} + \brs{y,\brs{z,x}} + \brs{z,\brs{x,y}} = 0$.
\end{description}
\end{definition}

\begin{definition}[The Primordial Lie Algebra]
Let $C$ be an $\mbb{F}$-algebra. We define on $C$ the \emph{commutator}
\begin{align*}
\brs{a,b} = ab - ba \text{.}
\end{align*}
This turns $C$ into a Lie algebra.
\end{definition}

\begin{example}
Let $A$ be an $\mbb{F}$-algebra. One has $\mrm{Der}_{\mbb{F}\mathbf{-Vec}}\prs{A,A} \leq \End_{\mbb{F}}\prs{A,A}$.
In particular, for $X,Y \in \mrm{Der}_{\mbb{F}}\prs{A,A}$ we have $X \circ Y \colon A \to A$. The composition $X \circ Y$ isn't a derivation, but $\brs{X,Y}$ is by direct computation.
We get a Lie algebra structure on $\mrm{Der}_{\mbb{F}}\prs{A,A}$.
\end{example}

\begin{example}
Given an $\mbb{F}$-affine variety $X$ one gets a ``large'' Lie algebra
$\mrm{Der}_{\mbb{F}}\prs{\mbb{F}\brs{X}, \mbb{F}\brs{X}}$ which is infinite-dimensional over $\mbb{F}$.

If $X = G$ is an algebraic group over $\mbb{F}$ there is a more interesting construction. Let $\lambda \colon G \to \mrm{GL}\prs{\mbb{F}\brs{G}}$ be the map with $\lambda\prs{g}\prs{f}\prs{x} = f\prs{g^{-1}x}$ for $g \in G, f \in \mbb{F}\brs{X}, x \in G$.

For $D \in \mrm{Der}_{\mbb{F}}\prs{\mbb{F}\brs{G}, \mbb{F}\prs{G}}$ and $g \in G$ we have
\[\lambda\prs{g} \circ D \circ \lambda\prs{g}^{-1} \in \mrm{Der}_{\mbb{F}}\prs{\mbb{F}\brs{G}, \mbb{F}\prs{G}} \text{.}\]
Geometrically, this moves the vector field matching $D$ by the $g$ action.
\end{example}

\begin{definition}
For $G$ as above, define
\[\Lie\prs{G} \coloneqq \set{D \in \mrm{Der}_{\mbb{F}}\prs{\mbb{F}\brs{G}, \mbb{F}\brs{G}}}{\forall g\in G \colon \lambda \prs{g} \circ D = D \circ \lambda\prs{g}} \text{.}\]
\end{definition}

\begin{remark}
It is easily checked that for $D_1, D_2 \in \Lie\prs{G}$ one has $\brs{D_1, D_2} \in \Lie\prs{G}$. We then call $\Lie\prs{G}$ the \emph{Lie algebra of $G$}.
\end{remark}

\begin{notation}
For $D \in \mrm{Der}_{\mbb{F}}\prs{\mbb{F}\brs{G}, \mbb{F}\brs{G}}$ and $g \in G$ denote
\[D_g \coloneqq \ev_g \circ D  \text{.}\]
\end{notation}

\begin{remark}
It holds that $D_g \in \Der_{\mbb{F}}\prs{\mbb{F}\brs{G}, \mbb{F}\brs{g}} = T_g\prs{G}$.
We have
\begin{align*}
\diff \prs{\lambda_g}_h \colon T_h\prs{G} &\to T_{gh}\prs{G} \\
\end{align*}
and
\begin{align*}
\lambda\prs{g} \circ D_h \circ \lambda\prs{g}^{-1} = \diff \prs{\lambda_g}_{g^{-1}h} \prs{D_{g^{-1}h}} \text{.}
\end{align*}
(Check this.)
We get that $D \in \Lie\prs{G}$ if and only if
\[\diff \prs{\lambda_g}_e \prs{D_e} = D_g \text{.}\]
\end{remark}

\begin{example}
Let $G = \prs{\mbb{F}^n, +}$. Then $A_n \coloneqq \mbb{F}\brs{G} = \mbb{F}\brs{x_1, \ldots, x_n}$.
We have
\begin{align*}
\Der_{\mbb{F}}\prs{A_n, A_n} &= \set{\sum_{i\in[n]} f_i \frac{\diff}{\diff x_i}}{\prs{f_i}_{i \in [n]} \subseteq A_n} \text{.}
\end{align*}
These correspond to global vector fields on $G$.
Then $D = \sum_{i \in [n]} f_i \frac{\diff}{\diff x_i}$ is in $\Lie\prs{G}$ if all the $f_i$ satisfy $f_i\prs{x+y} = f_i\prs{x}$ for all $y \in \mbb{F}^n$.
This is true exactly when $f_i \equiv a_i$ are constant, so $D = \sum_{i \in [n]} a_i \frac{\diff}{\diff x_i}$ corresponds to a constant vector field.
\end{example}

\begin{example}
Let $G \coloneqq \mrm{GL}_1\prs{\mbb{F}}$, so that $\mbb{F}\brs{G} \equiv \mbb{F}\brs{t,u} / \prs{tu - 1}$.
Then
\[\Der_{\mbb{F}}\prs{\mbb{F}\brs{G}, \mbb{F}\brs{G}} = \set{f \frac{\diff}{\diff t}}{f \in \mbb{F}\brs{G}} \text{.}\]
Now $f \frac{\diff}{\diff t} \in \Lie\prs{G}$ if and only if for all $a \in \mbb{F}^\times$ and $\rho \in \mbb{F}\brs{G} \cong \mbb{F}\brs{t,t^{-1}}$ we have
\[f\prs{t} \frac{\diff}{\diff t}\prs{\rho\prs{at}} = f\prs{at} \frac{\diff \rho}{\diff t}\prs{at} \text{.}\]
Then
\[a f\prs{t} \frac{\diff \rho}{\diff t}\prs{at} = f\prs{at} \frac{\diff \rho}{\diff t}\prs{at}\]
so
\[a f\prs{t} = f\prs{at}\]
so $f\prs{t} = \alpha t$ for some $\alpha \in \mbb{F}$.
We get that
\[\Lie\prs{G} = \spn\set{t \frac{\diff}{\diff t}} \text{.}\]
\end{example}

\begin{proposition}
The map
\begin{align*}
\eta \colon \Lie\prs{G} &\to T_e\prs{G} \\
D &\mapsto D_e
\end{align*}
is a vector-space isomorphism.
\end{proposition}

%LECTURE 23.12

\begin{proof}
\begin{description}
\item[Injectivity:]
Assume $D$ is such that $D_e = 0$. Then for all $g \in G$ we have
\[D_g = \diff\prs{\lambda_g}_e \prs{D_e} = 0 \text{.}\]
For $f \in \mbb{F}\brs{G}$ we then have
\[\ev_g\prs{D\prs{f}} = 0\]
so $D\prs{f} = 0$.

\item[Surjectivity:]

Let $X \in T_e\prs{G}$. Define
\[D_X \colon \mbb{F}\brs{G} \to \mbb{F}\brs{G}\]
via
\[D_X\prs{f}\prs{g} = X\prs{\lambda\prs{g^{-1}}\prs{f}}\]
for all $g \in G$.
We have to show that $D_X\prs{f} \in \mbb{F}\brs{G}$, and then check that $D_X$ is a derivation.

Let $m \colon G \times G \to G$ be the group product. Then for $f \in \mbb{F}\brs{G}$ with
\[f\prs{xy} = \sum_i u_i\prs{x} v_i\prs{y}\]
we have
\[\lambda\prs{g}\prs{f} = \sum_i u_i\prs{g^{-1}} v_i\]
and
\begin{align*}
D_X\prs{f}\prs{g} = \sum_i u_i\prs{g^{-1}} X\prs{v_i} \in \mbb{F}\brs{X} \text{.}
\end{align*}

Now, $D_X \in \Lie\prs{G}$ because for $h \in G$ we have
\begin{align*}
D_X\prs{\lambda\prs{h}\prs{f}}\prs{g} &= X\prs{\lambda\prs{\prs{h^{-1} g}^{-1}}\prs{f}}
\\&= \lambda\prs{h} D_X\prs{f} \prs{g}
\end{align*}
so \[D_X\prs{\lambda\prs{h} f} = \lambda\prs{h} D_X\prs{f} \text{.}\]


\end{description}
\end{proof}

Knowing the above proposition, we sometimes think of $\Lie\prs{G}$ as the tangent space $T_e\prs{G}$.
In fact, we could define a Lie algebra structure on $T_e\prs{G}$ directly by
\begin{equation}\label{equation:direct_lie_bracket}
\brs{X,Y} \coloneqq \prs{X \otimes Y - Y \otimes X} \circ m^*
\end{equation}
for every $X,Y \in T_e\prs{G} = \Der_{\mbb{F}}\prs{\mbb{F}\brs{G}, \mbb{F}\prs{G}}$ and where $m \colon G \times G \to G$ is the group-multiplication. But, this might lack motivation.

From the above proposition, we can identify
$T_e\prs{G} \cong \Lie\prs{G}$, and, pulling back this Lie-algebra structure we can define the bracket
$\brs{X,Y} \in T_e G$ for any $X,Y \in T_e\prs{G}$.

For an embedded algebraic group $G \leq \mrm{GL}_n\prs{\mbb{F}}$ we get an embedding $\Lie\prs{G} \cong T_e\prs{G} \leq M_n\prs{\mbb{F}}$. Write $\mbb{F}\brs{G} = \mbb{F}\brs{\set{T_{i,j}}_{i,j \in [n]}}/I$. Any $X \in T_e\prs{G}$ is determined by its action on generators. Write $x_{i,j} \coloneqq X\prs{T_{i,j}}$. We have
\begin{align*}
m^* \colon \mbb{F}\brs{G} &\to \mbb{F}\brs{G} \times \mbb{F}\brs{G} \\
T_{i,j} &\mapsto \sum_{k \in [n]} T_{i,k} \otimes T_{k,j} \text{.}
\end{align*}
Then
\[D_X\prs{T_{i,j}} = \sum_{k \in [n]} T_{i,k} x_{k,j} \text{.}\]
Hence $D_X\prs{T_{i,j}} \in M_n\prs{\mbb{F}\brs{G}}$ and
\begin{align*}
\prs{D_X\prs{T_{i,j}}}_{i,j} &= \prs{T_{i,j}} \cdot X
\end{align*}
under the identification of $X$ as the matrix $\prs{x_{i,j}}_{i,j \in [n]}$ in $M_n\prs{\mbb{F}\brs{G}}$.

Now,
\begin{align*}
\brs{X,Y}\prs{f} &= \prs{D_X \circ D_Y - D_Y \circ D_X} \prs{f}\prs{e}
\\&= X\prs{D_Y\prs{f}} - Y\prs{D_X\prs{f}} \text{.}
\end{align*}
Under the embedding we have
\begin{align*}
Y\prs{D_X\prs{T_{i,j}}} = \prs{YX}_{i,j}
\end{align*}
so
\[\brs{X,Y}\prs{T_{i,j}} = \prs{XY - YX}_{i,j} \text{.}\]

We then get the under the identification with matrices, $\brs{X,Y} = XY - YX$. I.e. for every embedding $G \rmono \mrm{GL}_n\prs{\mbb{F}}$ we get an isomorphism from $\Lie\prs{G}$ to a Lie subalgebra of $M_n\prs{\mbb{F}}$.

\begin{exercise}
Check that the definition of $\brs{\cdot,\cdot}$ on $T_e\prs{G}$ is the same as that of \eqref{equation:direct_lie_bracket}.
\end{exercise}

\begin{proposition}
Let $\phi \colon G \to H$ be a homomorphism of algebraic groups. Then
\[\diff \phi = \prs{\diff \phi}_e \colon \Lie\prs{G} \to \Lie\prs{H}\]
satisfies
\[\brs{\diff \phi\prs{X}, \diff \phi\prs{Y}} = \diff \phi \brs{X,Y} \text{.}\]

In other words, $\diff \phi \colon \Lie\prs{G} \to \Lie\prs{H}$ is a Lie-algebra homomorphism.
\end{proposition}

\begin{proof}
$\phi$ is a homomorphism, so $\phi \circ m_G = m_H \circ \prs{\phi \times \phi}$, so
\[\prs{\phi^* \otimes \phi^*} \circ m_H^* = m_G^* \circ \phi^* \text{.}\]
Then
\begin{align*}
\brs{\diff \phi\prs{X}, \diff \phi\prs{Y}} &= \prs{\prs{X \circ \phi^*} \otimes \prs{Y \circ \phi^*} - \prs{Y \circ \phi^*} \otimes \prs{X \circ \phi^*}} \circ m_H^*
\\&= \prs{X \otimes Y - Y \otimes X} \circ m_G^* \circ \phi^*
\\&= \brs{X,Y} \circ \phi^*
\\&= \diff \phi\prs{\brs{X,Y}} \text{.}
\end{align*}
\end{proof}



Let $G$ be an affine $\mbb{F}$-algebraic group.
$G$ acts on itself by conjugation.
For $g \in G$, denote
\begin{align*}
\mrm{In}\prs{g} \colon G &\to G \\
h &\mapsto g h g^{-1} \text{.}
\end{align*}
Then
\[\mrm{Ad}\prs{g} \coloneqq \diff \prs{\mrm{In}\prs{g}}_e \colon \Lie\prs{G} \to \Lie\prs{G}\]
gives a regular map
\[\mrm{Ad} \colon G \to \mrm{GL}\prs{\Lie\prs{G}}\]
which is a representation of $G$ called \emph{the adjoint representation of $G$}.
We get a map
\begin{align*}
\ad \coloneqq \diff\Ad \colon \Lie\prs{G} \to \End\prs{\Lie\prs{G}}
\end{align*}
such that
\[\ad\prs{X}\prs{Y} = \brs{X,Y} \text{.}\]

\begin{exercise}
We could define $\brs{X,Y}$ as $\ad\prs{X}\prs{Y}$.
\end{exercise}

\begin{remark}
From the above proposition, one has
\[\brs{\ad\prs{X}, \ad\prs{Y}} = \ad\brs{X,Y} \text{,}\]
which is equivalent to Jacobi's identity.
\end{remark}

\begin{exercise}
In finite dimensional it holds that $T_e\prs{G}^\circ = T_e\prs{G}$.
\end{exercise}

\begin{corollary}
One has $\dim \Lie\prs{G} = \dim T_e\prs{G} = \dim\prs{G^\circ}$.
\end{corollary}

\section{Categorical Approach to Algebraic Groups}

\subsection{Definitions}

An affine algebraic variety is defined by polynomial equations, $X = V\prs{f_1, \ldots, f_k} \leq \mbb{F}^n$ for $\prs{f_i}_{i \in [k]} \mbb{F}\brs{x_1, \ldots, x_n}$.
One can treat the equations $f_i = 0$ over any (commutative) $\mbb{F}$-algebra $R$.

\begin{definition}[Functor of Points]
Let $X$ be an affine algebraic variety over $\mbb{F}$, and let $R$ be an $\mbb{F}$-algebra.
Define
\[X\prs{R} = \hom_{\mbb{F}\mathbf{-Alg}}\prs{\mbb{F}\brs{X}, R}\]
and call the resulting functor $R \mapsto X\prs{R}$ the \emph{functor of points of $X$}.
\end{definition}

\begin{definition}[Category]
A \emph{(small) category} $\mcal{C}$ is a \emph{collection} (the exact type of which depends on the axiomatic approach) $\ob\prs{\mcal{C}}$ together with a \emph{set} of morphism $\hom_{\mcal{C}}\prs{X,Y}$ for every $X,Y \in \mcal{C}$ and a \emph{composition map}
\begin{align*}
\hom\prs{X,Y} \times \hom\prs{Y,Z} &\to \hom\prs{X,Z} \\
\prs{f,g} &\mapsto g \circ f
\end{align*}
satisfying the following.

\begin{enumerate}
\item For all $X \in \ob\prs{\mcal{C}}$ there's $\id_X \in \hom_{\mcal{C}}\prs{X,X}$ such that $\id_X \circ f = f$ and $g \circ \id_X$ for all $f,g$ (which make sense).
\item Composition is associative.
\end{enumerate}
\end{definition}

\begin{example}
\begin{enumerate}
\item Sets with functions between them form a category.
\item Groups with group homomorphisms.
\item $\mbb{F}$-algebras with their homomorphisms.
\end{enumerate}
\end{example}

\begin{definition}[Functor]
A \emph{(covariant) functor} $F \colon \mcal{C} \to \mcal{D}$ between categories is a ``function'' $F \colon \ob\prs{\mcal{C}} \to \ob\prs{\mcal{D}}$ and for each $X,Y \in \ob\prs{\mcal{C}}$ a function
\[F \colon \hom_{\mcal{C}}\prs{X,Y} \to \hom_{\mcal{D}}\prs{F\prs{X},F\prs{Y}}\]
such that $F$ respects composition and identities.
\end{definition}

\begin{definition}
If instead of respecting composition, $F$ satisfies $F\prs{g \circ f} = F\prs{f} \circ F\prs{g}$, we call $F$ a \emph{contravariant functor}.
\end{definition}

\begin{example}
The correspondence $X \mapsto \mbb{F}\brs{X}$ and $\phi \mapsto \phi^*$ forms a functor from affine $\mbb{F}$-varieties to commutative $\mbb{F}$-algebras.
\end{example}

\begin{example}
There is a functor $F \colon \mathbf{Vec}_{\mbb{F}} \to \mathbf{Vec}_{\mbb{F}}$ from the category of $\mbb{F}$-vector spaces to itself, sending $V \mapsto V^*$ and $T \mapsto T^*$.
\end{example}

\begin{example}
We have a functor $G \mapsto \Lie\prs{G}$ sending an affine algebraic group over $\mbb{F}$ to an vector space over $\mbb{F}$.
\end{example}

\begin{definition}
We call a functor which ``forgets'' some of the structure a \emph{forgetful functor.}
\end{definition}

\begin{definition}
Let $\mcal{C}$ be a category, and let $X \in \ob\prs{\mcal{C}}$. Define a (covariant) functor
\begin{align*}
h^X \colon \mcal{C} &\to \mathbf{Set} \\
Y &\mapsto \hom_{\mcal{C}}\prs{X,Y} \\
\phi &\mapsto \prs{f \mapsto \phi \circ f} \text{.}
\end{align*}
\end{definition}

\begin{corollary}
Let $X$ be an affine variety over $\mbb{F}$. One can see $X$ as a functor $X \colon \mbb{F}\mathbf{-Alg} \to \mathbf{Set}$ sending $R$ to $X\prs{R}$. This is the functor $X = h^{\mbb{F}\brs{X}}$.
\end{corollary}

\begin{definition}[Natural Transformation]
Let $F, G \colon \mcal{C} \to \mcal{D}$ be functors between categories. A \emph{natural transformation} $\eta \colon F \to G$ is the data $\eta_X \in \hom_{\mcal{D}}\prs{F\prs{X}, G\prs{X}}$ for all $X \in \ob\prs{\mcal{C}}$, such that the following diagram commutes for every $X,Y \in \ob\prs{\mcal{C}}$.

\[
\begin{tikzcd}
F\prs{X} \arrow[r, "\eta_X"] \arrow[d, "F\prs{\phi}"] & G\prs{X} \arrow[d, "G\prs{\phi}"] \\
F\prs{Y} \arrow[r, "\eta_Y"] & G\prs{Y}
\end{tikzcd}
\]
\end{definition}

\begin{corollary}
The collection $\mathbf{Fun}\prs{\mcal{C}, \mcal{D}}$ of functors $\mcal{C} \to \mcal{D}$ is a category together with natural transformations as morphisms.
\end{corollary}

\begin{example}
Consider the functors $\id_{\mbb{F}\mathbf{-Vec}}$ and $\prs{\prs{-}^*}^*$ on the category of $\mbb{F}$-vector spaces. The canonical identification $V \cong \prs{V^*}^*$ is a natural transformation $\id_{\mbb{F}\mathbf{-Vec}} \to \prs{\prs{-}^*}^*$.
\end{example}

\begin{theorem}[Yoneda's Lemma]\label{theorem:yoneda_lemma}
Let $\mcal{C}$ be a category and let $\phi \in \hom_{\mcal{C}}\prs{X,Y}$. Consider the induced natural transformation $T\prs{\phi} \in \hom\prs{h^Y, h^X}$ with
\begin{align*}
T\prs{\phi}_Z \colon \hom\prs{Y,Z} &to \hom\prs{X,Z} \\
f &\mapsto f \circ \phi \text{.}
\end{align*}
Then
\[T \colon \hom_{\mcal{C}}\prs{X,Y} \to \hom_{\mathbf{Fun}\prs{\mcal{C}, \mathbf{Set}}}\prs{h^Y, h^X}\]
is a bijection.
\end{theorem}

\begin{corollary}
Any $\mcal{C}$ is embedded as a subcategory of $\mathbf{Fun}\prs{\mcal{C}, \mathbf{Set}}$.
\end{corollary}

%29.12.21

\begin{definition}[Yoneda Embedding]
Define the \emph{Yoneda embedding} to be the contravariant functor
\begin{align*}
\mcal{C} &\to \mathbf{Fun}\prs{\mcal{C},\mathbf{Set}} \\
X &\mapsto h^X \text{.}
\end{align*}
This is an \emph{embedding} due to Yoneda's lemma.
\end{definition}

\begin{remark}
We saw that $X\prs{R}$ can be thought of as the solution of solutions defined in $\mbb{F}\brs{X}$ over $R$. This is a way to construct geometric objects from an $\mbb{F}$-algebra $R$.
\end{remark}

\begin{definition}[Corepresentable Functor]
A functor $F \colon \mcal{C} \to \mathbf{Set}$ is \emph{corepresentable} if there's $X \in \ob\prs{\mcal{C}}$ such that $F \cong h^X$.
\end{definition}

\newcommand{\mbf}[1]{\mathbf{#1}}

\begin{definition}[$\mbb{F}$-Affine Scheme]
An \emph{$\mbb{F}$-affine scheme} is a corepresentable functor $X \colon \mbb{F}\mbf{-Alg} \to \mbf{Set}$.
If $X$ is corepresentable by a finitely-generated object, we say $X$ is an \emph{algebraic scheme}.
\end{definition}

\begin{remark}
Remembering our definition of affine varieties over $\mbb{F}$, the algebra of regular  functions has no non-trivial nilpotent elements. So, the case of $\mbb{F}$-schemes is more general, but one has a similar duality.
\end{remark}

\begin{definition}
For an $\mbb{F}$-scheme $X \colon \mbb{F}\mbf{-Alg} \to \mbf{Set}$, let
\[A_X \coloneqq \hom_{\mbb{F}}\prs{X, \mbb{A}^1}\]
where $\mbb{A}^1 \coloneqq h^{\mbb{F}\brs{t}}$.
\end{definition}

Any $f \in \hom\prs{X,\mbb{A}^1}$ (a natural transformation between the functors) is composed of functions $f_R \colon X\prs{R} \to \mbb{A}^1\prs{R}$, such thatfor a morphism $t \colon R \to S$ the following diagram commutes.
\[
\begin{tikzcd}
X\prs{R} \arrow[r, "f_R"] \arrow[d, "X\prs{t}"] & \mbb{A}^1\prs{R} \arrow[d, "t"] \\
X\prs{S} \arrow[r, "f_S"] & S
\end{tikzcd}
\]
In particular, $A_X$ is an $\mbb{F}$-algebra. We get a natural transformation
\[\alpha^X \in \hom\prs{X, h^{A_X}}\]
with
\begin{align*}
\alpha^X_R \colon X\prs{R} &\to \hom\prs{A_X, R} \\
g &\mapsto \prs{f_R \mapsto f_R\prs{g}} \text{.}
\end{align*}
This is analogous to $\mrm{ev}_g$.

\begin{proposition}
$X \colon \mbb{F}\mbf{-Alg} \to \mbf{Set}$ is corepresentable (i.e. a scheme) if and only if $\alpha^X$ is an isomorphism.
\end{proposition}

\begin{proof}
If $\alpha^X$ is an isomorphism, $X \cong h^{A_X}$ which is a scheme by definition.

If $X \cong h^B$ then
\[A_X = \hom\prs{h^B, h^{\mbb{F}\brs{t}}} \underset{\mrm{Yoneda}}{\cong} \hom\prs{\mbb{F}\brs{t}, B} \cong B \text{.}\]
Hence $X \cong h^{A_X}$. Check that then $\alpha^X \in \hom\prs{h^{A_X}, h^{A_X}}$ is a natural isomorphism.
\end{proof}

\backmatter
\end{document}