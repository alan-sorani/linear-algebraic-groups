\documentclass[10pt,a4paper,twoside,openany,hidelinks]{book}

%JAPANESE
\usepackage{fontspec}
\defaultfontfeatures{Ligatures={NoCommon, NoDiscretionary, NoHistoric, NoRequired, NoContextual}}
\setmainfont{CMU Serif}
\usepackage{xeCJK}
\xeCJKsetup{CJKmath=true}
\setCJKmainfont{MS Mincho} % or some other Japanese font

\usepackage{maths}
\usepackage{stylish}

%\usepackage{tikz-graph}
%\GraphInit[vstyle = Shade]

\title{Lecture Notes to Linear Algebraic Groups \\ \large{Winter 2021-2022, Technion IIT}}
\author{Typed by Elad Tzorani}
\date{\today}

\usepackage{lipsum}
\usepackage{stmaryrd}

\tikzset{EdgeStyle/.style   = {thick,
                               double          = orange,
                               double distance = 1pt}}

\tikzset{
    labl0/.style={anchor=south, rotate=-30, inner sep=.5mm}
}

\tikzset{
    labl1/.style={anchor=south, rotate=30, inner sep=.5mm}
}

\tikzset{
    labl2/.style={anchor=south, rotate=60, inner sep=.5mm}
}

\tikzset{
    labl3/.style={anchor=south, rotate=90, inner sep=.5mm}
}

\tikzset{
    labl4/.style={anchor=south, rotate=-90, inner sep=.5mm}
}

\usepackage{scalerel,stackengine}

\begin{comment}
\stackMath
\newcommand\widecheck[1]{%
\savestack{\tmpbox}{\stretchto{%
  \scaleto{%
    \scalerel*[\widthof{\ensuremath{#1}}]{\kern-.6pt\bigwedge\kern-.6pt}%
    {\rule[-\textheight/2]{1ex}{\textheight}}%WIDTH-LIMITED BIG WEDGE
  }{\textheight}% 
}{0.5ex}}%
\stackon[1pt]{#1}{\scalebox{-1}{\tmpbox}}%
}
\parskip 1ex
\end{comment}

\newcommand{\from}{\leftarrow}

\usepackage[nottoc]{tocbibind}

\newcommand{\Ch}[1]{\catname{Ch}\catname{(mod-}#1\catname{)}}
\newcommand{\Chb}[1]{\catname{Ch}_{{b}}\catname{(mod-}#1\catname{)}}
\newcommand{\Chm}[1]{{\catname{Ch}_{{-}}\catname{(mod-}R\catname{)}}}
\newcommand{\Chp}[1]{{\catname{Ch}_{{+}}\catname{(mod-}R\catname{)}}}

\begin{document}
\frontmatter
\frontpage{front}{0.8\textwidth}{A cat.}
\tableofcontents
\countlectures
\newpage

\chapter*{Preface}
\addcontentsline{toc}{chapter}{Preface} \markboth{Preface}{}

\section*{Technicalities}
\addcontentsline{toc}{section}{Technicalities} %\markboth{Technicalities}{}

These aren't formal notes related to the course and henceforward there is \emph{absolutely no guarantee} that the recorded material is in correspondence with the course expectations, or that these notes lack any mistakes.\\
In fact, there probably are mistakes in the notes! I would highly appreciate if any comments or corrections were sent to me via email at \href{mailto:tzorani.elad@gmail.com}{tzorani.elad@gmail.com}.\\
Elad Tzorani.

\section*{Grade}

The course grade will consist of the following.

\begin{itemize}
\item 60\% for homework
\item 40\% for giving lectures on more advanced topics at the end of the semester
\end{itemize}

\mainmatter

\chapter{Linear Algebraic Groups}

\section{Preliminaries}

\subsection{Motivation \& Historical Background}

\subsubsection{Linear Algebraic Groups From Differential Equations}

Algebraic groups developed from the study of Lie groups. The latter were studied by Sophus Lie around 1870 in the context of differential equations. Lie groups can describe symmetries of solutions of differential equations; e.g. solutions of $\nabla y = 0$ are \emph{harmonic functions} and one is interested in linear isomorphisms $g \colon \mbb{R}^n \to \mbb{R}^n$ such that $\Delta\prs{y} = 0$ implies $\Delta\prs{y \circ g} = 0$. Lie noticed that such $g$ form a group $\mcal{O}_n\prs{\mbb{R}} \coloneqq \set{g \in \mrm{GL}_n\prs{\mbb{R}}}{g^T g = I_n}$.
Such groups for operators different than $\Delta$ are smooth manifolds with smooth group actions, called \emph{Lie groups.}
One of Lie's motivation was to have Galois theory for differentiable equations. It had already been known that in order to find roots of polynomials one uses the symmetries of field extensions.

Around 1880, Picard looked at differentiable equations of the form
\begin{align*}
\frac{\prs{\diff y}^n}{\diff x}^n + p_1\prs{x} \frac{\diff y^{n-1}}{\prs{\diff x}^{n-1}} + \ldots + p_n\prs{x} y = 0
\end{align*}
for $p_i$ rational functions.
The solution space for such an equation is the $n$-dimensional space
\[\spn\set{y_1\prs{x}, \ldots, y_n\prs{x}} \text{.}\]
Picard looked at a subgroup $G \leq \mrm{GL}_n\prs{\mbb{R}}$
which preserves the algebraic dependencies of the $y_i$ (i.e. preserves polynomials $p \in \mbb{R}_n\brs{x}$ for which $p\prs{y_1\prs{x}, \ldots, y_n\prs{x}} = 0$). These were the first treatments of algebraic groups.

Around 1870-1900, Mauren took homogeneous rational functions $f \colon \mbb{C}^n \to \mbb{C}$ (such as $f\prs{x_1, \ldots, x_n} = \sum_{i=1}^n x_i^2$ for which $G_f = \mcal{O}_n\prs{\mbb{C}}$) and studied the structure of
\[G_f \coloneqq \set{g \in \mrm{GL}_n\prs{\mbb{C}}}{f \circ g = f} \text{.}\]
One can take $f$ to be any quadratic form, e.g.
\[\mrm{Sp}_{2n}\prs{\mbb{C}} = \set{g \in \mrm{GL}_{2n}\prs{\mbb{C}}}{g^T \pmat{0 & I_n \\ -I_n & 0} g = \pmat{0 & I_n \\ -I_n & 0}}\]
is such  group.
Such groups are called \emph{classical groups}.

Mauren looked at the tangent space of such group. His motivation was his interest in Hilbert's 14\textsuperscript{th} problem:
Given $g \in G \leq \mrm{GL}_n\prs{\mbb{C}}$ we can consider $g$ as a map $\mbb{C}^n \to \mbb{C}^n$. Considering the action $G \acts \mbb{C}\brs{x_1, \ldots, x_n}$ on the coefficient, the problem is understanding the invariant space of this action.
E.g. the invariant space of $S_n$ are \emph{symmetric polynomials}.

\subsubsection{Later Developments}

The field of Lie groups gave great success. Semisimple Lie groups have complete combinatorical classification due to Cartan and Killing. This is considered one of the greatest achievements in mathematics.

Chevalley found out the every semisimple Lie group is defined by polynomials in integer coefficients, circa 1940. This led to the definition of algebraic groups and a new goal: to algebrize Lie theory and develop tools to study smooth symmetric ``without analysis'' and over more general fields. This should form a bridge between continuous groups and finite groups.
Chevalley used in his studies of the subject the formal expression
\[\mrm{exp}\prs{X} = \sum_{n=0}^\infty \frac{1}{n!} X^n\]
and required $\mrm{char}\prs{\mbb{F}} = 0$.
Later Kolchin returned to Picard's ideas and developed a differential Galois theory over a general field.

\subsubsection{Modern Developments}

From 1950 onwards, many mathematicians developed the study of algebraic groups, which was possible thanks to advances in algebraic geometry.
Some of the advances of the field are the following.

\begin{enumerate}
\item The classification of finite simple groups. Most of these groups are of ``Lie type'', which are of the form $\mrm{Sp}_{2n}\prs{\mbb{F}_q}$.

\item Results on \href{https://ncatlab.org/nlab/show/p-adic+completion}{$p$-adic} groups. For example, Bruhat-Tits buildings are homogeneous spaces with $p$-adic group actions and which are ``non-archimedean'' analogues to classical symmetric spaces.

\item Results in number theory.
\end{enumerate}

\subsubsection{The Langlands Program}

The Laglands program, circa 1960, tries to study properties in number theory through the study of groups. There are analogues to Riemann's zeta function which one hopes all arise from group actions in the following way.
Taking an algebraic group $G$, one looks at \emph{automorphic spaces} $V$ with $G\prs{\mbb{R}}$ and $G\prs{\mbb{Q}_p}$ actions which commute with each other, for some groups $G\prs{\mbb{R}}, G\prs{\mbb{Q}_p}$ over the respective fields.

\subsection{Definitions \& Course Goals}

\subsubsection{What are Algebraic Groups?}

Write $\mbb{F}$ for a field, and write $M_n\prs{\mbb{F}} \cong \mbb{F}^{n^2}$ for the space of $n \times n$ matrices over $\mbb{F}$.

\begin{definition}[Affine Algebraic Group]
A subset $G \leq M_n\prs{\mbb{F}}$ closed under multiplication and inverse is called an \emph{affine algebraic group} over $\mbb{F}$ if there are $f_1, \ldots, f_k \in \mbb{F}\brs{\set{x_{i,j}}_{i,j \in [n]}}$ such that
\[G = \set{A \in M_n\prs{\mbb{F}}}{f_1\prs{A} = \ldots = f_k\prs{A} = 0} \text{.}\]
\end{definition}

\begin{example}
$\mrm{SL}_n\prs{\mbb{F}}$ is n affine algebraic group. The determinant,
\[\det\prs{X} = \sum_{\sigma \in S_n} \prs{-1}^{\sgn\prs{\sigma}} x_{1,\sigma\prs{1}} \cdot \ldots \cdot x_{n,\sigma\prs{n}} \text{,}\]
is a polynomial and $\mrm{SL}_n\prs{\mbb{F}} = \set{A \in M_n\prs{\mbb{F}}{\det\prs{A} - 1 = 0}}$.
\end{example}

\begin{example}
Let $Q \in M_n\prs{\mbb{F}}$ and denote
\[\mcal{O}_{Q}\prs{\mbb{F}} \coloneqq \set{A \in \mrm{GL}_n\prs{\mbb{F}}}{A^t Q A = Q} \text{.}\]
Taking $Q = I_n$ one gets $\mcal{O}_Q\prs{\mbb{R}} = \mcal{O}_n\prs{\mbb{R}}$. More generally, matrix multiplication is polynomial and one can write $A^t Q A - Q = 0$ as a polynomial equation in the coefficients of $A$. We explain the condition $A \in \mrm{GL}_n\prs{\mbb{F}}$ later.
\end{example}

\begin{example}
Let $N \subseteq M_n\prs{\mbb{F}}$ be the subset of upper-triangular matrices with $1$ on the diagonal. This is an algebraic group with polynomial conditions $x_{i,j} = 0$ for $i > j$ and $x_{i,i} = 1$ for all $i \in [n]$.
One has $N \cong \mbb{F}^{\frac{n\prs{n-1}}{2}}$ as vector space, but this doesn't remember the group structure.
\end{example}

\begin{example}
The vector space $\mbb{F}^n$ with addition is an algebraic group.
We have
\begin{align*}
V \coloneqq \set{\pmat{1 & 0 & \cdots & 0 & * \\ 0 & & & & * \\ \vdots & & \ddots & & \vdots \\ & & & & * \\ 0 & & \cdots & & 1} \in M_{n+1}\prs{\mbb{F}}} \cong \mbb{F}^n \text{.}
\end{align*}
One denotes $G_a\prs{\mbb{F}} \coloneqq \prs{\mbb{F}, +}$ and calls this \emph{the additive group over $\mbb{F}$}.
\end{example}

\begin{remark}
One has
\begin{align*}
\mrm{GL}_n\prs{\mbb{F}} = \set{A \in M_n\prs{\mbb{F}}}{\det\prs{A} \neq 0}
\\&\cong \set{\pmat{A & 0 \\ 0 & a} \in M_{n+1}\prs{\mbb{F}}}{\det\prs{A} \cdot a = 1}
\end{align*}
and a bijection $A \leftrightarrow \pmat{A & 0 \\ 0 & \det\prs{A}^{-1}}$, but this looks weird. We then want the definition to be more general and capture groups that are isomorphic to what we defined as affine algebraic groups. We do that later in the course.
\end{remark}

\begin{example}
Every finite group $G$ is an algebraic group. One has an inclusion $G \rmono S_n$, and $S_n$ is an algebraic group where $\sigma$ is considered as $\prs{x_{i,j}}_{i,j \in [n]}$ with $x_{i,\sigma\prs{i}} = 1$ and $x_{i,j} = 0$ for any other $i,j \in [n]$.
\end{example}

\begin{exercise}
Every finite subset of $M_n\prs{\mbb{F}}$ is an algebraic set, in the sense that it's defined by the vanishing of polynomials.
\end{exercise}

To study properties of algebraic groups, one needs to use tools from algebraic geometry. Here there are two possible difficulties:

\begin{enumerate}
\item One needs to ask what generality is to be worked with. With our current definition it is difficult to use strong algebro-geometric tools, but with ``too general'' definitions it is more difficult to look at simple examples.

\item One should decide how much they want to rely on geometric results as facts and how much is to be proved.
\end{enumerate}

Our answer to the latter question is proving things at the beginning of the course and later on taking more things as facts. For the first difficulty\ldots you'll see as we go.

\subsubsection{A Course Overview}

During the course we plan to go over the following.

\begin{itemize}
\item Basic algebraic geometry.

\item General structure properties of algebraic groups. For example, a generalization of Jordan's decomposition to $\mrm{GL}_n$.

\item Generalization of the notion of an algebraic group.

\item Study of algebraic groups by looking at algebraic groups over the Galois closure and via Galois theory.

\item The classification of reductive groups over algebraically closed fields. An algebraic version of the Cartan-Killing classification.
\end{itemize}

\subsection{Preliminary Algebraic Geometry}

\subsubsection{Embedded $\mbb{F}$-Affine Varieties}

\begin{notation}
Denote $A_n \coloneqq \mbb{F}\brs{x_1, \ldots, x_n}$.
\end{notation}

\begin{definition}
For $C \subseteq A_n$ define
\[V\prs{C} \coloneqq \set{p \in \mbb{F}^n}{\forall f \in C \colon f\prs{p} = 0} \subseteq \mbb{F}^n \text{.}\]
A set of this form is called an \emph{embedded $\mbb{F}$-Affine Variety}.
\end{definition}

\begin{definition}
For $S \subseteq \mbb{F}^n$ define
\begin{align*}
I\prs{S} \coloneqq \set{f \in A_n}{\forall p \in S \colon f\prs{p} = 0} \text{.}
\end{align*}
\end{definition}

\begin{exercise}
For $S \subseteq \mbb{F}^n$ one has $I\prs{S} \ideal A_n$.
\end{exercise}

\begin{example}
One has $I\prs{\ns} = A_n$ and whenever $\mbb{F}$ is infinite one has $I\prs{\mbb{F}^n} = \set{0}$.
\end{example}

\begin{example}
For $I = \set{x_1^2 - x_2, x_1^3 - x_3}$ one has $V\prs{I} = \set{\prs{x, x^2, x^3}}{x \in \mbb{F}}$ which one calls the \emph{twisted cubic} over $\mbb{F}$
\end{example}

\begin{proposition}
One notices that for $S \subseteq \mbb{F}^n$ and $C \subseteq A_n$ we have
\begin{align*}
S &\subseteq V\prs{I\prs{S}} \\
C \subseteq I\prs{V\prs{C}} \text{.}
\end{align*}
\end{proposition}

\begin{definition}[The Zariski Topology]
The \emph{Zariski topology} on $\mbb{F}^n$ is the topology given by taking sets of the form $V\prs{C}$ for $C \subseteq A_n$ as the closed subsets.
\end{definition}

\begin{exercise}
Check that the above definition gives a well-defined topology.
\end{exercise}

\begin{exercise}
For $S \subseteq \mbb{F}^n$ one has $\overline{S} = V\prs{I\prs{S}}$.
\end{exercise}

\begin{example}
Consider the case $n=1$. Then closed subsets of $\mbb{F}$ are sets of the form $V\prs{C}$ for $C \subseteq A_n$. If $C$ contains a nonzero polynomial, $V\prs{C}$ is finite, and otherwise $V\prs{C} = \mbb{F}$. We get that the nontrivial closed sets are exactly the finite subsets of $\mbb{F}^n$.
\end{example}

\begin{remark}
$\mbb{F}^n$ with the Zariski topology is always \emph{quasi-compact}, meaning it's compact but not Hausdorff.
\end{remark}

\begin{theorem}[Hilbert's Basis Theorem]\label{theorem:basis_theorem}
Every ideal $I \ideal A_n$ is finitely-generated.
\end{theorem}

\begin{proof}
We prove the statement by induction on $n \in \mbb{N}$. The case $n = 0$ is trivial since $\mbb{F}$ is a field. Assume the statement is true for $n-1$, we show it for $n$.
Write $A_n \cong A_{n-1}\brs{x_n}$ and assume $I \ideal A_n$ is nonzero. Choose $f_1 \in I \setminus \set{0}$ of minimal degree and write $d_1 \coloneqq \deg_{A_{n-1}}{f_1}$. If $\prs{f_1} \neq I$, choose $f_2 \in I\setminus\prs{f_1}$ of minimal degree $d_2 \coloneqq \deg_{A_{n_1}}\prs{f_2}$. Continue this way to get $f_i$ with $d_i \coloneqq \deg_{A_{n-1}}\prs{f_i}$ and $d_1 \leq d_2 \leq d_3 \leq \ldots $.

Assume that this doesn't end a finite point (for otherwise we're done). Denote by $a_i \in A_{n-1}$ the leading coefficient of $f_i$. By assumption, $I' \coloneqq \prs{a_1, \ldots, a_i, \ldots} \ideal A_{n-1}$ is finitely-generated. We can then write $I' = \prs{a_1, \ldots, a_h}$ for some $h \in \mbb{N}$. Then
\[a_{h+1} = x_1 a_1 + \ldots + x_h a_h\]
for some $x_1, \ldots, x_h \in A_{n-1}$.
Let
\[g \coloneqq f_{h+1} - \sum_{i \in [h]} x_i \cdot f_i \cdot x^{d_{h+1} - d_i} \in I \text{.}\]
The coefficient of $x^{d_{h+1}}$ in $g$ vanishes so $\deg\prs{g} < d_{h+1}$ and $g \in \prs{f_1,\ ldots, f_h}$. Then also $f_{h+1} \in \prs{f_1, \ldots, f_h}$, in contradiction.
\end{proof}

\begin{definition}[Noetherian Topological Space]
A topological space $X$ is called \emph{Noetherian} if every decreasing sequence of closed subsets stabilises.
\end{definition}

\begin{corollary}
$\mbb{F}^n$ with the Zariski topology is Noetherian.
\end{corollary}

\begin{proof}
Let $\prs{X_i}_{i \in \mbb{N}_+} \subseteq \mbb{F}^n$ be a decreasing sequence of closed subsets, and for every $i$ denote $I_i \coloneqq I\prs{X_i}$. Then $X_i = V\prs{I_i}$. We get $I_1 \subseteq I_2 \subseteq I_3 \subseteq \ldots$. Denote $I = \bigcup_{i \in \mbb{N}_+} I_i$.  By Theorem \ref{theorem:basis_theorem} we have $I = \prs{f_1, \ldots, f_k}$ for some $k \in \mbb{N}_+$. Then $f_1, \ldots, f_k \in I_m$ for some $m \in \mbb{N}_+$. We get $I = I_m = I_{m+1} = \ldots$ and $X_m = X_{m+1} = X_{m+2} = \ldots$, as required.
\end{proof}

\begin{exercise}
Every closed subspace $X \subseteq \mbb{F}^n$ is quasi-compact.
\end{exercise}

\begin{example}
Consider $xy \in \mbb{F}\brs{x,y}$. $V\prs{x,y}$ is connected, but we would like to say it has two components. E.g. if $\mbb{F} = \mbb{R}$, the set $V\prs{x,y}$ is the union of two perpendicular axes. This leads to the following definition.
\end{example}

\begin{definition}[Irreducible Topological Space]
A topological space $X$ is \emph{irreducible} if there aren't strict closed subsets $X_1, X_2 \subsetneq X$ such that $X = X_1 \cup X_2$.
\end{definition}

\begin{exercise}
An irreducible Hausdorff topological space is a point.
\end{exercise}

\begin{exercise}
In a Noetherian space $X$ there are finitely many maximal irreducible subsets $X_1, \ldots, X_k$, and $X = \bigcup_{i \in [k]} X_i$.
\end{exercise}

\begin{proposition}
An algebraic variety $V \subseteq \mbb{F}^n$ is irreducible if and only if $I\prs{V}$ is prime.
\end{proposition}

\begin{proof}
Assume $V$ is irreducible. Let $f_1, f_2 \in A_n$, such that $f_1 f_2 \in I\prs{V}$, we want to show $f_1 \in I\prs{V}$ or $f_2 \in I\prs{V}$. We have $V \subseteq V\prs{f_1 f_2} = V\prs{f_1} \cup V\prs{f_2}$. Now $V = \prs{V \cap V\prs{f_1}} \cup \prs{V \cap V\prs{f_2}}$ and by irreducibility $V = V \cap V\prs{f_i}$ for $i \in [2]$, in which case $V \subseteq V\prs{f_i}$ and therefore $f_i \in I\prs{V}$.

The other direction is left as an exercise.
\end{proof}

%LECTURE 03.11.2021

\begin{example}
Consider $G \coloneqq \mrm{GL}_1\prs{\mbb{F}} \cong \mbb{F}^\times \subseteq \mbb{F}$.
We have $V\prs{x-1} = \set{1}$ and similarly $V\prs{\prs{x-1}^2} = \set{1}$. If $\mbb{F} = \mbb{R}$, one has $V\prs{x^5 - 1} = \set{1}$ and one gets $V\prs{x^5 - 1} = V\prs{x-1}$.
However, over $\mbb{F} = \mbb{C}$ the group $V\prs{x^5 - 1}$ is the roots of unity of order $5$.
\end{example}

\begin{example}
Let $V \subseteq \mbb{F}^n$ be closed, and for $f \in A_n$ define
\begin{align*}
V_f \coloneqq \set{x \in V}{f\prs{x} \neq 0} = V \setminus V\prs{f} \text{.}
\end{align*}
This is open in $V$ and such a set is called a \emph{principal open set}.
Every open set $U$ is a finite union of such sets, so the principal open sets form a basis for the Zariski topology:

If $U$ is open in $V$ let $W \coloneqq V \setminus U$ so that there are $\prs{f_i}_{i \in [k]} \subseteq A_n$ for which
\begin{align*}
V \setminus U = V \cap W = V\prs{f_1, \ldots, f_k}
\end{align*}
so
\begin{align*}
U = \bigcup_{i \in [k]} V_{f_i} \text{.}
\end{align*}

We sometimes want to think of $V_f$ as closed sets. This can be done by considering
\begin{align*}
\tilde{V}_f \coloneqq \set{\prs{v,y}}{\substack{v \in \mbb{F}^n \\ y \in \mbb{F} \\ f\prs{v} \cdot y = 1}} \subseteq \mbb{F}^{n+1} \text{.}
\end{align*}
There's a clear bijection $V_f \riso \tilde{V}_f$.
\end{example}

\subsubsection{Regular Maps}

\begin{definition}[Regular Map]
For embedded algebraic varieties $V \subseteq \mbb{F}^n$ and $W \subseteq \mbb{F}^m$, a \emph{morphism} $\phi \colon V \to W$, called also a \emph{regular map} is a map of the form
\[\phi\prs{x} = \prs{f_1\prs{x}, \ldots, f_m\prs{x}}\]
for $\prs{f_i}_{i \in \brs{m}} \subseteq A_n$.
\end{definition}

\begin{example}
The map
\begin{align*}
\phi \colon \mbb{F}^2 &\to \mbb{F}^2 \\
\prs{x,y} &\mapsto \prs{xy,y} 
\end{align*}
is a regular map.
\end{example}

\begin{exercise}
A regular map is continuous in the Zariski topology.
\end{exercise}

\begin{definition}
A \emph{regular function} on an embedded algebraic variety $V$ over $\mbb{F}$ is a regular map $V \to \mbb{F}$.
\end{definition}

\begin{remark}
Regular functions on $V$ are of the form $\rest{f}{V}$ for $f \in A_n$. We can think of these as elements of $\mbb{F}\brs{V} \coloneqq A_n / I\prs{V}$.
\end{remark}

\begin{definition}
A regular map $\phi \colon V \to W$ gives an $\mbb{F}$-algebra homomorphism
\begin{align*}
\phi^* \colon \mbb{F}\brs{W} &\to \mbb{F}\brs{V} \\
f &\mapsto f \circ \phi \text{.}
\end{align*}
\end{definition}

\begin{remark}
Sending every $V$ to $\mbb{F}\brs{V}$ and every $\phi$ to $\phi^*$ is a contravariant functor from the category of embedded algebraic varieties to that of finite-dimensional $\mbb{F}$-algebras.
\end{remark}

\begin{exercise}
\begin{enumerate}
\item Consider regular maps
\begin{align*}
\phi_1,\phi_2 \colon V \to W
\end{align*}
such that $\phi_1^* = \phi_2^*$. Show that $\phi_1 = \phi_2$.

\item Show that if $\phi \colon V \to W$ is a regular map such that $\phi^*$ is an isomorphism, $\phi$ is also an isomorphism.
\end{enumerate}
\end{exercise}

\begin{exercise}
Let $\phi \colon V \to W$ be a regular map, and assume $V$ is irreducible. Show that $\overline{\phi\prs{V}}$ is irreducible. 
\end{exercise}

\subsubsection{Revising Affine $\mbb{F}$-Varieties}

For a set $X$ one can consider the algebra of functions $M \coloneqq \hom_{\mathbf{Set}}\prs{X,\mbb{F}}$. Every $x \in X$ defines a homomorphism $\ev_x \colon M \to \mbb{F}$ given by $\ev_x\prs{f} = f\prs{x}$.

\begin{definition}[Affine $\mbb{F}$-Variety]
An \emph{affine $\mbb{F}$-variety} is a pair $\prs{X,A}$ where $X$ is any set and $A \subseteq \hom_{\mathbf{Set}}\prs{X,\mbb{F}}$ such that the following conditions hold.

\begin{enumerate}
\item $A$ is finite-generated.
\item The map
\begin{align*}
X &\to \hom_{\mbb{F}\mathbf{-Alg}}\prs{A, \mbb{F}} \\
x &\mapsto \rest{\ev_x}{A}
\end{align*}
is a bijection.
\end{enumerate}
\end{definition}

%LECTURE 4.11.2021

\begin{remark}
This new definition of an affine $\mbb{F}$-variety is coordinate-free and gives good results even for non-algebraically-closed fields.
\end{remark}

\begin{remark}
We could take any finitely-generated $\mbb{F}$-algebra $A$ and define $X = \hom_{\mbb{F}\mathbf{-Alg}}\prs{A,\mbb{F}}$. This would given an homomorphism $A \to \hom_{\mathbf{Set}}\prs{X,\mbb{F}}$ taking $f \in A$ to the map $x \mapsto x\prs{f}$.
If we require that this map is injective, we get an equivalent definition to that of an affine $\mbb{F}$-variety.
\end{remark}

\begin{definition}[Regular Maps]
Let $\prs{X,A}$ and $\prs{Y,B}$ be affine $\mbb{F}$-varieties. A \emph{regular map} $\phi \colon \prs{X,A} \to \prs{Y,B}$ is a map of sets $\phi \colon X \to Y$ such that for every $f \in B$ it holds that $f \circ \phi \in A$. 
\end{definition}

\begin{definition}[Regular Isomorphism]
A regular map $\phi \colon \prs{X,A} \to \prs{Y,B}$ is an \emph{isomorphism} if there's $\psi \colon \prs{Y,B} \to \prs{X,A}$ such that $\phi \circ \psi = \id$ and $\psi \circ \phi = \id$.
\end{definition}

\begin{remark}
Every homomorphism $\alpha \colon B \to A$ gives a regular map $\phi \colon X \to Y$ by sending $x$ to the $y \in Y$ for which $\ev_x \circ \alpha = \ev_Y$.
\end{remark}

\begin{proposition}
Embedded $\mbb{F}$-affine varieties are $\mbb{F}$-affine varieties.
\end{proposition}

\begin{proof}
Let $V \subseteq \mbb{F}^n$ be an embedded $\mbb{F}$-affine variety. We claim $\prs{V, \mbb{F}\brs{V}}$ is an affine $\mbb{F}$-variety.

$\mbb{F}\brs{V} = A_n/I\prs{V}$ is finitely-generated, so we have to show that all maps $\hom_{\mbb{F}\mathbf{-Alg}}\prs{\mbb{F}\brs{V}, \mbb{F}}$ are of the form $\ev_x$ (since $\mbb{F}\brs{V}$ separate points).
Let $\eps \colon \mbb{F}\brs{V} \to \mbb{F}$, we find $x \in V$ such that $\eps = \ev_x$. Consider the quotient map $\pi \colon A_n \to \mbb{F}\brs{V}$. Then $\eps \circ \pi \colon A_n \to \mbb{F}$. Let $x = \prs{\eps \circ \pi \prs{x_i}}_{i \in [n]} \in \mbb{F}^n$. Now $\ker\prs{\eps \circ \pi}$ is a maximal ideal contained in $I\prs{x}$ so
\begin{align*}
I\prs{V} = \ker\prs{\pi} \subseteq \ker\prs{\eps \circ \pi} = I\prs{x}
\end{align*}
so $x \in V$.
\end{proof}

\begin{remark}
Let $\prs{X,A}$ be an affine $\mbb{F}$-variety. Write $A \cong A_n / I$ and $\pi \colon A_n \riso A_n/I$ the quotient map, and define a map
\begin{align*}
\phi \colon X &\to \mbb{F}^n \\
x &\mapsto \prs{\ev_x \circ \pi\prs{x_i}}_{i \in \brs{n}} \text{.}
\end{align*}
Now $I = \ker\prs{\pi} \subseteq \ker\prs{\ev_p}$ for all $p \in \phi\prs{X}$, so $I \subseteq I\prs{\phi\prs{X}}$.
In the other direction, if $f \in I\prs{\phi\prs{X}}$ then $\ev_x\prs{\pi\prs{f}}$ for all $x \in X$. Therefore $\pi\prs{f} = 0$ so $I\prs{\phi\prs{X}} \subseteq \ker\prs{\pi} = I$.
Then $A \cong A_n / I\prs{\phi\prs{X}} = \mbb{F}\brs{\phi\prs{X}}$.
Then
$\prs{X,A} \cong \prs{\phi\prs{X}, \mbb{F}\brs{\phi\prs{X}}}$.

We want $\phi\prs{X}$ to be closed, which we explain later.
\end{remark}

\begin{exercise}
Check that $\phi$ in the above remark is injective.
\end{exercise}

\begin{definition}[Zariski Topology on an Affine $\mbb{F}$-Variety]
Let $\prs{X,A}$ be an affine $\mbb{F}$-variety. Define the \emph{Zariski topology} on $\prs{X,A}$ by choosing the closed sets to be sets of the form
\[V\prs{C} \coloneqq \set{x \in X}{\forall f \in C \colon f\prs{x} = 0}\]
for $C \subseteq A$.
\end{definition}

\begin{definition}
Let $\prs{X,A}$ be an affine $\mbb{F}$-variety. For $S \subseteq X$ we define
\begin{align*}
I\prs{S} \coloneqq \set{f \in A}{\forall x \in X \colon f\prs{x} = 0} \ideal A \text{.}
\end{align*}
\end{definition}

\begin{remark}
As before, if $Y \subseteq X$ is closed, we have $Y = V\prs{I\prs{Y}}$. Then $Y$ is itself an affine $\mbb{F}$-variety as $\prs{Y, A/I\prs{Y}}$ where $A/I\prs{Y}$ is considered as embedded in $\hom_{\mathbf{Set}}\prs{Y,\mbb{F}}$ by considering the restriction to $Y$.
Elements of
$\hom_{\mbb{F}\mathbf{-Alg}}\prs{A/I\prs{Y}, \mbb{F}}$
are elements of $\hom\prs{A,\mbb{F}} \cong X$ that vanish on $I\prs{Y}$. These are exactly $V\prs{I\prs{Y}} \cong Y$.
\end{remark}

\begin{definition}[Closed Embedding]
A regular map $\phi \colon \prs{X,A} \to \prs{Y,B}$ is called a \emph{closed embedding} if $\im\phi$ is closed and $\rest{\phi}{X}$ is a regular isomorphism.
\end{definition}

\begin{remark}
Requiring that $\phi$ is injective would not suffice. Consider
\begin{align*}
\phi \colon \mbb{F}_p &\to \mbb{F}_p \\
x &\mapsto x^p \text{.}
\end{align*}
We have
\begin{align*}
\phi^* \colon \mbb{F}_p\brs{x} &\to \mbb{F}_p\brs{x} \\
x &\mapsto x^p \text{,}
\end{align*}
which isn't surjective. Hence $\phi$ isn't an isomorphism.
\end{remark}

\begin{proposition}
Let $\phi \colon \prs{X,A} \to \prs{Y,B}$. $\phi$ is a closed embedding if and only if $\phi^* \colon \mbb{F}\brs{Y} \to \mbb{F}\brs{X}$ is surjective.
\end{proposition}

\begin{proof}
\begin{itemize}
\item Assume $\phi$ is a closed embedding. Consider the inclusion $i \colon \phi\prs{X} \to Y$ and write $\phi = i \circ \phi_0$. Then $\phi^* = \phi_0^* \circ i^*$. $\phi_0^*$ is an isomorphism, hence $\phi_0$ is an isomorphism. We're left to show that $i^*$ is surjective.
$\phi\prs{X} \subseteq Y$ is closed and $i^*$ is the quotient map from $\mbb{F}\brs{Y}$ to $\mbb{F}\brs{\phi\prs{X}}$, hence we get the result.

\item
Assume $\phi^*$ is surjective. We show that $\phi\prs{X} = V\prs{\ker\prs{\phi^*}}$ and $\ker\prs{\phi^*} = I\prs{\phi\prs{X}}$.

\begin{description}
\item[$\phi\prs{X} = V\prs{\ker\prs{\phi^*}}$:]
Let $f \in \ker\prs{\phi^*} \subseteq \mbb{F}\brs{Y}$. For every $x \in X$ we have
\[\ev_{\phi\prs{x}}\prs{f} = \ev_x \prs{\phi^*\prs{f}} = \ev_x\prs{0} = 0\text{.}\]
Hence $f \in I\prs{\phi\prs{X}}$. Hence $\ker\prs{\phi^*} \subseteq I\prs{\phi\prs{X}}$ so
$\phi\prs{X} \subseteq V\prs{\ker\prs{\phi^*}}$.

In the other direction, let $y \in V\prs{\ker\prs{\phi^*}}$. Then $\rest{\ev_y}{\ker\prs{\phi^*}} = 0$. Then $\ev_y \colon \mbb{F}\brs{Y} \to \mbb{F}$ factors through $\lambda \colon \mbb{F}\brs{X} \to \mbb{F}$ where $\lambda = \ev_y \circ \phi^*$. Hence there's $x \in X$ such that $\lambda = \ev_x$. Hence $y = \varphi\prs{x} \in \phi\prs{X}$ so $V\prs{\ker\prs{\phi^*}} \subseteq \phi\prs{X}$.

\item[$\ker\prs{\phi^*} = I\prs{\phi\prs{X}}$:]
We saw one inclusion in the previous part. Let $f \in I\prs{\phi\prs{X}}$, we have to show $f \in \ker\prs{\phi^*}$. Indeed, $\varphi^*\prs{f} = f \circ \phi = 0$.
\end{description}
\end{itemize}
\end{proof}

\subsubsection{Back to Algebraic Groups}

In order to define affine algebraic groups, we want the product and inverse maps to be regular. For that, we need to define the product variety. This structure comes from tensor products of $\mbb{F}$-algebras.

Let $\prs{X,A}, \prs{Y,B}$ be affine $\mbb{F}$-varieties. We have an embedding
\begin{align*}
\iota \colon A \otimes B &\to \hom_{\mathbf{Set}}\prs{X \times Y, F} \\
f \otimes g &\mapsto \prs{\prs{x,y} \mapsto f\prs{x} \cdot g\prs{y}} \text{.}
\end{align*}
We show this is injective.
Let $f_1, \ldots, f_k \in A$ linearly independent and $g_1, \ldots, g_\ell \in B$ linearly independent. Using properties of tensor products, we show that for $F = \sum_{\substack{i \in [k] \\ j \in [\ell]}} a_{i,j} f_i \otimes g_j$ such that $\iota\prs{F} = 0$ it holds that $F = 0$.
Indeed, for $y \in Y$ define
\[b_i\prs{y} = \sum_{j \in \brs{\ell}} a_{i,j} g_j\prs{y} \text{.}\]
Then
\[0 = \iota\prs{F} \prs{x,y} = \sum_{i \in \brs{k}} b_i\prs{y} f_i\prs{x}\]
for all $x \in X$. Hence
\[\sum_{i \in [k]} b_i\prs{y} f_i = 0\]
so for every $i \in [k]$ we have $b_i\prs{y} = 0$. Hence
\begin{align*}
0 = b_i = \sum_{j \in \brs{\ell}} a_{i,j} g_j \in B
\end{align*}
so $a_{i,j} = 0$ for every $i \in \brs{k}$ and $j \in \brs{\ell}$.

\begin{definition}[$\mbb{F}$-Algebraic Group]
An \emph{$\mbb{F}$-algebraic group} $G$ is an $\mbb{F}$-affine variety such that the product and inverse maps are regular.
\end{definition}

%LECTURE 10.11.2021

\begin{exercise}
Let $\prs{X,A}$ be an affine $\mbb{F}$-variety. Let $f \in A$, and define the \emph{principle open set}
\[X_f \coloneqq \set{v \in X}{f\prs{v} \neq 0} \subseteq X \text{.}\]
Show that $\prs{X_f, A\brs{\frac{1}{f}}}$ is an affine $\mbb{F}$-variety.
\end{exercise}

\begin{remark}
\begin{enumerate}
\item Consider $\set{0} \subseteq A_n$ and $X = V\prs{\set{0}} = \mbb{F}^n$. Then $I\prs{X} = \prs{0}$ if and only if $\mbb{F}$ is infinite. This is equivalent to $\mbb{F}\brs{X} \cong A_n$.
In particular, when $\mbb{F}$ is finite, $\prs{\mbb{F}, \mbb{F}\brs{x}}$ isn't an affine $\mbb{F}$-variety in our sense.

\item
Consider $\mbb{F} = \overline{\mbb{F}_p}$ and take $X = \mbb{F}$. We have
\begin{align*}
\mbb{F}\brs{X} = \mbb{F}\brs{X}
\end{align*}
and the map
\begin{align*}
\varphi \colon \mbb{F} &\to \mbb{F} \\
v &\mapsto v^p
\end{align*}
is bijective (check this).
The map
\begin{align*}
\varphi^* \colon \mbb{F}\brs{x} &\to \mbb{F}\brs{x} \\
x &\mapsto x^p
\end{align*}
isn't surjective, so $\varphi$ isn't an isomorphism!
\end{enumerate}
\end{remark}

Our current goals are the following.

\begin{enumerate}
\item Every embedded $\mbb{F}$-group is an $\mbb{F}$ algebraic group.

\item Every $\mbb{F}$ algebraic group has an embedding into $\mbb{F}^n$ for some $n \in \mbb{N}_+$.
\end{enumerate}

\begin{proposition}
Every embedded $\mbb{F}$-group is an algebraic group.
\end{proposition}

\begin{proof}
Firstly we notice that a restriction of a regular map to a closed subset is regular. Hence, if $G$ is an algebraic group and $H \leq G$ is closed, $H$ is algebraic.
It therefore suffices to show that $\mrm{GL}_n\prs{\mbb{F}}$ is an $\mbb{F}$-algebraic group.

Now, \[\mrm{GL}_n\prs{\mbb{F}} = \set{A \in M_n\prs{\mbb{F}}}{\det\prs{A} \neq 0} = \prs{\mbb{F}^{n^2}}_{\det}\]
is a principle open set in $\mbb{F}^{n^2}$. We have
\begin{align*}
\mbb{F}\brs{\mrm{GL}_n\prs{\mbb{F}}} = \mbb{F}\brs{\bigcup_{i,j \in [n]} \set{X_{i,j}, \frac{1}{\det\prs{X_{i,j}}}}} \text{.}
\end{align*}
Let $G \coloneqq \mrm{GL}_n\prs{\mbb{F}}$ and $m \colon G \times G \to G$ be the multiplication map.
Then
\begin{align*}
m^*\prs{X_{i,j}} = \sum_{k\in[n]} X_{i,k} \otimes X_{k,j} \in \mbb{F}\brs{G\times G} \cong \mbb{F}\brs{G} \otimes \mbb{F}\brs{G} \text{.}
\end{align*}
From multiplicativity of $\det$ we get
\begin{align*}
m^*\prs{\frac{1}{\det\prs{X_{i,j}}}} = \frac{1}{\det\prs{X_{i,j}}} \otimes \frac{1}{\det\prs{X_{i,j}}} \text{.}
\end{align*}

Let $i \colon G \to G$ be the inverse map. Cramer's rule described the inverse of a matrix as a polynomial by the coefficients, hence the inverse map is also regular.
\end{proof}

\begin{definition}The [Multiplicative Group]
Define \emph{the multiplicative group} over $\mbb{F}$ as $\mbb{G}_m\prs{\mbb{F}} \coloneqq \mrm{GL}_1\prs{\mbb{F}}$.
\end{definition}

\begin{exercise} \label{exercise:irreducible_product}
The product of irreducible algebraic varieties is irreducible.
\end{exercise}

\begin{proposition}
Let $G$ be an $\mbb{F}$ algebraic group. Let $G^\circ \subseteq G$ be the irreducible component of $G$ containing the identity element $e$ of $G$. Then

\begin{enumerate}
\item $G^\circ \ideal G$ is a normal subgroup of $G$ of finite index.

\item Every closed subgroup $H \leq G$ of finite index contains $G^\circ$.

\item $G = G^\circ$ if and only if $G$ is connected. \label{connected_group_component}
\end{enumerate}
\end{proposition}

\begin{proof}
\begin{enumerate}
\item By exercise \ref{exercise:irreducible_product}, $G^\circ \times G^\circ \subseteq G \times G$ is irreducible. Hence $e = e \cdot e \in \overline{G^\circ \cdot G^\circ}$. Since $G^\circ \subseteq \overline{G^\circ \cdot G^\circ}$ we get $G^\circ = \overline{G^\circ \cdot G^\circ}$. Hence $G^\circ$ is closed to multiplication.
Now, $g \mapsto g^{-1}$ is a homeomorphism so $\prs{G^\circ}^{-1}$ is irreducible.
Now $\overline{G_0 \cdot \prs{G_0}^{-1}}$ is irreducible containing $e$ so one gets $G_0 = \overline{G_0 \cdot \prs{G_0}^{-1}} = \prs{G_0}^{-1}$.

For every $x \in G$, the map $\lambda_x \colon G \to G$ given by $x \mapsto xy$ is a homeomorphism. Every coset $xG^\circ$ is an irreducible component of $G$ and $\bigsqcup_{x \in G} x G^\circ$. In an affine variety, there are finitely many irreducible components, hence $G^\circ$ has finitely many cosets in $G$. This proves part \ref{connected_group_component} as well.

We're left to show that $G^\circ$ is normal.
Indeed, conjugation $y \mapsto xyx^{-1}$ is a homeomorphism. For every $x \in G$, $x G^\circ x^{-1}$ is an irreducible component. Hence $e \in G^\circ x^{-1}$ implies $G_0 = x G_0 x^{-1}$.

\item
If $H \leq G$ is a closed subgroup of finite index, we can write $G = \bigcup_{x \in G} xH$ which is a finite disjoint union of the different cosets.
Then
$G^\circ = \bigcup_{x \in G} \prs{xH \cap G^\circ}$.
From irreducibility there's $x \in G$ such that $G^\circ = xH \cap G^\circ$. Then $e \in G^\circ \subseteq xH = H$.
\end{enumerate}
\end{proof}

%LECTURE 11.11.21

\begin{example}
$\mrm{GL}_1\prs{\mbb{R}}$ isn't connected as a Lie group (with the Euclidean topology from $\mbb{R}$, but \emph{is} connected as an algebraic group.
\end{example}

\begin{example}
Finite non-trivial groups aren't connected, since $G^\circ = \set{1}$.
\end{example}

\begin{example}
Consider the group
\[\mrm{O}_n\prs{\mbb{F}} \coloneqq \set{M_n\prs{\mbb{F}}}{g^t Q g = Q}\]
where $Q = \pmat{0 & & 1  \\ & \rddots & \\ 1 & & 0}$.
For $g \in O_n\prs{\mbb{F}}$ we get
\begin{align*}
\det\prs{g}^2 \det\prs{Q} &= \det\prs{g^t Q g}
\\&=
\det\prs{Q}
\end{align*}
so $\det\prs{g} \in \set{\pm 1}$.
$\det \colon \mrm{O}_n\prs{\mbb{F}} \to \set{\pm 1}$ is a homomorphism.
Now, $\mrm{SO}_n\prs{\mbb{F}} \coloneqq \mrm{O}_n\prs{\mbb{F}} \cap \mrm{SL}_n\prs{\mbb{F}}$ is a subgroup of index $2$ (or $1$). For this, it suffices to show that there's $g \in \mrm{O}_n\prs{\mbb{F}}$ such that $\det\prs{g} = -1$, which is the cast when $1+1 \neq 0$ (otherwise, the index is $1$).
Usually, $\mrm{SO}_n\prs{\mbb{F}} = \mrm{O}_n\prs{\mbb{F}}^\circ$. When $n=2$ and $\mbb{F}$ is algebraically-closed we get
\begin{align*}
\mrm{GL}_1\prs{\mbb{F}} \cong \mrm{SO}_2\prs{\mbb{F}} = \set{\pmat{a & 0 \\ 0 & a^{-1}}}{a \in \mbb{F} \setminus \set{0}}
\end{align*}
which is connected. Then there's \[g = \pmat{0 & 1 \\ 1 & 0} \in \mrm{O}_2\prs{\mbb{F}} \setminus \mrm{SO}_2\prs{\mbb{F}}\text{.}\]
\end{example}

\subsubsection{Group Actions}

We want to show that every $\mbb{F}$-algebraic group is isomorphic to a closed subgroup of $\mrm{GL}_n\prs{\mbb{F}}$. To every $G$, we later find a rational representation $\varphi \colon G \to \mrm{GL}_n\prs{\mbb{F}}$ (i.e. with $\varphi$ regular) such that $\varphi$ is a closed embedding.
For this, we look at actions of algebraic groups.

\begin{definition}[$\mbb{F}$-Affine $G$-Space]
Let $\prs{X,A}$ be an affine $\mbb{F}$-variety. We say $X$ is an \emph{$\mbb{F}$-affine $G$-space} if it has a regular map $G \times X \to X$ for an $\mbb{F}$-affine algebraic group $G$.
\end{definition}

\begin{example}
Take $X = G$. Then $G$ acts on $X$ by left multiplication $x \mapsto gx$, by right multiplication $x \mapsto x g^{-1}$ or by conjugation $x \mapsto gxg^{-1}$.
\end{example}

For $X$ an $\mbb{F}$-affine $G$-space, we get an action of $G$ on $\mbb{F}\brs{X}$.
Let $f \in \mbb{F}\brs{X}$ and $g \in G$. We have an action
\begin{align*}
\prs{gf} \prs{x} = f\prs{g^{-1} \cdot x} \text{.}
\end{align*}
The map
\begin{align*}
\mbb{F}\brs{X} &\to \mbb{F}\brs{x} \\
f &\mapsto gf
\end{align*}
is $\mbb{F}$-linear.
We get a map
\begin{align*}
G \to \mrm{GL}\prs{\mbb{F}\brs{X}} \text{.}
\end{align*}
Writing the $G$-action as
\begin{align*}
a \colon G \times X &\to X
\end{align*}
we get
\begin{align*}
a^* \colon \mbb{F}\brs{X} &\to \mbb{F}\brs{G} \otimes \mbb{F}\brs{X} \text{.}
\end{align*}
Now,
\begin{align*}
g \circ f = \prs{\ev_{g^{-1}} \otimes \id}\prs{a^*\prs{f}} \text{.}
\end{align*}

\begin{proposition}
\begin{enumerate}
\item If $V \subseteq \mbb{F}\brs{X}$ is a finite-dimensional $G$-invariant subspace, then
\begin{align*}
a^*\prs{V} \subseteq \mbb{F}\brs{G} \otimes V \text{,}
\end{align*}
and the resulting homomorphism $\varphi \colon G \to \mrm{GL}\prs{V}$ is regular (given any choice of basis for $V$).

\item For every finite-dimensional $W \subseteq \mbb{F}\brs{X}$ there's $V \subseteq \mbb{F}\brs{X}$ which contains $W$ and is $G$-invariant.
\end{enumerate}
\end{proposition}

\begin{proof}
\begin{enumerate}
\item Let $V \subseteq \mbb{F}\brs{X}$ be finite-dimensional and $G$-invariant.
Choose $\set{r_1, r_2, \ldots} \subseteq \mbb{F}\brs{X}$ linearly independent such that $\mbb{F}\brs{X} = V \oplus \spn\set{r_1, r_2, \ldots}$.
For $f \in V$ we can write
\[a^*\prs{f} = s + \sum_{i \in [k]} u_i \otimes r_{j_i}\]
for $s \in \mbb{F}\brs{G} \otimes V$ and $u_i \in \mbb{F}\brs{G}$.
Since $g^{-1}\cdot f \in V$ we get
\begin{align*}
\prs{\ev_v \otimes \id} \prs{a^*\prs{f}} \in V
\end{align*}
so
\begin{align*}
\sum_{i \in [k]} u_i\prs{g} r_{j_i} = 0 \in \mbb{F}\brs{X}
\end{align*}
for all $g \in G$.
By linear independency we get $u_i\prs{g} = 0$ for all $i$, so $u_i \equiv 0$.
Hence $a^*\prs{f} \in \mbb{F}\brs{G} \otimes V$.

Choose a basis $\prs{f_1, \ldots, f_n}$ for $V$. We get $a^*\prs{f_i} = \sum_{j \in [n]} m_{j,i} f_j$ for $m_{i,j} \in \mbb{F}\brs{G} \otimes V$.
If we look at the action of $G$ on $V$ and write the map $\varphi \colon G \to \mrm{GL}_n\prs{\mbb{F}}$ given by choosing the basis $\prs{f_1, \ldots, f_n}$, we get
\[\varphi\prs{g} = \prs{m_{j,i} \prs{g^{-1}}}\]
and
\[g \cdot f_i = \sum_{j \in [k]} m_{j,i} \prs{g^{-1}} f_j \text{.}\]
Then, the matrix coefficients of $\varphi\prs{g}$ are given by regular functions.

\item Take a basis $\prs{h_1, \ldots, h_k}$ of $W$ and examine the finite sum
$a^*\prs{h_i} = \sum_j u_j \otimes f_{i_j}$.
We have
\[g \cdot h_i = \sum_j u_j\prs{g^{-1}} f_{i,j} \in \mbb{F}\brs{X} \text{.}\]
Take $V' = \spn\set{f_{i,j}}_{i,j} \subseteq \mbb{F}\brs{X}$ which is finite-dimensional.
Then
\[V = \spn\set{g \cdot f}{\substack{g \in G \\ f \in \mbb{F}\brs{X}}} \subseteq V'\]
is a $G$-invariant subspace.
\end{enumerate}
\end{proof}

\begin{proposition}
Let $G$ be an $\mbb{F}$-algebraic group. There's a regular homomorphism $\varphi \colon G \to \mrm{GL}_n\prs{\mbb{F}}$ which is a closed embedding.
\end{proposition}

\begin{proof}
Consider the right action of $G$ on itself. Take generators $f_1, \ldots, f_k \in \mbb{F}\brs{G}$ of the $\mbb{F}$-algebra $\mbb{F}\brs{G}$. From the previous proposition, there's a finite-dimensional $G$-invariant $V \subseteq \mbb{F}\brs{G}$ under the induced action on $\mbb{F}\brs{G}$, which contains each $f_i$.

Choose a basis $\prs{e_1, \ldots, e_n}$ of $V$. We get a homomorphism $\varphi \colon G \to \mrm{GL}_n\prs{\mbb{F}} \cong \mrm{GL}\prs{V}$ as in the previous proposition. We want to show that
\begin{align*}
\varphi^* \colon \mbb{F}\brs{\mrm{GL}_n\prs{\mbb{F}}} \to \mbb{F}\brs{G}
\end{align*}
is surjective.
We may assume $\varphi\prs{g} = \prs{m_{j,i}\prs{g^{-1}}}$ for $m_{i,j} \in \mbb{F}\brs{G}$ for which
\begin{align*}
\forall x,g \in G \colon e_i\prs{xg} = \sum_j m_{j,i}\prs{g^{-1}} e_j\prs{x} \text{.}
\end{align*}
Taking $x = e$ we get
\begin{align*}
e_i\prs{g} &= \sum_j m_{j,i} e_j\prs{e} m_{j,i}\prs{g^{-1}} \text{.}
\end{align*}
Write $\tilde{m}_{i,j}\prs{g} = m_{j,i}\prs{g^{-1}}$.
Then
\[e_i = \sum_j e_j\prs{e} \tilde{m}_{j,i} \in \mbb{F}\brs{G}\]
but
$\tilde{m}_{i,j} = \varphi^*\prs{T_{i,j}}$.
Hence $e_i \in \im\prs{\varphi}$.
Hence $f_j \in \im\prs{\varphi^*}$ and these generate $\mbb{F}\brs{G}$ so $\im\prs{\varphi^*} = \mbb{F}\brs{G}$.
\end{proof}

Consider a homomorphism $\varphi \colon G \to H$ of $\mbb{F}$-algebraic groups. The kernel $\ker\prs{\varphi} = \varphi^{-1}\prs{e} \leq G$ is a closed subgroup of $G$. A question we would like to ask whether or not $\im\prs{\varphi}$ is closed. This happens to be true when $\mbb{F}$ is algebraically-closed, but requires use of an intricate result in algebraic geometry.

\begin{remark}
In Lie groups, the situation is different than that mentioned above. Consider the map
\begin{align*}
\varphi \colon \mbb{R} &\to \mrm{GL}_4\prs{\mbb{R}} \\
\theta &\mapsto \pmat{\cos \theta & \sin \theta  & 0 & 0 \\ -\sin \theta & \cos \theta & 0 & 0 \\ 0 & 0 & \cos\prs{\alpha \theta} & \sin\prs{\alpha \theta} \\ 0 & 0 & -\sin\prs{\alpha \theta} & \cos\prs{\alpha \theta}} \text{.}
\end{align*}
If $\alpha / 2 \pi$ is irrational, this is an embedding of $\mbb{R}$ which isn't closed. Moreover, $\im\varphi$ has an empty interior.
\end{remark}

We later show that if $\varphi \colon X \to Y$ is regular between $\mbb{F}$-affine varieties and $\mbb{F}$ is algebraically-closed, there's an open $U \subseteq \overline{\varphi\prs{X}}$ such that $U \subseteq \varphi\prs{X}$.

\backmatter
\end{document}